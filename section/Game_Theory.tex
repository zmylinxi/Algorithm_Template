

\section{博弈论}

\subsection{Bash 博弈}

有一堆石子共有N个。A B两个人轮流拿,A先拿。每次最少拿1颗,最多拿K颗,拿到最后1颗石子的人获胜。假设A B都非常聪明,拿石子的过程中不会出现失误。给出 N 和 K,问最后谁能赢得比赛。

Bash 博弈先手必胜当且仅当 $N \% (K + 1) = 0$

\subsection{Nim 博弈}

有 N 堆石子。A B 两个人轮流拿,A 先拿。每次只能从一堆中取若干个,可将一堆全取走,但不可不取,拿到最后 1 颗石子的人获胜。假设A B都非常聪明,拿石子的过程中不会出现失误。给出N及每堆石子的数量,问最后谁能赢得比赛。

Nim 博弈先手必胜,当且仅当 $X_1 xor X_2 xor …… xor X_n \neq 0$

\subsection{Wythoff 博弈}

有 2 堆石子。A B 两个人轮流拿,A先拿。每次可以从一堆中取任意个或从2堆中取相同数量的石子,但不可不取。拿到最后1颗石子的人获胜。假设A B都非常聪明,拿石子的过程中不会出现失误。给出2堆石子的数量,问最后谁能赢得比赛。

\begin{lstlisting}[language=C++]
void Wythoff(int n, int m){
    if(n > m) swap(n, m);
    int tmp = (m - n)*(sqrt(5) + 1)/2;
    if(n == tmp) puts("B");
    else puts("A");
}
\end{lstlisting}

\subsection{公平组合游戏}

若一个游戏满足:

\begin{enumerate}
    \item 游戏由两个人参与,二者轮流做出决策
    \item 在游戏进程的任意时刻,可以执行的合法行动与轮到哪名玩家无关
    \item 有一个人不能行动时游戏结束
\end{enumerate}

则称这个游戏是一个公平组合游戏(Impartial Combinatorial Games, ICG),NIM游戏就是一个公平组合游戏

\subsubsection{SG-组合游戏}

一个公平组合游戏若满足:

\begin{enumerate}
    \item 两人的决策都对自己最有利
    \item 当有一人无法做出决策时游戏结束,无法做出决策的人输,且游戏一定能在有限步数内结束
    \item 游戏中的同一个状态不可能多次抵达,且游戏不会出现平局
\end{enumerate}

则这类游戏可以用SG函数解决,我们称之为SG-组合游戏

\subsubsection{Sprague-Grundy 函数}

SG函数是对游戏图中每一个节点的评估函数。

即一个状态的SG函数值等于mex{其所有后继状态的SG函数值}

其中,mex(minimal excludant)是定义在整数集合上的操作,它的自变量是任意整数集合,函数值是不属于该集合的最小自然数

f(X) == 0 当且仅当X为必败态

\subsubsection{游戏的和}

游戏的和是指:

考虑多个同时进行的SG-组合游戏,这些SG-组合游戏的和是这样的一个SG-组合游戏,在它进行的过程中,游戏者可以任意挑选其中任意一个游戏进行决策

\subsubsection{Sprague-Grundy 定理}

游戏的和的SG函数值等于它包含的各个子游戏SG函数值得异或和。即

$$SG(G) = SG(G_1) xor SG(G_2) xor \cdots xor SG(G_m)$$

\subsubsection{Anti-SG 游戏}

定义:

Anti-SG 游戏规定,决策集合为空的游戏者赢。

Anti-SG 游戏的其他规则与 SG 游戏相同。

SG 定理:

对于任意一个 Anti-SG 游戏,如果我们规定当局面中所有的单一游戏的 SG 值为 0 时游戏结束,则先手必胜当且仅当:

(1)游戏的 SG 函数不为 0 且游戏中某个单一游戏的 SG 函数值大于 1
(2)游戏的 SG 函数为 0 且游戏中没有单一游戏的 SG 函数值大于 1

\subsection{删边游戏}

\subsubsection{树的删边游戏}

给出一个有 N 个点的树,有一个点作为树的根节点。

游戏者轮流从树中删边,删去一条边后,不与根节点相连的部分将被移走。

无法行动者输。

有如下定理:叶子节点的 SG 值为 0;其它节点的 SG 值为它的所有子节点的 SG 值加 1 后的异或和。

\subsubsection{无向图删边游戏}

一个无向连通图,有一个点作为图的根。

游戏者轮流从图中删去边,删去一条边后,不与根节点相连的部分被移走,无法行动者输。

Fusion Principle :
       
我们可以对无向图做如下改动:将图中的任意一个偶环缩成一个新点,任意一个奇环缩成一个新点加一个新边;所有连到原先环上的边全部改为与新点相连。这样的改动不会影响图的SG值。

这样我们就可以将任意一个无向图改成树结构。
