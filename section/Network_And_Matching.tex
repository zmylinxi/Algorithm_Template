

\section{网络流与匹配问题}

\subsection{二分图匹配}

\subsubsection{KM 算法 $O(n ^ 3)$}

\begin{lstlisting}[language=C++]
const int maxn = 2e2 + 10 ; // 点的个数
const int inf = 0x3f3f3f3f;
int lx[maxn], ly[maxn], slack[maxn];
int match[maxn];
int g[maxn][maxn];
int X, Y; // X点集大小,Y点集大小
bool visx[maxn], visy[maxn];
void init(){
    for(int i = 1; i <= X; i++) // 根据X点集点的编号修改循环起始
        for(int j = 1; j <= Y; j++) // 根据Y点集点的编号修改循环起始
            g[i][j] = -inf; // 根据题意更改初始化值
    return ;
}
// 调用初始化函数后读图,由于使用邻接矩阵存图,根据题意判重边,处理重边信息
int path(int x){
    visx[x] = true;
    for(int i = 1; i <= Y; i++) {
        if(visy[i])
            continue;
        int tmp = lx[x] + ly[i] - g[x][i];
        if(tmp == 0){
            visy[i] = true;
            if(match[i] == -1 || path(match[i])){
                match[i] = x;
                return true;
            }
        }
        else
            slack[i] = min(slack[i], tmp);
    }
    return false;
}
int KM(){ // 返回二分图权值和最大的完备匹配
    memset(ly, false, sizeof(ly));
    memset(match, -1, sizeof(match));
    for(int i = 1; i <= X; i++) { // 根据X点集点的编号修改循环起始
        lx[i] = -inf;
        for(int j = 1; j <= Y; j++){ // 根据Y点集点的编号修改循环起始
            lx[i] = max(lx[i], g[i][j]);
        }
    }
    for(int i = 1; i <= X; i++) { // 根据X点集点的编号修改循环起始
        memset(slack, inf, sizeof(slack));
        while(true) {
            memset(visx, false, sizeof(visx));
            memset(visy, false, sizeof(visy));
            if(path(i))
                break;
            int d = inf;
            for(int j = 1; j <= Y; j++){ // 根据Y点集点的编号修改循环起始
                if(!visy[j])
                    d = min(d, slack[j]);
            }
            for(int j = 1; j <= Y; j++){ // 根据Y点集点的编号修改循环起始
                if(visx[j])
                    lx[j] -= d ;
            }
            for(int j = 1; j <= Y; j++){ // 根据Y点集点的编号修改循环起始
                if(visy[j])
                    ly[j] += d;
                else
                    slack[j] -=d;
            }
        }
    }
    int ans = 0;
    for(int i = 1; i <= Y; i++) // 根据Y点集点的编号修改循环起始
        ans += g[match[i]][i]; // 注意根据是否可以完全匹配加判断语句
    return ans;
}
\end{lstlisting}

\subsubsection{HK 算法 (非稠密图 $O(n\sqrt{n})$)}

\begin{lstlisting}[language=C++]
const int maxn = 2e5 + 5;
int Nl, Nr, now;
int matchl[maxn], matchr[maxn], levell[maxn], levelr[maxn], visited[maxn];
vector<int> g[maxn];
int q[maxn];
bool bfs() {
    int* begin(q);
    int* end(q);
    bool ret = false;
    for (int i = 1; i <= Nl; ++i) {
        if (matchl[i]) {
            levell[i] = 0;
        } else {
            levell[i] = 1;
            *end++ = i;
        }
    }
    for (int i = 1; i <= Nr; ++i) levelr[i] = 0;
    while (begin != end) {
        int u = (*begin++);
        int sz = g[u].size();
        for (int i = 0; i < sz; ++i) {
            int to = g[u][i];
            if (levelr[to] == 0) {
                levelr[to] = levell[u] + 1;
                if (matchr[to]) {
                    levell[matchr[to]] = levelr[to] + 1;
                    *end++ = matchr[to];
                }
                else
                    ret = true;
            }
        }
    }
    return ret;
}
int dfs(int u) {
    int sz = g[u].size();
    for (int i = 0; i < sz; ++i) {
        int to = g[u][i];
        if (levelr[to] == levell[u] + 1 && visited[to] != now) {
            visited[to] = now;
            if (matchr[to] == 0 || dfs(matchr[to])) {
                matchr[to] = u; matchl[u] = to;
                return 1;
            }
        }
    }
    return 0;
}
inline int HK() {
    int ret = 0;
    while(bfs()) {
        ++now;
        for(int i = 1; i <= Nl; ++i)
            if(!matchl[i]) ret += dfs(i);
    }
    return ret;
}
void clr() {
    for(int i = 1; i <= Nl; ++i) {
        g[i].clear();
        matchl[i] = 0;
        levell[i] = 0;
    }
    now = 0;
    for(int i = 1; i <= Nr; ++i) {
        matchr[i] = 0;
        levelr[i] = 0;
        visited[i] = 0;
    }
}
\end{lstlisting}

\subsection{最大流}

\subsubsection{最大流(Dinic 算法)}

\begin{lstlisting}[language=C++]
const int maxn = 1e5 + 5;
const int maxm = 2e5 + 5;
const int inf = 0x3f3f3f3f;
struct Edge {
    int to, nxt;
    int w;
} E[maxm << 1];
int head[maxn], tot;
int n, m, s, t; // vertex number, edge number, source id, sink id
void init_edge() {
    // 1 ~ n
    for(int i = 1; i <= n; ++i) head[i] = -1;
    tot = 0;
}
// undirected edge: add_edge(u, v, w); add_edge(u, v, w);
// directed edge: add_edge(u, v, w); add_edge(u, v, 0);
void add_edge(int u, int v, int w) {
    E[tot] = {v, head[u], w};
    head[u] = tot++;
}
int dep[maxn];
bool bfs() {
    if(s == t) return true;
    for(int i = 1; i <= n; ++i) dep[i] = 0;
    queue<int> q;
    dep[s] = 1; q.push(s);
    while(!q.empty()) {
        int x = q.front(); q.pop();
        for(int i = head[x]; i != -1; i = E[i].nxt) {
            int to = E[i].to; int w = E[i].w;
            if(w == 0 || dep[to] != 0) continue;
            dep[to] = dep[x] + 1;
            q.push(to);
            if(to == t) return true;
        }
    }
    return false;
}
int cur[maxn];
int dfs(int x, int flow) {
    if(x == t) return flow;
    int ret = 0;
    for(int &i = cur[x]; i != -1; i = E[i].nxt) {
        int to = E[i].to; int w = E[i].w;
        if(w == 0 || dep[to] != dep[x] + 1) continue;
        int now = dfs(to, min(flow - ret, w));
        if(now > 0) {
            E[i].w -= now;
            E[i ^ 1].w += now;
            ret += now;
            if(ret == flow) break;
        }
    }
    return ret;
}
int dinic() {
    int ret = 0;
    while(bfs()) {
        for(int i = 1; i <= n; ++i) cur[i] = head[i];
        ret += dfs(s, inf);
    }
    return ret;
}
\end{lstlisting}

\subsubsection{最小费用最大流(dinic + SPFA)}

\begin{lstlisting}[language=C++]
const int maxn = 1e5 + 5;
const int maxm = 2e5 + 5;
const int inf = 0x3f3f3f3f;
struct Edge {
    int to, nxt;
    int cost, flow;
} E[maxm << 1];
int head[maxn], tot;
int n, m, s, t; // point, edge, source, sink
int min_cost, max_flow;
void init_edge() {
    for(int i = 1; i <= n; ++i) { // 1 ~ n
        head[i] = -1;
    }
    tot = 0;
}
void add_edge(int u, int v, int cost, int flow) {
    E[tot] = {v, head[u], cost, flow};
    head[u] = tot++;
}
void add_network_edge(int u, int v, int cost, int flow) {
    add_edge(u, v, cost, flow);
    add_edge(v, u, -cost, 0);
}
int dist[maxn];
bool inq[maxn];
bool spfa() {
    for(int i = 1; i <= n; ++i) {
        dist[i] = inf;
        inq[i] = false;
    }
    queue<int> q;
    q.push(s);
    dist[s] = 0; inq[s] = true;
    while(!q.empty()) {
        int x = q.front(); q.pop();
        for(int i = head[x]; i != -1; i = E[i].nxt) {
            int to = E[i].to, d = E[i].cost + dist[x], f = E[i].flow;
            if(f == 0 || d >= dist[to]) continue;
            dist[to] = d;
            if(!inq[to]) {
                inq[to] = true;
                q.push(to);
            }
        }
        inq[x] = false;
    }
    if(dist[t] == inf) return false;
    return true;
}
int cur[maxn];
bool vis[maxn];
int dfs(int x, int flow) {
    if(x == t) {
        max_flow += flow;
        return flow;
    }
    vis[x] = true;
    int ret = 0;
    for(int &i = cur[x]; i != -1; i = E[i].nxt) {
        int to = E[i].to, d = dist[x] + E[i].cost, f = E[i].flow;
        if(vis[to] || f == 0 || d != dist[to]) continue;
        int now = dfs(to, min(flow - ret, f));
        if(now > 0) {
            E[i].flow -= now;
            E[i ^ 1].flow += now;
            min_cost += E[i].cost*now;
            ret += now;
            if(ret == flow) break;
        }
    }
    return ret;
}
void min_cost_max_flow() {
    min_cost = 0; max_flow = 0;
    while(spfa()) {
        for(int i = 1; i <= n; ++i) {
            vis[i] = false;
            cur[i] = head[i];
        }
        dfs(s, inf);
    }
}
\end{lstlisting}

\subsection{一般图最大匹配(带花树开花算法)}

第一行一个整数,表示最多产生多少个小组。

接下来一行 n 个整数,描述一组最优方案。第 v 个整数表示 v 号男生所在小组的另一个男生的编号。如果 v 号男生没有小组请输出 0。

\begin{lstlisting}[language=C++]
#include <bits/stdc++.h>
using namespace std;
const int maxn = 5e2 + 5;
const int maxm = maxn*(maxn - 1)/2;
struct Edge {
    int to, nxt;
} E[maxm << 1];
int head[maxn], tot;
void init_edge(int n) {
    for(int i = 0; i <= n; ++i) head[i] = 0;
    tot = 0;
}
void add_edge(int x, int y) {
    E[++tot] = {y, head[x]};
    head[x] = tot;
}
void add(int x, int y) {
    add_edge(x, y);
    add_edge(y, x);
}
int *Top;
int Stp; // time stamp
int n; // vertex number: [1, n]
// Nxt[]: match id
int Pre[maxn], Nxt[maxn], S[maxn], Q[maxn], Vis[maxn];
int fa[maxn];
void init_match(int n) {
    init_edge(n);
    Top = Q;
    Stp = 0;
    for(int i = 0; i <= n; ++i) {
        fa[i] = i;
        Pre[i] = Nxt[i] = Vis[i] = 0;
    }
}
int findfa(int x) {
    if(fa[x] != x) fa[x] = findfa(fa[x]);
    return fa[x];
}

int lca(int x, int y) {
    for(Stp++, x = findfa(x), y = findfa(y); ; x ^= y ^= x ^= y) {
        if(x) {
            if(Vis[x] == Stp) return x;
            Vis[x] = Stp;
            x = findfa(Pre[Nxt[x]]);
        }
    }
}

void blossom(int x, int y, int l) {
    while(findfa(x) != l) {
        Pre[x] = y, y = Nxt[x];
        assert(x != -1);
        S[*Top = y] = 0, *Top++;
        if(fa[x] == x) fa[x] = l;
        if(fa[y] == y) fa[y] = l;
        x = Pre[y];
    }
}

int match(int x) {
    for(int i = 1; i <= n; ++i) fa[i] = i;
    memset(S, -1, sizeof(S));
    S[*(Top = Q) = x] = 0, Top++;
    for(int *i = Q; i != Top; *i++)
        for(int T = head[*i]; T != 0; T = E[T].nxt) {
            int g = E[T].to;
            if(S[g] == -1) {
                Pre[g] = *i, S[g] = 1;
                if(Nxt[g] == 0) {
                    for(int u = g, v = *i, lst; v; u = lst, v = Pre[u])
                        lst = Nxt[v], Nxt[v] = u, Nxt[u] = v;
                    return 1;
                }
                S[*Top = Nxt[g]] = 0, Top++;
            } else if(!S[g] && findfa(g) != findfa(*i)) {
                int l = lca(g, *i);
                blossom(g, *i, l);
                blossom(*i, g, l);
            }
        }
    return 0;
}
int max_match() {
    int ret = 0;
    for(int i = n; i >= 1; --i) {
        if(!Nxt[i]) ret += match(i);
    }
    return ret;
}
int main() {
    int m;
    scanf("%d%d", &n, &m);
    init_match(n);
    for(int i = 1; i <= m; ++i) {
        int x, y; scanf("%d%d", &x, &y);
        add(x, y);
    }
    int ans = max_match();
    printf("%d\n", ans);
    for(int i = 1; i <= n; i++) {
        if(i > 1) printf(" ");
        printf("%d", Nxt[i]);
    }
    puts("");
    return 0;
}
\end{lstlisting}

\subsection{平面图相关}

\subsubsection{平面图的性质}

平面图的欧拉公式:

如果一个连通的平面图有 n 个点, m 条边和 f 个面,那么 f = m - n + 2

\subsubsection{平面图 G 与其对偶图 G*}

G*中的每个点对应G中的一个面

对于G中的每条边e
e属于两个面f1、f2,加入边(f1*,f2*)
e只属于一个面f,加入回边(f*,f*)

G的面数等于G*的点数,G*的点数等于G的面数,G与G*边数相同

G*中的环与G中的割一一对应

\subsubsection{s-t 平面图}

原图为平面图

图中的一个点为源点 s,另外一个点为汇点 t,且 s 和 t 都在图中的无界面的边界上

\subsubsection{应用 s-t 平面图求最小割}

在平面图中的起点和终点之间加一条边

将平面图(即原来的图)转化为对偶图,记起点和终点构成的边组成的面为源点,无界面为汇点,每条边的权等于原图中对应边的权

删去源点和汇点之间的边

在对偶图上求最短路即为原图的最大流(或最小割)
