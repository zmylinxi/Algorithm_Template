

\section{数据结构}

\subsection{线段树}

\subsubsection{动态开点线段树}

\begin{lstlisting}[language=C++]
struct SegTree {
    int l, r;
    int ls, rs;
    int sum, lazy;
} st[maxn];
int stc = 0; // Segment Tree counting number
int root; // root of Segment Tree
void pushUp(int x) {
    SegTree& now = st[x];
    SegTree& ls = st[now.ls];
    SegTree& rs = st[now.rs];
    now.sum = ls.sum + rs.sum;
    now.l = ls.l;
    now.r = rs.r;
}
void build(int lf, int rt, int x) {
    SegTree& now = st[x];
    if(lf == rt) {
        now.ls = now.rs = now.lazy = -1;
        now.l = now.r = lf;
        return ;
    }
    int mid = (lf + rt)>>1;
    now.ls = stc++; now.rs = stc++;
    build(lf, mid, now.ls);
    build(mid + 1, rt, now.rs);
    pushUp(x);
}
void pushDown(int x) {
    SegTree& now = st[x];
    SegTree& ls = st[now.ls];
    SegTree& rs = st[now.rs];
    ls.sum = now.lazy*(ls.r - ls.l + 1);
    rs.sum = now.lazy*(rs.r - rs.l + 1);
    ls.lazy = now.lazy;
    rs.lazy = now.lazy;
    now.lazy = -1;
}
void update(int lf, int rt, int c, int x) {
    SegTree& now = st[x];
    if(lf <= now.l && now.r <= rt) {
        now.sum = c*(now.r - now.l + 1);
        now.lazy = c;
        return ;
    }
    if(now.lazy != -1) pushDown(x);
    int mid = (now.l + now.r)>>1;
    if(lf <= mid) update(lf, rt, c, now.ls);
    if(rt > mid) update(lf, rt, c, now.rs);
    pushUp(x);
}
int query(int lf, int rt, int x) {
    SegTree& now = st[x];
    if(lf <= now.l && now.r <= rt) return now.sum;
    if(now.lazy != -1) pushDown(x);
    int ret = 0, mid = (now.l + now.r)>>1;
    if(lf <= mid) ret += query(lf, rt, now.ls);
    if(rt > mid) ret += query(lf, rt, now.rs);
    return ret;
}

int main() {
    root = stc++;
    build(1, n, root);

    return 0;
}
\end{lstlisting}

\subsubsection{区间最大字段和}

有 n 个数,a[1] 到 a[n]。

接下来 q 次查询,每次动态指定两个数 l, r,求 a[l] 到 a[r] 的最大子段和。

子段的意思是连续非空区间。

\begin{lstlisting}[language=C++]
const int maxn = 5e4 + 5;
ll a[maxn];
struct node{
    int lf, rt;
    ll sum;
    ll mxl, mxr, mxa;
} tree[maxn << 2];
void pushUp(int x) {
    node& ls = tree[x<<1];
    node& rs = tree[x<<1|1];
    node& now = tree[x];
    now.sum = ls.sum + rs.sum;
    now.mxl = max(ls.mxl, ls.sum + rs.mxl);
    now.mxr = max(rs.mxr, rs.sum + ls.mxr);
    ll a = ls.mxa, b = rs.mxa, c = ls.mxr + rs.mxl;
    now.mxa = max(a, max(b, c));
}
void build(int x, int l, int r){
    node& now = tree[x];
    now.lf = l, now.rt = r;
    if(l == r){
        now.sum = now.mxa = a[l];
        now.mxl = now.mxr = a[l];
        return ;
    }
    int mid = (l + r)>>1;
    int ls = (x<<1), rs = (x<<1)|1;
    build(ls, l, mid);
    build(rs, mid + 1, r);
    pushUp(x);
}
ll querySum(int x, int l, int r){
     node& now = tree[x];
    if(l == now.lf && r == now.rt) return now.sum;
    int mid = (now.lf + now.rt)>>1;
    int ls = x<<1, rs = (x<<1)|1;
    if(mid >= r) return querySum(ls, l, r);
    if(mid < l) return querySum(rs, l, r);
    ll ret = querySum(ls, l, mid) + querySum(rs, mid + 1, r);
    return ret;
}
ll queryLf(int x, int l, int r){
    node& now = tree[x];
    if(l == now.lf && r == now.rt) return now.mxl;
    int mid = (now.lf + now.rt)>>1;
    int ls = x<<1, rs = (x<<1)|1;
    if(mid >= r) return queryLf(ls, l, r);
    if(mid < l) return queryLf(rs, l, r);
    ll ret = max(queryLf(ls, l, mid), queryLf(rs, mid + 1, r) + querySum(ls, l, mid));
    return ret;
}
ll queryRt(int x, int l, int r){
    node& now = tree[x];
    if(l == now.lf && r == now.rt) return now.mxr;
    int mid = (now.lf + now.rt)>>1;
    int ls = x<<1, rs = (x<<1)|1;
    if(mid >= r) return queryRt(ls, l, r);
    if(mid < l) return queryRt(rs, l, r);
    ll ret = max(queryRt(rs, mid + 1, r), queryRt(ls, l, mid) + querySum(rs, mid + 1, r));
    return ret;
}
ll queryAll(int x, int l, int r){
    node& now = tree[x];
    if(l == now.lf && r == now.rt) return now.mxa;
    int mid = (now.lf + now.rt)>>1;
    int ls = x<<1, rs = (x<<1)|1;
    if(mid >= r) return queryAll(ls, l, r);
    if(mid < l) return queryAll(rs, l, r);
    ll ret = max(queryAll(ls, l, mid), queryAll(rs, mid + 1, r));
    ret = max(ret, queryLf(rs, mid + 1, r) + queryRt(ls, l, mid));
    return ret;
}
int main(){
    int n; scanf("%d", &n);
    for(int i = 1; i <= n; ++i) scanf("%lld", &a[i]);
    build(1, 1, n);
    int q; scanf("%d", &q);
    while(q--){
        int x, y; scanf("%d%d", &x, &y);
        ll ans = queryAll(1, x, y);
        printf("%lld\n", ans);
    }
    return 0;
}
\end{lstlisting}

\subsubsection{权值线段树启发式合并}

\begin{lstlisting}[language=C++]
const int maxn = 1e5 + 5;
struct SegTree {
    int l, r, lc, rc;
    int num;
    SegTree() {
        l = r = 0;
        lc = rc = 0;
        num = 0;
    }
}tree[maxn << 5];  // size = n*log_n
int root[maxn], fa[maxn], tot = 0;
int reuse[maxn << 5], cnt = 0;
void ist(int& x, int lf, int rt, int y) {
    if(x == 0) {
        if(cnt == 0) x = ++tot;
        else x = reuse[--cnt];
    }
    SegTree& st = tree[x];
    st.l = lf, st.r = rt;
    if(lf == rt) {
        st.num = 1;
        return ;
    }
    int mid = (lf + rt) >> 1;
    if(y <= mid) ist(st.lc, lf, mid, y);
    else ist(st.rc, mid + 1, rt, y);
    st.num = tree[st.lc].num + tree[st.rc].num;
}
int kth(int x, int k) {
    SegTree& st = tree[x];
    if(k > st.num) return -1;
    if(st.l == st.r) return st.l;
    int lc = st.lc, rc = st.rc;
    if(tree[lc].num >= k) return kth(lc, k);
    else return kth(rc, k - tree[lc].num);
}
int query(int x, int lf, int rt) {
    if(x == 0) return 0;
    SegTree& st = tree[x];
    if(st.l >= lf && st.r <= rt) return st.num;
    int ret = 0;
    int mid = (st.l + st.r) >> 1;
    if(mid >= lf) ret += query(st.lc, lf, rt);
    if(mid < rt) ret += query(st.rc, lf, rt);
    return ret;
}
void del(int x) {
    if(x == 0) return ;
    tree[x].l = tree[x].r = 0;
    tree[x].lc = tree[x].rc = 0;
    tree[x].num = 0;
    reuse[cnt++] = x;
}
int mrg(int x, int y) {
    if(x == 0) return y;
    if(y == 0) return x;
    if(tree[x].num < tree[y].num) swap(x, y);
    int xlc = tree[x].lc, xrc = tree[x].rc;
    int ylc = tree[y].lc, yrc = tree[y].rc;
    tree[x].lc = mrg(xlc, ylc);
    tree[x].rc = mrg(xrc, yrc);
    tree[x].num += tree[y].num;
    del(y);
    return x;
}
\end{lstlisting}

\subsubsection{矩形周长并}

\begin{lstlisting}[language=C++]
#include <cstdio>
#include <algorithm>
#include <vector>
#include <unordered_map>
using namespace std;
typedef long long ll;
const ll mod = 1e9 + 7;
const int inf = 0x3f3f3f3f;
const ll linf = 0x3f3f3f3f3f3f3f3f;
// (max N) * 4
const int maxn = (5e4 + 5)*4;
struct Rect {
    int lf, rt, up, down;
} r[maxn];
struct Event {
    int lf, rt, tp;
    bool operator <(const Event& rhs) const {
        return tp > rhs.tp;
    }
};
vector<Event> E[maxn];
struct SegTree {
    int lf, rt;
    ll mi, mx, lazy, len;
    void pt() {
        printf("[%d, %d]: mi: %lld mx: %lld lazy: %lld len: %lld\n", lf, rt, mi, mx, lazy, len);
    }
} st[maxn << 2];
vector<int> v;
ll len[maxn];
void Build(int x, int lf, int rt) {
    SegTree& now = st[x];
    now.lf = lf, now.rt = rt;
    now.mi = 0; now.mx = 0;
    now.lazy = 0;
    if(lf == rt) {
        now.len = len[lf];
        return ;
    }
    int mid = (lf + rt) >> 1;
    Build(x << 1, lf, mid);
    Build(x << 1 | 1, mid + 1, rt);
    now.len = st[x << 1].len + st[x << 1 | 1].len;
}
void PushUp(int x) {
    SegTree& now = st[x];
    const SegTree& lc = st[x << 1];
    const SegTree& rc = st[x << 1 | 1];
    now.mi = min(lc.mi, rc.mi);
    now.mx = max(lc.mx, rc.mx);
}
void PushDown(int x) {
    SegTree& now = st[x];
    if(now.lazy == 0) return ;
    SegTree& lc = st[x << 1];
    SegTree& rc = st[x << 1 | 1];
    lc.lazy += now.lazy;
    lc.mi += now.lazy;
    lc.mx += now.lazy;
    rc.lazy += now.lazy;
    rc.mi += now.lazy;
    rc.mx += now.lazy;
    now.lazy = 0;
}
ll add(int x, int lf, int rt) {
    if(lf > rt) return 0LL;
    SegTree& now = st[x];
    if(lf <= now.lf && rt >= now.rt) {
        if(now.mx <= 0) {
            ++now.mi; ++now.mx;
            ++now.lazy;
            return now.len;
        }
        if(now.mi > 0) {
            ++now.mi; ++now.mx;
            ++now.lazy;
            return 0;
        }
    }
    PushDown(x);
    ll ret = 0;
    int mid = (now.lf + now.rt) >> 1;
    if(mid >= lf) ret += add(x << 1, lf, rt);
    if(mid < rt) ret += add(x << 1 | 1, lf, rt);
    PushUp(x);
    return ret;
}
ll del(int x, int lf, int rt) {
    if(lf > rt) return 0LL;
    SegTree& now = st[x];
    if(lf <= now.lf && rt >= now.rt) {
        if(now.mx <= 1) {
            --now.mi; --now.mx;
            --now.lazy;
            return now.len;
        }
        if(now.mi > 1) {
            --now.mi; --now.mx;
            --now.lazy;
            return 0;
        }
    }
    PushDown(x);
    ll ret = 0;
    int mid = (now.lf + now.rt) >> 1;
    if(mid >= lf) ret += del(x << 1, lf, rt);
    if(mid < rt) ret += del(x << 1 | 1, lf, rt);
    PushUp(x);
    return ret;
}
int n, sz;
ll solve() {
    ll ret = 0;
    Build(1, 1, sz);
    for(int i = 1; i <= sz; ++i) {
        sort(E[i].begin(), E[i].end());
        int sz = E[i].size();
        for(int j = 0; j < sz; ++j) {
            Event& now = E[i][j];
            if(now.tp > 0) ret += add(1, now.lf, now.rt);
            else ret += del(1, now.lf, now.rt);
        }
        E[i].clear();
    }
    return ret;
}
unordered_map<int, int> id;
void init() {
    sort(v.begin(), v.end());
    v.erase(unique(v.begin(), v.end()), v.end());
    sz = v.size();
    for(int i = 0; i < sz; ++i) {
        id[v[i]] = i + 1;
        if(i > 0) len[i] = v[i] - v[i - 1];
    }
}
int main(){
    scanf("%d", &n);
    for(int i = 0; i < n; ++i) {
        scanf("%d%d%d%d", &r[i].lf, &r[i].down, &r[i].rt, &r[i].up);
        v.push_back(r[i].lf); v.push_back(r[i].rt);
        v.push_back(r[i].down); v.push_back(r[i].up);
    }
    init();
    for(int i = 0; i < n; ++i) {
        r[i].lf = id[r[i].lf];
        r[i].rt = id[r[i].rt];
        r[i].down = id[r[i].down];
        r[i].up = id[r[i].up];
    }
    for(int i = 0; i < n; ++i) {
        E[r[i].down].push_back({r[i].lf, r[i].rt - 1, 1});
        E[r[i].up].push_back({r[i].lf, r[i].rt - 1, -1});
    }
    ll ans = solve();
    for(int i = 0; i < n; ++i) {
        E[r[i].lf].push_back({r[i].down, r[i].up - 1, 1});
        E[r[i].rt].push_back({r[i].down, r[i].up - 1, -1});
    }
    ans += solve();
    printf("%lld\n", ans);
    return 0;
}
\end{lstlisting}

\subsubsection{主席树}

区间第 k 大

\begin{lstlisting}[language=C++]
const int maxn = 1e5 + 5;
struct node{
    int l, r;
    int sum;
} T[maxn << 5];
int a[maxn], cnt, root[maxn];
void update(int l, int r, int& x, int y, int pos){
    T[++cnt] = T[y]; ++T[cnt].sum; x = cnt;
    if(l == r) return ;
    int mid = (l + r) >> 1;
    if(mid >= pos) update(l, mid, T[x].l, T[y].l, pos);
    else update(mid + 1, r, T[x].r, T[y].r, pos);
}
int query(int l, int r, int x, int y, int k){
    if(l == r) return l;
    int mid = (l + r) >> 1;
    int sum = T[T[y].l].sum - T[T[x].l].sum;
    if(sum >= k) return query(l, mid, T[x].l, T[y].l, k);
    return query(mid + 1, r, T[x].r, T[y].r, k - sum);
}
\end{lstlisting}

\subsection{Trie 树 (字典树)}

\begin{lstlisting}[language=C++]
int trie[400001][26];
int tot, sum[400001];
char s[11];
void ist(){
    int ls = strlen(s);
    int root = 0;
    for(int i = 0; i < ls; ++i){
        int id = s[i] - 'a';
        if(!trie[root][id]){
            ++tot;
            trie[root][id] = tot;
        }
        ++sum[trie[root][id]];
        root = trie[root][id];
    }
    return ;
}

int src(){
    int root = 0;
    int ls = strlen(s);
    for(int i = 0; i < ls; ++i){
        int id = s[i] - 'a';
        if(!trie[root][id])
            return 0;
        root = trie[root][id];
    }
    return sum[root];
}
\end{lstlisting}

\subsection{RMQ 问题}

\begin{lstlisting}[language=C++]
int n, logn, bs2[maxn], pos[maxn][20];
ll a[maxn], mx[maxn][20];
void RMQ_Init() {
    logn = int(ceil(log2(n)) + eps);
    int bs = 1, num = 1;
    for(int i = 1; i <= n; ++i) {
        mx[i][0] = a[i];
        pos[i][0] = i;
        if((num << 1) <= i) {
            ++bs;
            num <<= 1;
        }
        bs2[i] = bs - 1;
    }
    for(int i = 1; i <= logn; ++i) {
        for(int j = 1; j + (1 << i) - 1 <= n; ++j) {
            if(mx[j][i - 1] > mx[j + (1 << (i - 1))][i - 1]) {
                mx[j][i] = mx[j][i - 1];
                pos[j][i] = pos[j][i - 1];
            } else {
                mx[j][i] = mx[j + (1 << (i - 1))][i - 1];
                pos[j][i] = pos[j + (1 << (i - 1))][i - 1];
            }
        }
    }
}
ll Qmx; int Qpos;
void Query(int lf, int rt) {
    int bs = bs2[rt - lf + 1];
    if(mx[lf][bs] > mx[rt - (1 << bs) + 1][bs]) {
        Qmx = mx[lf][bs];
        Qpos = pos[lf][bs];
    } else {
        Qmx = mx[rt - (1 << bs) + 1][bs];
        Qpos = pos[rt - (1 << bs) + 1][bs];
    }
}
\end{lstlisting}

\subsection{树链剖分(轻重链剖分) + 线段树维护链上信息}

Heavy-light Decomposition :HLD

边剖分可以将权赋给一条边上 dep 更大的那个节点,转化成点权问题。

最后 x 和 y 在同一个重链的时候,直接 $update(pos[x] + 1 ,  pos[y], 1,  maxn, 1)$

注意是 $pos[x] + 1$.

\subsubsection{模板}

\begin{lstlisting}[language=C++]
struct Edge {
    int to, nxt;
    int v;
    Edge(int to = 0, int nxt = 0, int v = 0): to(to), nxt(nxt), v(v){}
} e[maxn << 1];
int head[maxn], tot;
void init(){
    memset(head, -1, sizeof(head));
    tot = 0;
}
void addEdge(int x, int y, int z){
    e[tot] = Edge(y, head[x], z);
    head[x] = tot++;
}

int rk[maxn]; // dfs order to original order

struct TreeChain {
    int fa, d, sz;
    int hs; // heavy son
    int tp; // top of heavy link
    int id; // original order to dfs order
    TreeChain(){
        hs = 0;
    }
} tc[maxn];
int tcc; // Tree Chain dfs order
void dfs1(int x, int y, int d){
    TreeChain& now = tc[x];
    now.sz = 1; now.fa = y;
    now.d = d;
    for(int i = head[x]; i != -1; i = e[i].nxt){
        int to = e[i].to;
        if(to == y) continue;
        dfs1(to, x, d + 1);
        now.sz += tc[to].sz;
        int hs = now.hs;
        if(hs == 0 || tc[hs].sz < tc[to].sz)
            now.hs = to;
    }
}
void dfs2(int x, int y, int z){
    TreeChain& now = tc[x];
    now.tp = z; now.id = tcc;
    rk[tcc++] = x;
    int hs = now.hs;
    if(hs == 0) return ;
    dfs2(hs, x, z);
    for(int i = head[x]; i != -1; i = e[i].nxt){
        int to = e[i].to;
        if(to == y || to == hs) continue;
        dfs2(to, x, to);
    }
}
void TreeChainPartition(int root){
    tcc = 1;
    memset(tc, 0, sizeof(tc));
    dfs1(root, -1, 0);
    dfs2(root, -1, root);
}
\end{lstlisting}

\subsubsection{树链剖分维护软件依赖关系}

给你一棵树,有两种操作:

把一个节点变为 1,把一个节点变为 0

如果一个节点变为 1,那么他的所有祖先节点都变为 1.

如果一个节点变为 0,那么他的所有后代节点都变为 0.

对于每一个操作,输出发生改变的节点数量。


\begin{lstlisting}[language=C++]
const int maxn = 3e5 + 3;
struct Edge{
    int to, nxt;
    Edge(int to = 0, int nxt = 0): to(to), nxt(nxt){}
} e[maxn << 1];
int head[maxn], tot;
void init(){
    memset(head, -1, sizeof(head));
    tot = 0;
}
void addEdge(int x, int y){
    e[tot] = Edge(y, head[x]);
    head[x] = tot++;
}

int rk[maxn]; // dfs order to original order

struct TreeChain{
    int fa, d, sz;
    int hs; // heavy son
    int tp; // top of heavy link
    int id; // original order to dfs order
    TreeChain(){
        hs = -1;
    }
} tc[maxn];
int tcc; // Tree Chain dfs order
void dfs1(int x, int y, int d){
    TreeChain& now = tc[x];
    now.sz = 1; now.fa = y;
    now.d = d;
    for(int i = head[x]; i != -1; i = e[i].nxt){
        int to = e[i].to;
        if(to == y) continue;
        dfs1(to, x, d + 1);
        now.sz += tc[to].sz;
        int hs = now.hs;
        if(hs == -1 || tc[hs].sz < tc[to].sz)
            now.hs = to;
    }
}
void dfs2(int x, int y, int z){
    TreeChain& now = tc[x];
    now.tp = z; now.id = tcc;
    rk[tcc++] = x;
    int hs = now.hs;
    if(hs == -1) return ;
    dfs2(hs, x, z);
    for(int i = head[x]; i != -1; i = e[i].nxt){
        int to = e[i].to;
        if(to == y || to == hs) continue;
        dfs2(to, x, to);
    }
}
void TreeChainPartition(int root){
    tcc = 1;
    dfs1(root, -1, 0);
    dfs2(root, -1, root);
}

struct SegTree{
    int l, r;
    int ls, rs;
    int sum, lazy;
}st[maxn<<2];
int stc = 0; // Segment Tree counting number
int root; // root of Segment Tree
void pushUp(int x){
    SegTree& now = st[x];
    SegTree& ls = st[now.ls];
    SegTree& rs = st[now.rs];
    now.sum = ls.sum + rs.sum;
    now.l = ls.l;
    now.r = rs.r;
}
void build(int lf, int rt, int x){
    SegTree& now = st[x];
    if(lf == rt){
        now.ls = now.rs = now.lazy = -1;
        now.l = now.r = lf;
        return ;
    }
    int mid = (lf + rt)>>1;
    now.ls = stc++; now.rs = stc++;
    build(lf, mid, now.ls);
    build(mid + 1, rt, now.rs);
    pushUp(x);
}
void pushDown(int x){
    SegTree& now = st[x];
    SegTree& ls = st[now.ls];
    SegTree& rs = st[now.rs];
    ls.sum = now.lazy*(ls.r - ls.l + 1);
    rs.sum = now.lazy*(rs.r - rs.l + 1);
    ls.lazy = now.lazy;
    rs.lazy = now.lazy;
    now.lazy = -1;
}
void update(int lf, int rt, int c, int x){
    SegTree& now = st[x];
    if(lf <= now.l && now.r <= rt){
        now.sum = c*(now.r - now.l + 1);
        now.lazy = c;
        return ;
    }
    if(now.lazy != -1) pushDown(x);
    int mid = (now.l + now.r)>>1;
    if(lf <= mid) update(lf, rt, c, now.ls);
    if(rt > mid) update(lf, rt, c, now.rs);
    pushUp(x);
}
int query(int lf, int rt, int x){
    SegTree& now = st[x];
    if(lf <= now.l && now.r <= rt) return now.sum;
    if(now.lazy != -1) pushDown(x);
    int ret = 0, mid = (now.l + now.r)>>1;
    if(lf <= mid) ret += query(lf, rt, now.ls);
    if(rt > mid) ret += query(lf, rt, now.rs);
    return ret;
}

int install(int x){
    int ret = 0, y = tc[x].tp;
    while(y != 0){
        int lf = tc[y].id, rt = tc[x].id;
        ret += rt - lf + 1 - query(lf, rt, root);
        update(lf, rt, 1, root);
        x = tc[y].fa;
        y = tc[x].tp;
    }
    int lf = tc[y].id, rt = tc[x].id;
    ret += rt - lf + 1 - query(lf, rt, root);
    update(lf, rt, 1, root);
    return ret;
}

int uninstall(int x){
    int lf = tc[x].id, rt = tc[x].id + tc[x].sz - 1;
    int ret = query(lf, rt, root);
    update(lf, rt, 0, root);
    return ret;
}
char s[15];
int main(){
    init();
    int n; scanf("%d", &n);
    for(int i = 1; i < n; ++i){
        int x; scanf("%d", &x);
        addEdge(x, i);
        addEdge(i, x);
    }
    TreeChainPartition(0);
    root = stc++;
    build(1, n, root);
    int q; scanf("%d", &q);
    while(q--){
        int x; scanf("%s %d", s, &x);
        if(s[0] == 'i') printf("%d\n", install(x));
        else printf("%d\n", uninstall(x));
    }
    return 0;
}
\end{lstlisting}

\subsubsection{边权问题}

给出一棵树,每条边编号 $[1, n - 1]$ 有权值,可修改第i条边的权值,支持询问从节点 x 到节点 y 的路径上的最大边权。

\begin{lstlisting}[language=C++]
const int maxn = 1e5 + 5;
int ea[maxn], eb[maxn], ec[maxn];
struct Edge{
    int to, nxt;
    int v;
    Edge(int to = 0, int nxt = 0, int v = 0): to(to), nxt(nxt), v(v){}
}e[maxn<<1];
int head[maxn], tot;
void init(){
    memset(head, -1, sizeof(head));
    tot = 0;
}
void addEdge(int x, int y, int z){
    e[tot] = Edge(y, head[x], z);
    head[x] = tot++;
}

int rk[maxn]; // dfs order to original order

struct TreeChain{
    int fa, d, sz;
    int hs; // heavy son
    int tp; // top of heavy link
    int id; // original order to dfs order
    TreeChain(){
        hs = 0;
    }
}tc[maxn];
int tcc; // Tree Chain dfs order
void dfs1(int x, int y, int d){
    TreeChain& now = tc[x];
    now.sz = 1; now.fa = y;
    now.d = d;
    for(int i = head[x]; i != -1; i = e[i].nxt){
        int to = e[i].to;
        if(to == y) continue;
        dfs1(to, x, d + 1);
        now.sz += tc[to].sz;
        int hs = now.hs;
        if(hs == 0 || tc[hs].sz < tc[to].sz)
            now.hs = to;
    }
}
void dfs2(int x, int y, int z){
    TreeChain& now = tc[x];
    now.tp = z; now.id = tcc;
    rk[tcc++] = x;
    int hs = now.hs;
    if(hs == 0) return ;
    dfs2(hs, x, z);
    for(int i = head[x]; i != -1; i = e[i].nxt){
        int to = e[i].to;
        if(to == y || to == hs) continue;
        dfs2(to, x, to);
    }
}
void TreeChainPartition(int root){
    tcc = 1;
    memset(tc, 0, sizeof(tc));
    dfs1(root, -1, 0);
    dfs2(root, -1, root);
}

struct SegTree{
    int l, r;
    int ls, rs;
    int mx;
}st[maxn<<2];
int stc = 0; // Segment Tree counting number
int root; // root of Segment Tree
void pushUp(int x){
    SegTree& now = st[x];
    SegTree& ls = st[now.ls];
    SegTree& rs = st[now.rs];
    now.mx = max(ls.mx, rs.mx);
    now.l = ls.l;
    now.r = rs.r;
}
void build(int lf, int rt, int x){
    SegTree& now = st[x];
    if(lf == rt){
        now.ls = now.rs = -1;
        now.l = now.r = lf;
        now.mx = -0x3f3f3f3f;
        return ;
    }
    int mid = (lf + rt)>>1;
    now.ls = stc++; now.rs = stc++;
    build(lf, mid, now.ls);
    build(mid + 1, rt, now.rs);
    pushUp(x);
}
void update(int pos, int c, int x){
    SegTree& now = st[x];
    if(now.l == now.r){
        now.mx = c;
        return ;
    }
    int mid = (now.l + now.r)>>1;
    if(pos <= mid) update(pos, c, now.ls);
    if(pos > mid) update(pos, c, now.rs);
    pushUp(x);
}
int query(int lf, int rt, int x){
    SegTree& now = st[x];
    if(lf <= now.l && now.r <= rt) return now.mx;
    int ret = -0x3f3f3f3f, mid = (now.l + now.r)>>1;
    if(lf <= mid) ret = max(ret, query(lf, rt, now.ls));
    if(rt > mid) ret = max(ret, query(lf, rt, now.rs));
    return ret;
}

int QUERY(int x, int y){
    int ret = -0x3f3f3f3f;
    int tx = tc[x].tp, ty = tc[y].tp;
    while(tx != ty){
        if(tc[tx].d < tc[ty].d){
            swap(tx, ty);
            swap(x, y);
        }
        ret = max(ret, query(tc[tx].id, tc[x].id, root));
        x = tc[tx].fa; tx = tc[x].tp;
    }
    if(x == y) return ret;
    if(tc[x].d > tc[y].d) swap(x, y);
    ret = max(ret, query(tc[tc[x].hs].id, tc[y].id, root));
    return ret;
}
void CHANGE(int x, int y){
    update(tc[eb[x]].id, y, root);
}
char s[15];

int main(){
    int t; scanf("%d", &t);
    while(t--){
        init();
        int n; scanf("%d", &n);
        for(int i = 1; i < n; ++i){
            int a, b, c; scanf("%d%d%d", &a, &b, &c);
            addEdge(a, b, c);
            addEdge(b, a, c);
            ea[i] = a, eb[i] = b, ec[i] = c;
        }
        TreeChainPartition(1);
        stc = 0;
        root = stc++;
        build(1, n, root);
        for(int i = 1; i < n; ++i){
            if(tc[ea[i]].d > tc[eb[i]].d) swap(ea[i], eb[i]);
            update(tc[eb[i]].id, ec[i], root);
        }
        while(scanf("%s", s) != EOF){
            if(s[0] == 'D') break;
            int x, y; scanf("%d %d", &x, &y);
            if(s[0] == 'Q') printf("%d\n", QUERY(x, y));
            else CHANGE(x, y);
        }
        if(t) puts("");
    }
    return 0;
}
\end{lstlisting}

\subsubsection{点权问题}

给出一棵树,每个点编号 $[1, n]$ 有权值,可修改第 i 个点的权值,支持询问从节点 x 到节点 y 的路径上的最大点权。

\begin{lstlisting}[language=C++]
const int maxn = 3e4 + 5;
const ll inf = 0x3f3f3f3f3f3f3f3f;

struct Edge{
    int to, nxt;
    int v;
    Edge(int to = 0, int nxt = 0, int v = 0):to(to), nxt(nxt), v(v){}
}e[maxn<<1];
int head[maxn], tot;
void init(){
    memset(head, -1, sizeof(head));
    tot = 0;
}
void addEdge(int x, int y, int z){
    e[tot] = Edge(y, head[x], z);
    head[x] = tot++;
}

int rk[maxn];

struct TreeChain{
    int fa, d, sz;
    int hs, tp, id;
    TreeChain(){
        hs = 0;
    }
}tc[maxn];
int tcc;
void dfs1(int x, int y, int d){
    TreeChain& now = tc[x];
    now.sz = 1, now.fa = y;
    now.d = d;
    for(int i = head[x]; i != -1; i = e[i].nxt){
        int to = e[i].to;
        if(to == y) continue;
        dfs1(to, x, d + 1);
        now.sz += tc[to].sz;
        int hs = now.hs;
        if(hs == 0 || tc[hs].sz < tc[to].sz) now.hs = to;
    }
}

void dfs2(int x, int y, int z){
    TreeChain& now = tc[x];
    now.tp = z; now.id = tcc;
    rk[tcc++] = x;
    int hs = now.hs;
    if(hs == 0) return ;
    dfs2(hs, x, z);
    for(int i = head[x]; i != -1; i = e[i].nxt){
        int to = e[i].to;
        if(to == y || to == hs) continue;
        dfs2(to, x, to);
    }
}

void TreeChainPartition(int root){
    tcc = 1;
    memset(tc, 0, sizeof(tc));
    dfs1(root, -1, 0);
    dfs2(root, -1, root);
}

struct SegTree{
    int l, r;
    int ls, rs;
    ll mx, sum;
}st[maxn<<2];

int stc = 0;
int root;

ll a[maxn];

void pushUp(int x){
    SegTree& now = st[x];
    SegTree& ls = st[now.ls];
    SegTree& rs = st[now.rs];
    now.mx = max(ls.mx, rs.mx);
    now.sum = ls.sum + rs.sum;
}

void build(int x, int lf, int rt){
    SegTree& now = st[x];
    now.l = lf, now.r = rt;
    if(lf == rt){
        now.ls = now.rs = -1;
        now.mx = -inf;
        now.sum = 0;
        return ;
    }
    int mid = (lf + rt)>>1;
    now.ls = stc++; now.rs = stc++;
    build(now.ls, lf, mid);
    build(now.rs, mid + 1, rt);
    pushUp(x);
}

void update(int x, int pos, ll val){
    SegTree& now = st[x];
    if(now.l == now.r){
        now.mx = val;
        now.sum = val;
        return ;
    }
    int mid = (now.l + now.r)>>1;
    if(pos <= mid) update(now.ls, pos, val);
    else update(now.rs, pos, val);
    pushUp(x);
}

ll queryMax(int x, int lf, int rt){
    SegTree& now = st[x];
    if(lf <= now.l && now.r <= rt) return now.mx;
    ll ret = -inf; int mid = (now.l + now.r)>>1;
    if(lf <= mid) ret = max(ret, queryMax(now.ls, lf, rt));
    if(rt > mid) ret = max(ret, queryMax(now.rs, lf, rt));
    return ret;
}

ll querySum(int x, int lf, int rt){
    SegTree& now = st[x];
    if(lf <= now.l && now.r <= rt) return now.sum;
    ll ret = 0; int mid = (now.l + now.r)>>1;
    if(lf <= mid) ret += querySum(now.ls, lf, rt);
    if(rt > mid) ret += querySum(now.rs, lf, rt);
    return ret;
}
char s[15];

ll QMAX(int x, int y){
    ll ret = -inf;
    int tx = tc[x].tp, ty = tc[y].tp;
    while(tx != ty){
        if(tc[tx].d < tc[ty].d){
            swap(tx, ty);
            swap(x, y);
        }
        ret = max(ret, queryMax(root, tc[tx].id, tc[x].id));
        x = tc[tx].fa; tx = tc[x].tp;
    }
    if(tc[x].d > tc[y].d) swap(x, y);
    ret = max(ret, queryMax(root, tc[x].id, tc[y].id));
    return ret;
}

ll QSUM(int x, int y){
    ll ret = 0;
    int tx = tc[x].tp, ty = tc[y].tp;
    while(tx != ty){
        if(tc[tx].d < tc[ty].d){
            swap(tx, ty);
            swap(x, y);
        }
        ret += querySum(root, tc[tx].id, tc[x].id);
        x = tc[tx].fa; tx = tc[x].tp;
    }
    if(tc[x].d > tc[y].d) swap(x, y);
    ret += querySum(root, tc[x].id, tc[y].id);
    return ret;
}

void CHANGE(int x, ll y){
    update(root, tc[x].id, y);
}

int main(){
    init();
    int n; scanf("%d", &n);
    for(int i = 1; i < n; ++i){
        int u, v;
        scanf("%d%d", &u, &v);
        addEdge(u, v, 0);
        addEdge(v, u, 0);
    }
    TreeChainPartition(1);
    stc = 0;
    root = stc++;
    build(root, 1, n);
    for(int i = 1; i <= n; ++i){
        int x; scanf("%d", &x);
        update(root, tc[i].id, x);
    }
    int q; scanf("%d", &q);
    while(q--){
        scanf("%s", s);
        int x, y; scanf("%d%d", &x, &y);
        if(s[1] == 'M'){
            ll ans = QMAX(x, y);
            printf("%lld\n", ans);
        }else if(s[1] == 'S'){
            ll ans = QSUM(x, y);
            printf("%lld\n", ans);
        }else CHANGE(x, (ll)y);
    }
    return 0;
}
\end{lstlisting}

\subsection{带权并查集}

\begin{lstlisting}[language=C++]
int fa[maxn], r[maxn];
int findfa(int x){
    if(x != fa[x]){
        int fx = fa[x];
        fa[x] = findfa(fx);
        r[x] = (r[fx] + r[x])%3;
    }
    return fa[x];
}
void mrg(int x, int y, int d){
    int fx = findfa(x), fy = findfa(y);
    if(fx == fy) return ;
    fa[fx] = fy;
    r[fx] = (d + r[y] - r[x] + 3)%3;
}

int fa[maxn], w[maxn];
int findfa(int x) {
    if(fa[x] != x) {
        int fx = fa[x];
        fa[x] = findfa(fx);
        w[x] += w[fx];
    }
    return fa[x];
}
bool mrg(int x, int y, int z) {
    int fx = findfa(x);
    int fy = findfa(y);
    if(fx != fy) {
        fa[fy] = fx;
        w[fy] = z + w[x] - w[y];
    } else {
        if(w[x] + z != w[y]) return false;
    }
    return true;
}
\end{lstlisting}

\subsection{二维树状数组}

\subsubsection{单点修改 + 区间查询}

\begin{lstlisting}[language=C++]
const int maxn = 1e3 + 5;
#define lowbit(x) (x&(-x))
int tree[maxn][maxn];
void update(int x, int y, int z){
    for(int i = x; i < maxn; i += lowbit(i))
        for(int j = y; j < maxn; j += lowbit(j))
            tree[i][j] += z;
}
int query(int x, int y){
    int ret = 0;
    for(int i = x; i > 0; i -= lowbit(i))
        for(int j = y; j > 0; j -= lowbit(j))
            ret += tree[i][j];
    return ret;
}
\end{lstlisting}

\subsubsection{区间修改 + 单点查询}

二维前缀和:

sum[i][j] = sum[i − 1][j] + sum[i][j − 1] − sum[i − 1][j − 1] + a[i][j]

那么我们可以令差分数组d[i][j] 表示 a[i][j] 与 a[i − 1][j] + a[i][j − 1] − a[i − 1][j − 1] 的差。

我们在差分数组d[][]上建树状数组,便可以维护区间修改的值,并支持单点查询

\begin{lstlisting}[language=C++]
const int maxn = 1e3 + 5;
#define lowbit(x) (x&(-x))
int tree[maxn][maxn];
void update(int x, int y, int z){
    for(int i = x; i < maxn; i += lowbit(i))
        for(int j = y; j < maxn; j += lowbit(j))
            tree[i][j] += z;
}
void rangeUpdate(int xa, int ya, int xb, int yb, int z){
    update(xa, ya, z);
    update(xa, yb + 1, -z);
    update(xb + 1, ya, -z);
    update(xb + 1, yb + 1, z);
}
int query(int x, int y){
    int ret = 0;
    for(int i = x; i > 0; i -= lowbit(i))
        for(int j = y; j > 0; j -= lowbit(j))
            ret += tree[i][j];
    return ret;
}
\end{lstlisting}

\subsubsection{区间修改 + 区间查询(待补)}

\subsection{离线分治算法}

\subsubsection{CDQ分治 三维偏序(陌上花开)}

有n朵花,每朵花有三个属性:花形(s)、颜色(c)、气味(m),由三个整数表示。

现要对每朵花评级,一朵花的级别是它拥有的美丽能超过的花的数量。定义一朵花 A 比另一朵花 B 要美丽,当且仅当Sa >= Sb, Ca >= Cb, Ma >= Mb。显然,两朵花可能有同样的属性。

需要统计出评出每个等级的花的数量。

\begin{lstlisting}[language=C++]
const int maxn = 4e5 + 5;
struct node{
    int s, c, m;
    int id;
    int num, ans;
    bool operator < (const node& nd) const{
        if(s == nd.s){
            if(c == nd.c) return m < nd.m;
            return c < nd.c;
        }
        return s < nd.s;
    }
    bool operator == (const node& nd) const{
        if(s != nd.s) return false;
        if(c != nd.c) return false;
        if(m != nd.m) return false;
        return true;
    }
}a[maxn], b[maxn], c[maxn];
int cnt[maxn];
#define lowbit(x) (x&(-x))
int tree[maxn];
void update(int x, int val){
    while(x < maxn){
        tree[x] += val;
        x += lowbit(x);
    }
}
int query(int x){
    int ret = 0;
    while(x){
        ret += tree[x];
        x -= lowbit(x);
    }
    return ret;
}
void cdq(int lf, int rt){
    if(lf == rt){
        b[lf].ans = b[lf].num - 1;
        return ;
    }
    int mid = (lf + rt)>>1;
    cdq(lf, mid); cdq(mid + 1, rt);
    int p1 = lf, p2 = mid + 1;
    int p3 = lf;
    while(p1 <= mid && p2 <= rt){
        if(b[p1].c <= b[p2].c){
            update(b[p1].m, b[p1].num);
            c[p3++] = b[p1++];
        }else{
            b[p2].ans += query(b[p2].m);
            c[p3++] = b[p2++];
        }
    }
    while(p1 <= mid){
        update(b[p1].m, b[p1].num);
        c[p3++] = b[p1++];
    }
    while(p2 <= rt){
        b[p2].ans += query(b[p2].m);
        c[p3++] = b[p2++];
    }
    for(int i = lf; i <= mid; ++i) update(b[i].m, -b[i].num);
    for(int i = lf; i <= rt; ++i) b[i] = c[i];
}
int main(){
    int n, k; scanf("%d%d", &n, &k);
    for(int i = 1; i <= n; ++i){
        scanf("%d%d%d", &a[i].s, &a[i].c, &a[i].m);
        a[i].id = i;
    }
    sort(a + 1, a + n + 1);
    int tot = 0;
    b[++tot] = a[1];
    b[tot].ans = 0; b[tot].num = 1;
    for(int i = 2; i <= n; ++i){
        if(a[i] == a[i - 1]) ++b[tot].num;
        else{
            b[++tot] = a[i];
            b[tot].ans = 0;
            b[tot].num = 1;
        }
    }
    cdq(1, tot);
    for(int i = 1; i <= tot; ++i) cnt[b[i].ans] += b[i].num;
    for(int i = 0; i < n; ++i) printf("%d\n", cnt[i]);
    return 0;
}
\end{lstlisting}

\subsubsection{基于值域的整体分治算法}

求解区间第 k 小

\begin{lstlisting}[language=C++]
const int maxn = 2e5 + 5;
struct node {
    int op, x, y, z;
} a[maxn], b[maxn], c[maxn];
int tree[maxn], ans[maxn], n, m;
#define lowbit(x) (x&(-x))
void add(int x, int y) {
    while(x <= n) {
        tree[x] += y;
        x += lowbit(x);
    }
}
int query(int x) {
    int ret = 0;
    while(x) {
        ret += tree[x];
        x -= lowbit(x);
    }
    return ret;
}
void solve(int mi, int mx, int lf, int rt) {
    if(lf > rt) return ;
    if(mi == mx) {
        for(int i = lf; i <= rt; ++i) ans[a[i].op] = mi;
        return ;
    }
    int mid = (mi + mx) >> 1;
    int p1 = 0, p2 = 0;
    for(int i = lf; i <= rt; ++i) {
        if(a[i].op == 0) {
            if(a[i].y <= mid) {
                add(a[i].x, 1);
                b[++p1] = a[i];
            } else {
                c[++p2] = a[i];
            }
        } else {
            int num = query(a[i].y) - query(a[i].x - 1);
            if(num >= a[i].z) {
                b[++p1] = a[i];
            } else {
                a[i].z -= num;
                c[++p2] = a[i];
            }
        }
    }
    for(int i = lf; i <= rt; ++i) {
        if(a[i].op == 0 && a[i].y <= mid) add(a[i].x, -1);
    }
    for(int i = 1; i <= p1; ++i) a[lf + i - 1] = b[i];
    for(int i = 1; i <= p2; ++i) a[lf + p1 + i - 1] = c[i];
    solve(mi, mid, lf, lf + p1 - 1);
    solve(mid + 1, mx, lf + p1, rt);
}

int main(){
    scanf("%d%d", &n, &m);
    int tot = 0;
    for(int i = 1; i <= n; ++i) {
        a[++tot].op = 0;
        a[tot].x = i;
        scanf("%d", &a[tot].y);
    }
    for(int i = 1; i <= m; ++i) {
        a[++tot].op = i;
        scanf("%d%d%d", &a[tot].x, &a[tot].y, &a[tot].z);
    }
    solve(-inf, inf, 1, tot);
    for(int i = 1; i <= m; ++i) printf("%d\n", ans[i]);
    return 0;
}
\end{lstlisting}

\subsection{Splay}

\subsubsection{基础用法}

第 1 行:1 个正整数 n,表示操作数量, $100 \le n \le 200,000$

第 2..n + 1 行:可能包含下面 3 种规则:

1 个字母 'I',紧接着1个数字 k,表示插入一个数字 k 到树中,$1 \le k \le 1,000,000,000$ ,保证每个 k 都不相同

1 个字母 'Q',紧接着1个数字 k。表示询问树中不超过 k 的最大数字

1 个字母 'D',紧接着2个数字 a, b ,表示删除树中在区间 $[a, b]$ 的数。

\begin{lstlisting}[language=C++]
const int maxn = 2e5 + 5;
// 注意inf是否已经足够大
const int inf = 0x3f3f3f3f;
struct SplayTree {
    int son[2];
    int fa, key, sz, cnt;
} S[maxn];
int root, tot;
void init() {
    root = tot = 0;
    S[0].son[0] = S[0].son[1] = 0;
    S[0].fa = S[0].key = 0;
    S[0].key = S[0].cnt = 0;
}
void PushUp(int x) {
    if(x == 0) return ;
    SplayTree& now = S[x];
    now.sz = now.cnt;
    if(now.son[0]) {
        SplayTree& ls = S[now.son[0]];
        now.sz += ls.sz;
    }
    if(now.son[1]) {
        SplayTree& rs = S[now.son[1]];
        now.sz += rs.sz;
    }
}
int GetId(int x) {
    if(S[S[x].fa].son[1] == x) return 1;
    return 0;
}
void SetSon(int x, int fa, int y) {
    if(x) S[x].fa = fa;
    if(fa == 0) {
        root = x;
        return ;
    }
    S[fa].son[y] = x;
    PushUp(fa);
}
void Rotate(int x) {
    int f = S[x].fa, ff = S[f].fa;
    int sx = GetId(x), sf = GetId(f);
    int p = S[x].son[sx ^ 1];
    SetSon(p, f, sx);
    SetSon(x, ff, sf);
    SetSon(f, x, sx ^ 1);
    PushUp(f); PushUp(x);
}
void Splay(int x, int y) {
    while(S[x].fa != y) {
        int f = S[x].fa, ff = S[f].fa;
        if(ff != y) {
            if(GetId(x) == GetId(f)) Rotate(f);
            else Rotate(x);
        }
        Rotate(x);
    }
    if(y == 0) root = x;
}
int Find(int x, int val) {
    if(x == 0) return 0;
    SplayTree& now = S[x];
    if(val < now.key) return Find(now.son[0], val);
    if(val > now.key) return Find(now.son[1], val);
    Splay(x, 0);
    return x;
}
void Ist(int& x, int& hot, int val, int fa) {
    if(x == 0) {
        x = ++tot; hot = x;
        SplayTree& now = S[x];
        now.key = val, now.fa = fa;
        now.son[0] = now.son[1] = 0;
        now.sz = now.cnt = 1;
        return ;
    }
    SplayTree& now = S[x];
    if(val == now.key) {
        ++now.cnt;
        hot = x;
    } else {
        if(val < now.key) Ist(now.son[0], hot, val, x);
        else Ist(now.son[1], hot, val, x);
    }
    PushUp(x);
}
void Insert(int val ) {
    int hot;
    Ist(root, hot, val, 0);
    Splay(hot, 0);
}
void DelRoot() {
    SplayTree& now = S[root];
    if(now.cnt > 1) {
        --now.cnt;
        --now.sz;
        return ;
    }
    if(now.son[0] == 0) {
        S[now.son[1]].fa = 0;
        root = now.son[1];
        return ;
    }
    int cur = S[root].son[0];
    while(S[cur].son[1]) cur = S[cur].son[1];
    Splay(cur, root);
    S[cur].son[1] = S[root].son[1];
    root = cur; S[cur].fa = 0;
    if(S[root].son[1]) S[S[root].son[1]].fa = root;
    PushUp(root);
}
void Del(int val) {
    int k = Find(root, val);
    if(k) DelRoot();
}
int Kth(int x, int k) {
    if(x == 0) return 0;
    SplayTree& now = S[x];
    int lt = S[now.son[0]].sz;
    int le = lt + now.cnt;
    if(k <= lt) return Kth(now.son[0], k);
    if(k > le) return Kth(now.son[1], k - le);
    Splay(x, 0);
    return x;
}
int Rank(int x, int val) {
    if(x == 0) return 0;
    SplayTree& now = S[x];
    int lt = S[now.son[0]].sz;
    int le = lt + now.cnt;
    if(now.key > val) return Rank(now.son[0], val);
    if(now.key < val) return le + Rank(now.son[1], val);
    return lt + 1;
}
int Prec(int x, int val) {
    if(x == 0) return 0;
    SplayTree& now = S[x];
    if(val <= now.key) return Prec(now.son[0], val);
    int ret = Prec(now.son[1], val);
    if(ret == 0) ret = x;
    return ret;
}
int Succ(int x, int val) {
    if(x == 0) return 0;
    SplayTree& now = S[x];
    if(val >= now.key) return Succ(now.son[1], val);
    int ret = Succ(now.son[0], val);
    if(ret == 0) ret = x;
    return ret;
}
void DelSec(int a, int b) {
    Insert(a); Insert(b);
    int lst = Prec(root, a);
    int nxt = Succ(root, b);
    Splay(lst, 0); Splay(nxt, lst);
    S[nxt].son[0] = 0;
}
char s[55];

inline int read() {
    int ret = 0;
    char ch = getchar();
    while(ch < '0' || ch > '9') ch = getchar();
    while(ch >= '0' && ch <= '9'){
        ret = ret*10 + ch - '0';
        ch = getchar();
    }
    return ret;
}

int main(){
    init();
    Insert(-inf);
    Insert(inf);

    int n = read();
    for(int cs = 1; cs <= n; ++cs) {
        scanf("%s", s);
        if(s[0] == 'I') {
            int k = read();
            Insert(k);
        } else if(s[0] == 'Q') {
            int k = read();
            int id = Find(root, k);
            if(id){
                printf("%d\n", k);
                continue;
            }
            id = Prec(root, k);
            printf("%d\n", S[id].key);
        } else {
            int a = read();
            int b = read();
            DelSec(a, b);
        }
    }
    return 0;
}
\end{lstlisting}

\subsubsection{区间翻转}

您需要维护一个序列,其中需要提供以下操作:

翻转一个区间,例如原有序列是 $5, 4, 3, 2, 1$ ,翻转区间是 $[2, 4]$ 的话,结果是 $5, 2, 3, 4, 1$ 。

第一行为 $n, m$

n 表示初始序列有 n 个数,这个序列依次是 $1, 2, \cdots, n − 1, n$

m 表示翻转操作次数。

接下来 m 行每行两个数 $[l, r]$ , 表示翻转区间 $[l, r]$,数据保证 $1 \le l \le r \le n$ 。

\begin{lstlisting}[language=C++]
const int maxn = 1e5 + 5;
const int inf = 0x3f3f3f3f;
struct SplayTree {
    int son[2];
    int fa, key, sz;
    int lazy;
} S[maxn];
int root, tot;
void init() {
    root = tot = 0;
    S[0].son[0] = S[0].son[1] = 0;
    S[0].fa = S[0].key = 0;
    S[0].sz = S[0].lazy = 0;
}
int AddNode(int key, int fa) {
    ++tot;
    S[tot].son[0] = S[tot].son[1] = 0;
    S[tot].fa = fa; S[tot].key = key;
    S[tot].lazy = 0; S[tot].sz = 1;
    return tot;
}
void PushUp(int x) {
    if(x == 0) return ;
    SplayTree& now = S[x];
    SplayTree& ls = S[now.son[0]];
    SplayTree& rs = S[now.son[1]];
    now.sz = 1 + ls.sz + rs.sz;
}
int Build(int lf, int rt, int fa) {
    if(lf > rt) return 0;
    int mid = (lf + rt) >> 1;
    int id = AddNode(mid, fa);
    SplayTree& now = S[id];
    now.son[0] = Build(lf, mid - 1, id);
    now.son[1] = Build(mid + 1, rt, id);
    PushUp(id);
    return id;
}
int Judge(int x) {
    if(S[S[x].fa].son[1] == x) return 1;
    return 0;
}
void SetSon(int x, int fa, int id) {
    if(x) S[x].fa = fa;
    if(fa == 0) {
        root = x;
        return ;
    }
    S[fa].son[id] = x;
    PushUp(fa);
}
void PushDown(int x) {
    SplayTree& now = S[x];
    if(now.lazy == 0) return ;
    now.lazy = 0;
    S[now.son[0]].lazy ^= 1;
    S[now.son[1]].lazy ^= 1;
    swap(now.son[0], now.son[1]);
}
void Rotate(int x) {
    int f = S[x].fa, ff = S[f].fa;
    int sx = Judge(x), sf = Judge(f);
    int p = S[x].son[sx ^ 1];
    SetSon(p, f, sx);
    SetSon(x, ff, sf);
    SetSon(f, x, sx ^ 1);
    PushUp(f); PushUp(x);
}
void Splay(int x, int y) {
    while(S[x].fa != y) {
        int f = S[x].fa, ff = S[f].fa;
        if(ff != y) {
            if(Judge(x) == Judge(f)) Rotate(f);
            else Rotate(x);
        }
        Rotate(x);
    }
    if(y == 0) root = x;
}
int Find(int x, int k) {
    if(x == 0) return 0;
    SplayTree& now = S[x];
    PushDown(x);
    int lt = S[now.son[0]].sz;
    int le = lt + 1;
    if(k <= lt) return Find(now.son[0], k);
    if(k > le) return Find(now.son[1], k - le);
    return x;
}
void Filp(int lf, int rt) {
    int lst = Find(root, lf - 1 + 1);
    int nxt = Find(root, rt + 1 + 1);
    Splay(lst, 0);
    Splay(nxt, lst);
    SplayTree& now = S[S[nxt].son[0]];
    now.lazy ^= 1;
}
int n, m, cnt = 0;
void solve(int x) {
    if(x == 0) return ;
    SplayTree& now = S[x];
    PushDown(x);
    if(now.key > 0) solve(now.son[0]);
    if(now.key > 0 && now.key < n + 1) {
        if(++cnt > 1) printf(" ");
        printf("%d", S[x].key);
    }
    if(now.key < n + 1) solve(now.son[1]);
}
int main(){
    init();
    scanf("%d%d", &n, &m);
    root = Build(0, n + 1, 0);
    for(int cm = 1; cm <= m; ++cm) {
        int lf, rt; scanf("%d%d", &lf, &rt);
        Filp(lf, rt);
    }
    solve(root); puts("");
    return 0;
}
\end{lstlisting}

\subsubsection{区间加减,区间第 K 大}

插入、区间加、区间减、区间第 k 大

\begin{lstlisting}[language=C++]
typedef long long ll;
const int maxn = 1e5 + 5;
const int inf = 0x3f3f3f3f;
struct SplayTree {
    int son[2], fa;
    ll sz, cnt, key, lazy;
    void pt() {
        printf("Son: %d %d, Fa %d\n", son[0], son[1], fa);
        printf("sz: %lld cnt: %lld key: %lld lazy %lld\n", sz, cnt, key, lazy);
    }
} S[maxn];
int root, tot;
void init() {
    root = tot = 0;
    S[0].son[0] = S[0].son[1] = S[0].fa = 0;
    S[0].key = S[0].lazy = 0;
    S[0].sz = S[0].cnt = 0;
}
void PushUp(int x) {
    if(x == 0) return ;
    SplayTree& now = S[x];
    now.sz = now.cnt;
    if(now.son[0]) {
        SplayTree& ls = S[now.son[0]];
        now.sz += ls.sz;
    }
    if(now.son[1]) {
        SplayTree& rs = S[now.son[1]];
        now.sz += rs.sz;
    }
}
void PushDown(int x) {
    if(x == 0) return ;
    SplayTree& now = S[x];
    if(now.lazy == 0) return ;
    if(now.son[0]) {
        SplayTree& ls = S[now.son[0]];
        ls.key += now.lazy;
        ls.lazy += now.lazy;
    }
    if(now.son[1]) {
        SplayTree& rs = S[now.son[1]];
        rs.key += now.lazy;
        rs.lazy += now.lazy;
    }
    now.lazy = 0;
}
int Judge(int x) {
    if(S[S[x].fa].son[1] == x) return 1;
    return 0;
}
void SetSon(int x, int fa, int y) {
    if(x) S[x].fa = fa;
    if(fa == 0) {
        root = x;
        return ;
    }
    S[fa].son[y] = x;
    PushUp(fa);
}
void Rotate(int x) {
    int f = S[x].fa, ff = S[f].fa;
    int sx = Judge(x), sf = Judge(f);
    int p = S[x].son[sx ^ 1];
    SetSon(p, f, sx);
    SetSon(x, ff, sf);
    SetSon(f, x, sx ^ 1);
    PushUp(f); PushUp(x);
}
void Splay(int x, int y) {
    while(S[x].fa != y) {
        int f = S[x].fa, ff = S[f].fa;
        if(ff != y) {
            if(Judge(x) == Judge(f)) Rotate(f);
            else Rotate(x);
        }
        Rotate(x);
    }
    if(y == 0) root = x;
}
void Ist(int& x, int& hot, ll val, int fa) {
    if(x == 0) {
        x = ++tot;
        hot = x;
        SplayTree& now = S[x];
        now.key = val; now.fa = fa;
        now.son[0] = now.son[1] = 0;
        now.sz = now.cnt = 1; now.lazy = 0;
        return ;
    }
    PushDown(x);
    SplayTree& now = S[x];
    if(val == now.key) {
        ++now.cnt;
        hot = x;
    } else {
        if(val < now.key) Ist(now.son[0], hot, val, x);
        else Ist(now.son[1], hot, val, x);
    }
    PushUp(x);
}
void Insert(ll val) {
    int hot;
    Ist(root, hot, val, 0);
    Splay(hot, 0);
}
int Kth(int x, int k) {
    if(x == 0) return 0;
    PushDown(x);
    SplayTree& now = S[x];
    ll lt = S[now.son[0]].sz;
    ll le = lt + now.cnt;
    if(k <= lt) return Kth(now.son[0], k);
    if(k > le) return Kth(now.son[1], k - le);
    Splay(x, 0);
    return x;
}
int Succ(int x, ll val) {
    if(x == 0) return 0;
    PushDown(x);
    SplayTree& now = S[x];
    if(val >= now.key) return Succ(now.son[1], val);
    int ret = Succ(now.son[0], val);
    if(ret == 0) ret = x;
    return ret;
}
void Update(ll val) {
    if(root == 0) return ;
    SplayTree& now = S[root];
    now.key += val;
    now.lazy += val;
}
ll mi, ans;
void clr() {
    int id = Succ(root, mi - 1LL);
    if(id == 0) {
        ans += S[root].sz;
        root = 0;
        return ;
    }
    Splay(id, 0);
    ans += S[S[id].son[0]].sz;
    S[id].son[0] = 0;
    PushUp(id);
}
char opt[5];
int main(){
    int n; scanf("%d%lld", &n, &mi);
    init(); ans = 0;
    for(int cn = 1; cn <= n; ++cn) {
        scanf("%s", opt);
        if(opt[0] == 'I') {
            ll k; scanf("%lld", &k);
            if(k >= mi) Insert(k);
        } else if(opt[0] == 'A') {
            ll k; scanf("%lld", &k);
            Update(k);
        } else if(opt[0] == 'S') {
            ll k; scanf("%lld", &k);
            Update(-k);
        } else {
            int k; scanf("%d", &k);
            int num = S[root].sz;
            if(num < k) puts("-1");
            else printf("%lld\n", S[Kth(root, num - k + 1)].key);
        }
        clr();
    }
    printf("%lld\n", ans);
    return 0;
}
\end{lstlisting}

\subsection{Link-Cut Tree}

\begin{lstlisting}[language=C++]
const int maxn = 3e5 + 5;
struct LCT {
    int tp, st[maxn];
    int fa[maxn], ch[maxn][2];
    bool lazy[maxn];
    int mx[maxn], val[maxn];
    void init(int x, int y) {
        fa[x] = ch[x][0] = ch[x][1] = 0;
        lazy[x] = false;
        mx[x] = x; val[x] = y;
    }
    bool is_root(int x) {
        return ch[fa[x]][0] != x && ch[fa[x]][1] != x;
    }
    void push_up(int x) {
        int lc = ch[x][0], rc = ch[x][1];
        mx[x] = x;
        if (val[mx[x]] < val[mx[lc]]) mx[x] = mx[lc];
        if (val[mx[x]] < val[mx[rc]]) mx[x] = mx[rc];
    }
    void push_down(int x) {
        if (!lazy[x]) return ;
        int lc = ch[x][0], rc = ch[x][1];
        lazy[x] = !lazy[x];
        lazy[lc] = !lazy[lc];
        lazy[rc] = !lazy[rc];
        swap(ch[x][0], ch[x][1]);
    }
    void rotate(int x) {
        int y = fa[x], z = fa[y], lc, rc;
        if (ch[y][0] == x) lc = 0;
        else lc = 1;
        rc = lc ^ 1;
        if (!is_root(y)) {
            if (ch[z][0] == y) ch[z][0] = x;
            else ch[z][1] = x;
        }
        fa[x] = z; fa[y] = x; fa[ch[x][rc]] = y;
        ch[y][lc] = ch[x][rc]; ch[x][rc] = y;
        push_up(y); push_up(x);
    }
    void splay(int x) {
        tp = 0; st[tp++] = x;
        for (int i = x; !is_root(i); i = fa[i]) st[tp++] = fa[i];
        while (tp > 0) push_down(st[--tp]);
        while (!is_root(x)) {
            int y = fa[x], z = fa[y];
            if (!is_root(y)) {
                if ((ch[y][0] == x) == (ch[z][0] == y)) rotate(x);
                else rotate(y);
            }
            rotate(x);
        }
    }
    void access(int x) {
        for (int t = 0; x != 0; t = x, x = fa[x]) {
            splay(x); ch[x][1] = t; push_up(x);
        }
    }
    void make_root(int x) {
        access(x); splay(x); lazy[x] = !lazy[x];
    }
    void link(int x, int y) {
        make_root(x); fa[x] = y;
    }
    void cut(int x, int y) {
        make_root(x); access(y); splay(y);
        ch[y][0] = fa[x] = 0; push_up(y);
    }
    int query(int x, int y) {
        make_root(x); access(y); splay(y);
        int ret = mx[y];
        while (ch[y][0] != 0) {
            y = ch[y][0];
            if (val[ret] < val[mx[y]]) val[ret] = mx[y];
        }
        if (x != y) return -1;
        return ret;
    }
} splay;
\end{lstlisting}

\subsection{Euler Tour Tree}

\begin{lstlisting}[language=C++]
struct Euler_Tour_Tree{
    stack<int> sta;
    int tot;
    int tagp[300005];//记录ETT的点的出现第一次
    map<int,int> e[300005];//记录ETT的边出现
    struct node {
        int ls,rs,fa,data,size;
        int rd;
        bool tag,stag;//是否有第一次出现的点
    } v[600005];
    int new_node(int pt){
        int x;
        if(!sta.empty())x=sta.top(),sta.pop();
        else x=++tot;
        v[x].fa=v[x].ls=v[x].rs=0;
        v[x].tag=v[x].stag=0;
        v[x].size=1;
        v[x].data=pt;
        v[x].rd=rand();
        return x;
    }
    void del_node(int x){
        sta.push(x);
        return;
    }
    void push_up(int rt){
        v[rt].stag=v[v[rt].ls].stag|v[v[rt].rs].stag|v[rt].tag;
        v[rt].size=v[v[rt].ls].size+v[v[rt].rs].size+1;
    }
    int mer_ge(int x,int y){
        if(!x||!y)return x|y;
        if(v[x].rd<=v[y].rd){
            v[x].rs=mer_ge(v[x].rs,y);
            v[v[x].rs].fa=x;
            push_up(x);
            return x;
        } else if(v[x].rd>=v[y].rd) {
            v[y].ls=mer_ge(x,v[y].ls);
            v[v[y].ls].fa=y;
            push_up(y);
            return y;
        }
    }
    int get_root(int x){
        while(v[x].fa)x=v[x].fa;
        return x;
    }
    void split(int x,int d,int &r1,int &r2){
        int s1=v[x].ls,s2=v[x].rs;
        if(d==-1){
            v[x].rs=0;
            push_up(x);
            s1=x;
            v[s2].fa=0;
        }
        if(d==1){
            v[x].ls=0;
            push_up(x);
            s2=x;
            v[s1].fa=0;
        }
        if(d!=1&&d!=-1)v[s1].fa=v[s2].fa=0;
        while(v[x].fa){
            int p=v[x].fa;
            if(v[p].ls==x){
                v[p].ls=s2;
                v[s2].fa=p;
                v[s1].fa=0;
                s2=p;
            }
            if(v[p].rs==x){
                v[p].rs=s1;
                v[s1].fa=p;
                v[s2].fa=0;
                s1=p;
            }
            push_up(x=p);
        }
        r1=s1;
        r2=s2;
    }
    int get_p(int x){
        if(e[x].empty())return 0;
        return e[e[x].begin()->first][x];
    }
    void link(int x,int y){
        int p1=new_node(y),p2=new_node(x);
        if(tagp[x]==-1){
            v[p2].tag=v[p2].stag=1;
            tagp[x]=p2;
        }
        if(tagp[y]==-1){
            v[p1].tag=v[p1].stag=1;
            tagp[y]=p1;
        }
        int t1,t2,t3;
        if(e[x].empty())t1=t3=0;
        else split(e[x].begin()->second,1,t1,t3);
        if(e[y].empty())t2=0;
        else {
            int tt1,tt2;
            split(e[y].begin()->second,1,tt1,tt2);
            t2=mer_ge(tt2,tt1);
        }
        e[x][y]=p1;
        e[y][x]=p2;
        mer_ge(mer_ge(mer_ge(t1,p1),t2),mer_ge(p2,t3));
    }
    void cut(int x,int y) {
        int p1=e[x][y],p2=e[y][x];
        e[x].erase(e[x].find(y));
        e[y].erase(e[y].find(x));
        int t11,t12;
        split(p1,0,t11,t12);
        if(get_root(p2)==t11){
            int t21,t22;
            split(p2,0,t21,t22);
            mer_ge(t21,t12);
        }else{
            int t21,t22;
            split(p2,0,t21,t22);
            mer_ge(t11,t22);
        }
        if(v[p1].tag){
            if(e[y].empty())tagp[y]=-1;
            else{
                tagp[y]=get_p(y);
                v[tagp[y]].tag=1;
                for(int i=tagp[y];i;i=v[i].fa)push_up(i);
            }
        }
        if(v[p2].tag){
            if(e[x].empty())tagp[x]=-1;
            else{
                tagp[x]=get_p(x);
                v[tagp[x]].tag=1;
                for(int i=tagp[x];i;i=v[i].fa)push_up(i);
            }
        }
        del_node(p1);
        del_node(p2);
    }
    bool islink(int x, int y) {
        int p1 = get_p(x), p2 = get_p(y);
        if (!p1 || !p2) return false;
        return get_root(p1) == get_root(p2);
    }
} ETT;
\end{lstlisting}

\subsection{李超树(二维平面维护直线)}

\begin{lstlisting}[language=C++]
const int maxn = 1e5 + 5;
int tree[maxn<<2];
char s[15];
double k[maxn], b[maxn];
int tot;
double cal(int id, double x) {
    double ret = k[id]*x + b[id];
    return ret;
}
int sgn(double x) {
    if(fabs(x) < eps) return 0;
    if(x < 0) return -1;
    return 1;
}
double getPos(double k1, double b1, double k2, double b2) {
    double ret = (b2 - b1)/(k1 - k2);
    return ret;
}
void ist(int x, int lf, int rt, int id) {
    if(tree[x] == 0) {
        tree[x] = id;
        return ;
    }
    double vl1 = cal(id, (double)lf), vl2 = cal(tree[x], (double)lf);
    double vr1 = cal(id, (double)rt), vr2 = cal(tree[x], (double)rt);
    int dt1 = sgn(vl1 - vl2), dt2 = sgn(vr1 - vr2);
    if(dt1 <= 0 && dt2 <= 0) return ;
    if(dt1 >= 0 && dt2 >= 0) {
        tree[x] = id;
        return ;
    }
    int mid = (lf + rt) >> 1;
    double dm = mid;
    double pos = getPos(k[id], b[id], k[tree[x]], b[tree[x]]);
    if(dt1 >= 0) {
        if(sgn(pos - dm) <= 0) {
            ist(x << 1, lf, mid, id);
            tree[x] = tree[x];
        } else {
            ist(x << 1 | 1, mid + 1, rt, tree[x]);
            tree[x] = id;
        }
    } else {
        if(sgn(pos - dm) <= 0) {
            ist(x << 1, lf, mid, tree[x]);
            tree[x] = id;
        } else {
            ist(x << 1 | 1, mid + 1, rt, id);
            tree[x] = tree[x];
        }
    }
}
double query(int x, int lf, int rt, double pos) {
    double ret = 0.0;
    if(tree[x] != 0) ret = max(ret, cal(tree[x], pos));
    if(lf == rt) return ret;
    int mid = (lf + rt) >> 1;
    double dm = mid;
    if(sgn(dm - pos) >= 0) ret = max(ret, query(x << 1, lf, mid, pos));
    else ret = max(ret, query(x << 1 | 1, mid + 1, rt, pos));
    return ret;
}
\end{lstlisting}

\subsection{左偏树(可并堆)(log(n)时间内合并两个堆)}

\begin{lstlisting}[language=C++]
template<class T>
struct LeftistTree {
    static const int NodeNum = 1e6 + 5;
    int tot;
    struct Data {
        T key;
        int l, r, dist;
    } tr[NodeNum << 1];
    void Init() {
        tot = 0;
    }
    LeftistTree() {
        Init();
    }
    int New(T key) {
        ++tot;
        tr[tot].l = tr[tot].r = 0;
        tr[tot].key = key;
        tr[tot].dist = 0;
        return tot;
    }
    int Merge(int x, int y) {
        if(x == 0) return y;
        if(y == 0) return x;
        if(tr[y].key > tr[x].key) swap(x, y);
        tr[x].r = Merge(tr[x].r, y);
        if(tr[tr[x].l].dist < tr[tr[x].r].dist)
            swap(tr[x].l, tr[x].r);
        tr[x].dist = tr[tr[x].r].dist + 1;
        return x;
    }
    void Push(int &x, T key) {
        x = Merge(x, New(key));
    }
    T Pop(int& x) {
        T ret = tr[x].key;
        x = Merge(tr[x].l, tr[x].r);
        return ret;
    }
    T Top(int& x) {
        return tr[x].key;
    }
};
LeftistTree<int> Tree;
\end{lstlisting}

\subsection{哈希表}

\begin{lstlisting}[language=C++]
template <typename T1, typename T2>
class Hash {
public:
    static const int HashMod = 7679977;
    // 99991, 3000017, 7679977, 19260817
    int tp, st[HashMod + 5];
    bool vis[HashMod + 5];
    T1 h[HashMod + 5];
    T2 val[HashMod + 5];
    int locate(const T1& x) const {
        int p = x%HashMod;
        while(vis[p] && h[p] != x) {
            if(++p == HashMod) p = 0;
        }
        return p;
    }
    Hash() {
        tp = 0;
        memset(vis, false, sizeof(false));
    }
    void ist(const T1& x, const T2& y) {
        const int p = locate(x);
        assert(!vis[p] || h[p] == x);
        if (!vis[p]) {
            h[p] = x; vis[p] = true;
            val[p] = y;
            st[++tp] = p;
        } else if (h[p] == x) {
            val[p] = y;
        } else {
            assert(false);
        }
    }
    bool get(const T1& x, T2& y) const {
        const int p = locate(x);
        if(vis[p] && h[p] == x) {
            y = val[p];
            return true;
        }
        return false;
    }
    void prt() {
        for (int i = 1; i <= tp; ++i) {
            if (i > 1) cout << " ";
            cout << h[st[i]];
        }
        cout << endl;
    }
    void clr() {
        while (tp > 0) vis[st[tp--]] = false;
    }
};
Hash<int, int> H;
\end{lstlisting}