

\section{杂项}

\subsection{java 输入输出挂}

\begin{lstlisting}[language=java]
static class InputReader {
    public BufferedReader reader;
    public StringTokenizer tokenizer;
    public InputReader(InputStream stream) {
        reader = new BufferedReader(new InputStreamReader(stream), 32768);
        tokenizer = null;
    }
    public String next() {
        while(tokenizer == null || !tokenizer.hasMoreTokens()) {
            try {
                tokenizer = new StringTokenizer(reader.readLine());
            } catch (IOException e) {
                throw new RuntimeException(e);
            }
        }
        return tokenizer.nextToken();
    }
    public int nextInt() {
        return Integer.parseInt(next());
    }
    public long nextLong() {
        return Long.parseLong(next());
    }
    public double nextDouble() {
        return Double.parseDouble(next());
    }
}
\end{lstlisting}

\subsection{模拟退火(Simulate Anneal)}

Sta:状态结构体

J(y):在状态y时的评价函数值

r: 用于控制降温的快慢

T: 系统的温度,系统初始应该要处于一个高温的状态

$T_{min}$ :温度的下限,若温度 $T$ 达到 $T_{min}$ ,则停止搜索

\begin{lstlisting}[language=C++]
    const double T_min = 1e-3;
const double r = 0.98;
struct Sta{
    int x, y;
    Sta(int x = 0, int y = 0):x(x), y(y){}
};
Sta move(Sta now){
    Sta ret;
    return ret;
}
int J(Sta x){
    int ret = 0;

    return ret;
}
int simulate_anneal(double T){
    srand(time(NULL));
    int ret = 0;
    Sta now = Sta();
    while(T > T_min){
        Sta nxt = move(now);
        int j1 = J(now), j2 = J(nxt);
        int dE = j2 - j1;
        if(ret < j1) ret = j1;
        if(dE >= 0){ //表达移动后得到更优解,则总是接受移动
            now = nxt; //接受从Y(i)到Y(i+1)的移动
        }else{
            // 函数exp(dE/T)的取值范围是(0, 1),dE/T越大,则exp(dE/T)也
            double val = exp(dE/T), rn = (double)(rand()%123456)/123456.0;
            if(val > rn)
                now = nxt; //接受从Y(i)到Y(i+1)的移动
        }
        T = r*T ; //降温退火 ,0<r<1 。r越大,降温越慢;r越小,降温越快
        /*
        * 若r过大,则搜索到全局最优解的可能会较高,但搜索的过程也就较长。若r过小,则搜索的过程会很快,但最终可能会达到一个局部最优值
        */
    }
    return ret;
}
\end{lstlisting}

\subsection{慢速乘(防止乘法溢出)}

\begin{lstlisting}[language=C++]
ll mulmod(ll a, ll b){
    ll ans = 0;
    for(; b; b>>=1){
        if(b & 1) ans = (ans + a) % mod;
        a = a * 2 % mod;
    }
    return ans;
}
\end{lstlisting}

\subsection{三分查找}

\begin{lstlisting}[language=C++]
const ll linf = 0x3f3f3f3f3f3f3f3f;
ll cal(int x){
    ll ret = 0;

    return ret;
}
ll solve(int lf, int rt){
    while(lf + 2 < rt){
        int len = rt - lf;
        int mid1 = len/3 + lf;
        int mid2 = len/3*2 + lf;
        // 此处为三分查找最小值
        if(cal(mid1) >= cal(mid2)) lf = mid1;
        else rt = mid2;
    }
    ll ret = linf;
    for(int i = lf; i <= rt; ++i) ret = min(ret, cal(i));
    return ret;
}
\end{lstlisting}

\subsection{分数类}

\begin{lstlisting}[language=C++]
template <typename T>
struct Frac {
    T up, down;
    Frac(T u = 0, T d = 1) {
        T g = __gcd(u, d);
        up = u/g; down = d/g;
        if(up > 0 && down < 0) {
            up = -up;
            down = -down;
        }
    }
    bool operator <(const Frac& f) const {
        if(up*f.down < down*f.up) return true;
        return false;
    }
    bool operator ==(const Frac& f) const {
        if(up == f.up && down == f.down) return true;
        return false;
    }
    Frac operator +(const Frac& f) const {
        T g = __gcd(down, f.down);
        T u = f.down/g*up + down/g*f.up;
        T d = down/g*f.down;
        return Frac(u, d);
    }
    Frac operator -(const Frac& f) const {
        T g = __gcd(down, f.down);
        T u = f.down/g*up - down/g*f.up;
        T d = down/g*f.down;
        return Frac(u, d);
    }
    Frac operator *(const Frac& f) const {
        T g1 = __gcd(up, f.down);
        T g2 = __gcd(f.up, down);
        T u = (up/g1)*(f.up/g2);
        T d = (f.down/g1)*(down/g2);
        return Frac(u, d);
    }
    Frac operator /(const Frac& f) const {
        T g1 = __gcd(up, f.up);
        T g2 = __gcd(down, f.down);
        T u = (up/g1)*(f.down/g2);
        T d = (f.up/g1)*(down/g2);
        return Frac(u, d);
    }
    void pt() {
        cout << up;
        // 根据需要修改
        if(down != 1) cout << "/" << down;
        cout << endl;
    }
};
Frac<ll> num;
\end{lstlisting}

\subsection{十进制水仙花数}

\begin{lstlisting}[language=C++]
    // [0, 88] 十进制自然数中共 89 个水仙花数
string str[90] = {
    "0", "1", "2", "3", "4", "5", "6", "7", "8", "9",
    "153", "370", "371", "407",
    "1634", "8208", "9474",
    "54748", "92727", "93084",
    "548834",
    "1741725", "4210818", "9800817", "9926315",
    "24678050", "24678051", "88593477",
    "146511208", "472335975", "534494836", "912985153",
    "4679307774",
    "32164049650", "32164049651", "40028394225", "42678290603", "44708635679", "49388550606", "82693916578", "94204591914",
    "28116440335967",
    "4338281769391370", "4338281769391371",
    "21897142587612075", "35641594208964132", "35875699062250035",
    "1517841543307505039", "3289582984443187032", "4498128791164624869", "4929273885928088826",
    "63105425988599693916",
    "128468643043731391252", "449177399146038697307",
    "21887696841122916288858", "27879694893054074471405", "27907865009977052567814", "28361281321319229463398", "35452590104031691935943",
    "174088005938065293023722", "188451485447897896036875", "239313664430041569350093",
    "1550475334214501539088894", "1553242162893771850669378", "3706907995955475988644380", "3706907995955475988644381", "4422095118095899619457938",
    "121204998563613372405438066", "121270696006801314328439376", "128851796696487777842012787", "174650464499531377631639254", "177265453171792792366489765",
    "14607640612971980372614873089", "19008174136254279995012734740", "19008174136254279995012734741", "23866716435523975980390369295",
    "1145037275765491025924292050346", "1927890457142960697580636236639", "2309092682616190307509695338915",
    "17333509997782249308725103962772",
    "186709961001538790100634132976990", "186709961001538790100634132976991",
    "1122763285329372541592822900204593",
    "12639369517103790328947807201478392", "12679937780272278566303885594196922",
    "1219167219625434121569735803609966019",
    "12815792078366059955099770545296129367",
    "115132219018763992565095597973971522400", "115132219018763992565095597973971522401"
};
\end{lstlisting}

\subsection{牛顿迭代法(大数开方)}

\begin{lstlisting}[language=java]
import java.util.*;
import java.math.BigInteger;
public class Main {
    public static void main(String[] args) {
        Scanner cin = new Scanner(System.in);
        BigInteger n = cin.nextBigInteger();
        int len = n.toString().length()/2;
        BigInteger x0 = new BigInteger(n.toString().substring(len));
        while(true) {
            BigInteger x1 = x0.multiply(x0).add(n).divide(BigInteger.valueOf(2).multiply(x0));
            if(x1.compareTo(x0) == 0) break;
            x0 = x1;
        }
        System.out.println(x0);
        cin.close();
    }
}
\end{lstlisting}


\subsection{一些奇奇怪怪的性质}

\begin{itemize}
    \item 给定母串,多次询问子串,保证子串总长度小于等于sum,则有,其询问的字符串长度种类数最多为 $O(\sqrt sum)$
    \item 若顶点坐标为整数,坐标值的范围不超过 M 的凸多边形的顶点数只有 $O((sqrt(M))^\frac{2}{3}) (O(M^\frac{1}{3})$ ?)
    \item 字典序最小的拓扑序:拓扑排序 + 优先队列
    \item 最小值最早出现(次小值在最小值最早出现的前提下最早出现......)的拓扑序:反向建图,跑出字典序最大的拓扑排序,倒序输出
    \item 随机边权的图,最短路径上的边数很少,生成树不唯一的概率为 0
\end{itemize}

\subsection{丢人}

\begin{itemize}
    \item 调用rand函数之前必须使用srand设置随机种子
    \item 二分输出方案时, 最后要再judge一次正解
    \item 对于负数 /2 向0取整, >> 向下取整
    \item 二叉树(线段树,堆...)中注意儿子是<<, 父亲是>>
\end{itemize}