

\section{字符串}

\subsection{最长回文子串 (Manacher 算法)}

\begin{lstlisting}[language=C++]
const int maxn = 5e5 + 5;
int len;
char t[maxn << 1];
int p[maxn << 1];
char str[maxn];
void Manacher(){
    int mx = 0, id = -1;
    len = strlen(str);
    for(int i = 0; i <= len*2; ++i) {
        p[i] = 0;
        t[i] = 0;
    }
    for(int i = 0; i <= len*2; ++i) {
        if((i & 1) == 1) {
            t[i] = str[i/2];
        } else {
            t[i] = '*';
        }
    }
    for(int i = 1; i < len*2; ++i) {
        if(mx > i) {
            p[i] = min(p[2*id - i], mx - i);
        } else {
            p[i] = 1;
        }
        while(i - p[i] >= 0 && t[i + p[i]] == t[i - p[i]]) {
            ++p[i];
        }
        if(p[i] + i > mx) {
            mx = p[i] + i;
            id = i;
        }
    }
    for(int i = 0; i < len*2; ++i) {
        int pos = i/2;
        int plen = p[i] - 1;
        int lf = pos - plen/2;
        int rt = lf + plen - 1;
        if(lf > rt) {
            continue;
        }
        // str[lf, rt] is longest palindrome around str[pos]
        printf("[%d, %d]\n", lf, rt);
        for(int j = lf; j <= rt; ++j) {
            putchar(str[j]);
        }
        puts("");
    }
}
\end{lstlisting}

\subsection{KMP 算法(单母串匹配)}

\subsubsection{模板}

\begin{lstlisting}[language=C++]
const int maxn = 1e6 + 10;
int nxt[maxn];
char s1[maxn], s2[maxn];
int len1, len2;
void getNext(){
    int i = 0, j = -1;
    nxt[0] = -1;
    while(i < len2) {
        if(j == -1 || s2[i] == s2[j]) nxt[++i] = ++j;
        else j = nxt[j];
    }
}

void kmp() {
    int i = 0, j = 0;
    getNext();
    while(i < len1){
        if(j == -1 || s1[i] == s2[j]) ++i, ++j;
        else j = nxt[j];
        // 匹配成功 s1[i - len2, i) = s2[0, len2)
        if(j == len2) { 
//            printf("%d\n", i - len2);
            j = nxt[j];
        }
    }
}
\end{lstlisting}

\subsubsection{循环节}

如果对于 next 数组中的 i, 符合 

$$i\%( i - next[i] ) == 0 \&\& next[i] != 0$$

则说明字符串循环,而且

循环节长度为: i - next[i]

循环次数为: i / ( i - next[i] )

对于一个不完整的循环串要补充多少个才能使得其完整

$$(循环节长度) - len\%(循环节长度)$$

即

$$(len - next[len]) - len\%(len - next[len])$$

\subsubsection{拓展kmp (求 T 与 {S的每一个后缀:s[i ~ n - 1]} 的最长公共前缀)}

\begin{lstlisting}[language=C++]
int nxt[maxn], ex[maxn];
void getNext(char *str) {
    int i = 0, j, pos, len = strlen(str);
    nxt[0] = len;
    while(str[i] == str[i + 1] && i + 1 < len) ++i;
    nxt[1] = i; pos = 1;
    for(i = 2; i < len; ++i){
        if(nxt[i - pos] + i < nxt[pos] + pos) {
            nxt[i] = nxt[i - pos];
            continue;
        }
        j = nxt[pos] + pos - i;
        if(j < 0) j = 0;
        while(i + j < len && str[j] == str[j + i]) ++j;
        nxt[i] = j; pos = i;
    }
}

void exkmp(char *s1, char *s2){
    int i = 0, j, pos, l1 = strlen(s1), l2 = strlen(s2);
    getNext(s2);
    while(s1[i] == s2[i] && i < l2 && i < l1) ++i;
    ex[0] = i; pos = 0;
    for(i = 1; i < l1; ++i){
        if(nxt[i - pos] + i < ex[pos] + pos) {
            ex[i] = nxt[i - pos];
            continue;
        }
        j = ex[pos] + pos - i;
        if(j < 0) j = 0;
        while(i + j < l1 && j < l2 && s1[j + i] == s2[j]) ++j;
        ex[i] = j; pos = i;
    }
}
\end{lstlisting}

\subsection{AC自动机(多母串匹配)}

\begin{lstlisting}[language=C++]
const int maxn = 1e6 + 5;
const int TrieSize = 1e6 + 5;
const int SigmaSize = 26;

namespace Trie {
    int T[TrieSize][SigmaSize];
    int num[TrieSize];
    bool vis[TrieSize];
    int tot;
    void init() {
        tot = 0;
        num[0] = 0;
        vis[0] = false;
        for(int i = 0; i < SigmaSize; ++i) {
            T[0][i] = 0;
        }
    }
    int getID(char ch) {
        int ret = ch - 'a';
        return ret;
    }
    int newNode() {
        ++tot;
        num[tot] = 0;
        vis[tot] = false;
        for(int i = 0; i < SigmaSize; ++i) {
            T[tot][i] = 0;
        }
        return tot;
    }
    void ist(char s[]) {
        int len = strlen(s);
        int now = 0;
        for(int i = 0; i < len; ++i) {
            int id = getID(s[i]);
            if(T[now][id] == 0) {
                T[now][id] = newNode();
            }
            now = T[now][id];
        }
        ++num[now];
    }
}
namespace ACAM {
    using namespace Trie;
    int F[TrieSize];
    int lst[TrieSize];
    void getFail() {
        queue<int> q;
        F[0] = lst[0] = 0;
        for(int i = 0; i < SigmaSize; ++i) {
            int to = T[0][i];
            if(to != 0) {
                F[to] = lst[to] = 0;
                q.push(to);
            }
        }
        while(!q.empty()) {
            int now = q.front(); q.pop();
            for(int i = 0; i < SigmaSize; ++i) {
                int to = T[now][i];
                if(to == 0) {
                    T[now][i] = T[F[now]][i];
                } else {
                    q.push(to);
                    int fn = F[now];
                    while(fn != 0 && T[fn][i] == 0) {
                        fn = F[fn];
                    }
                    F[to] = T[fn][i];
                    if(num[F[to]] > 0) {
                        lst[to] = F[to];
                    } else {
                        lst[to] = lst[F[to]];
                    }
                }
            }
        }
    }
    int calc(int now) {
        int ret = 0;
        while(now != 0 && !vis[now]) {
            vis[now] = true;
            ret += num[now];
            now = lst[now];
        }
        return ret;
    }
    int query(char s[]) {
        int len = strlen(s);
        int ret = 0, now = 0;
        for(int i = 0; i < len; ++i) {
            int id = getID(s[i]);
            now = T[now][id];
            if(num[now] > 0) {
                ret += calc(now);
            } else {
                int nxt = lst[now];
                if(nxt != 0) {
                    ret += calc(nxt);
                }
            }
        }
        return ret;
    }
    void show() {
        puts("****************");
        for(int i = 0; i <= tot; ++i) {
            printf("Trie[%d]: \n", i);
            printf("num: %d, fail: %d, last: %d\n", num[i], F[i], lst[i]);
            printf("to:");
            for(int j = 0; j < SigmaSize; ++j) {
                if(T[i][j] == 0) continue;
                printf(" (%c, %d)", j + 'a', T[i][j]);
            }
            puts("");
        }
        puts("****************");
    }
}
char s[maxn];
int main() {
    int T; scanf("%d", &T);
    for(int cs = 1; cs <= T; ++cs) {
        Trie::init();
        int n; scanf("%d", &n);
        for(int i = 0; i < n; ++i) {
            scanf("%s", s);
            Trie::ist(s);
        }
        // ACAM::show();
        ACAM::getFail();
        // ACAM::show();
        scanf("%s", s);
        printf("%d\n", ACAM::query(s));
    }
    return 0;
}
\end{lstlisting}

\subsection{循环串的最小表示法}

\begin{lstlisting}[language=C++]
char s[maxn];
int getMin() {
    int n = strlen(s);
    int i = 0, j = 1, k = 0;
    while(i < n && j < n && k < n){
        int dt = s[(i + k)%n] - s[(j + k)%n];
        if(dt == 0) ++k;
        else{
            if(dt > 0) i += k + 1;
            else j += k + 1;
            if(i == j) ++j;
            k = 0;
        }
    }
    // 返回字典序最小的循环串的起始位置
    int ret = min(i, j);
    return ret;
}
\end{lstlisting}

\subsection{序列自动机}

字符串下标需从 1 开始,会方便很多

nxt[i][j] 表示第 i 个字符的下一个 j 字母出现的位置(在原串中的下标)

cnt[i][j] s表示第 i 个字符后面 j 字母的个数

\begin{lstlisting}[language=C++]
char s[maxn];
int nxt[maxn][27], pos[27];
int cnt[maxn][27], num[27];
void getSeqAM() {
    int len = strlen(s + 1);
    for(int j = 0; j < 26; ++j) {
        pos[j] = 0;
        num[j] = 0;
    }
    for(int i = len; i >= 1; --i) {
        for(int j = 0; j < 26; ++j) {
            nxt[i][j] = pos[j];
            cnt[i][j] = num[j];
        }
        int id = s[i] - 'a';
        pos[id] = i; ++num[id];
    }
    for(int j = 0; j < 26; ++j) {
        nxt[0][j] = pos[j];
        cnt[0][j] = num[j];
    }
}
\end{lstlisting}

\subsection{任意子串 Hash}

字符串下标需从 1 开始

本质为倒序排列的 key 进制数

\begin{lstlisting}[language=C++]
typedef unsigned long long ull; // %llu
const ull key = 137; // 233 769
const ull mod = 998244353; // 1000000007
ull H[maxn], xp[maxn];
void Init_Hash(char *s){
    int n = strlen(s + 1);
    H[0] = 0; xp[0] = 1;
    for(int i = 1; i <= n; ++i) H[i] = H[i - 1]*key + (s[i] - 'a');
    for(int i = 1; i <= n; ++i) xp[i] = xp[i - 1]*key;
}
ull Hash(int l, int r){
    return H[r] - H[l - 1] * xp[r - l + 1];
}
\end{lstlisting}

\subsection{shift-and 算法}

给你一个 $n$ 位数,每一位可以有 $m _ i$ 种选择。

给定一个需要匹配的母串,让你输出母串中有多少个可行的 $n$ 位子串。

\begin{lstlisting}[language=C++]
bitset<1005> b[10], ans;
char s[maxn];
int main() {
    int n; scanf("%d", &n);
    for(int i = 0; i < n; ++i) {
        int m; scanf("%d", &m);
        while(m--) {
            int x; scanf("%d", &x);
            b[x].set(i);
        }
    }
    scanf("%s", s);
    int len = strlen(s);
    for(int i = 0; i < len; ++i) {
        ans.set(0);
        ans &= b[s[i] - '0'];
        if(ans[n - 1] == 1) {
            // printf("[%d, %d]\n", i - n + 1, i);
            char ch = s[i + 1];
            s[i + 1] = '\0';
            puts(s + i - n + 1);
            s[i + 1] = ch;
        }
        ans <<= 1;
    }
    return 0;
}
\end{lstlisting}

\subsection{后缀数组}

后缀数组诱导排序(SuffixArray-InducedSort)

\begin{lstlisting}[language=C++]
const int maxn = 1e6 + 5;
namespace SA {
    int s[maxn << 1], t[maxn << 1], p[maxn], cnt[maxn], cur[maxn];
    int n, sa[maxn], rank[maxn], height[maxn], mi[maxn][25];
    #define pushS(x) sa[cur[s[x]]--] = x
    #define pushL(x) sa[cur[s[x]]++] = x
    #define inducedSort(v) fill_n(sa, n, -1); fill_n(cnt, m, 0); \
        for(int i = 0; i < n; ++i) cnt[s[i]]++; \
        for(int i = 1; i < m; ++i) cnt[i] += cnt[i-1]; \
        for(int i = 0; i < m; ++i) cur[i] = cnt[i]-1; \
        for(int i = n1 - 1; i != -1; --i) pushS(v[i]); \
        for(int i = 1; i < m; ++i) cur[i] = cnt[i-1]; \
        for(int i = 0; i < n; ++i) if(sa[i] > 0 &&  t[sa[i] - 1]) pushL(sa[i] - 1); \
        for(int i = 0; i < m; ++i) cur[i] = cnt[i]-1; \
        for(int i = n - 1;  i != -1; --i) if(sa[i] > 0 && !t[sa[i]-1]) pushS(sa[i]-1)
    void sais(int n, int m, int *s, int *t, int *p) {
        int n1 = t[n - 1] = 0, ch = rank[0] = -1, *s1 = s + n;
        for(int i = n - 2; i != -1; --i) t[i] = s[i] == s[i + 1] ? t[i + 1] : s[i] > s[i + 1];
        for(int i = 1; i < n; ++i) rank[i] = (t[i-1] && !t[i]) ? (p[n1] = i, n1++) : -1;
        inducedSort(p);
        for(int i = 0, x, y; i < n; ++i) if(~(x = rank[sa[i]])) {
            if(ch < 1 || p[x+1] - p[x] != p[y+1] - p[y]) ++ch;
            else for(int j = p[x], k = p[y]; j <= p[x + 1]; ++j, ++k)
                if((s[j] << 1 | t[j]) != (s[k] << 1 | t[k])) {++ch; break;}
            s1[y = x] = ch;
        }
        if(ch + 1 < n1) sais(n1, ch + 1, s1, t + n, p + n1);
        else for(int i = 0; i < n1; i++) sa[s1[i]] = i;
        for(int i = 0; i < n1; i++) s1[i] = p[sa[i]];
        inducedSort(s1);
    }
    template<typename T>
    int mapCharToInt(int n, const T *str) {
        int m = *max_element(str, str + n);
        fill_n(rank, m + 1, 0);
        for(int i = 0; i < n; i++) rank[str[i]] = 1;
        for(int i = 0; i < m; i++) rank[i + 1] += rank[i];
        for(int i = 0; i < n; i++) s[i] = rank[str[i]] - 1;
        return rank[m];
    }
    template<typename T>
    void suffixArray(int n, const T *str) {
        int m = mapCharToInt(++n, str);
        sais(n, m, s, t, p);
        for(int i = 0; i < n; i++) rank[sa[i]] = i;
        for(int i = 0, h = height[0] = 0; i < n - 1; ++i) {
            int j = sa[rank[i] - 1];
            while(i + h < n && j + h < n && s[i + h] == s[j + h]) ++h;
            if(height[rank[i]] = h) --h;
        }
    }
    void RMQ_init() {
        for(int i = 0; i < n; ++i) mi[i][0] = height[i + 1];
        for(int j = 1; (1 << j) <= n; ++j){
            for(int i = 0; i + (1 << j) <= n; ++i){
                mi[i][j] = min(mi[i][j - 1], mi[i + (1 << (j - 1))][j - 1]);
            }
        }
    }
    int RMQ(int L, int R) {
        int k = 0, len = R - L + 1;
        while((1 << (k + 1)) <= len) ++k;
        return min(mi[L][k], mi[R - (1 << k) + 1][k]);
    }
    int LCP(int i, int j) {
        if(rank[i] > rank[j]) swap(i, j);
        return RMQ(rank[i], rank[j] - 1);
    }
    template<typename T>
    void init(T *str){
        n = strlen(str);
        str[n] = 0;
        suffixArray(n, str);
        RMQ_init();
    }
    void print() {
        puts("Suffix Array Print Start");
        for(int i = 0; i <= n; ++i) {
            printf("s[%d] = %d, sa = %d rank = %d height = %d\n", i, s[i], sa[i], rank[i], height[i]);
        }
        puts("Suffix Array Print End");
    }
};
\end{lstlisting}

\subsection{后缀自动机(将所有子串压缩到一个 DAG 中)}

\subsubsection{模板}

\begin{lstlisting}[language=C++]
const int maxn = 1e6 + 5;
const int MaxState = 2*maxn; // 至少开字符串长度的两倍
const int SigmaSize = 26;
template<class T> class SAM{
private:
    int tot, lst, len;
    int trans[MaxState][SigmaSize], mx[MaxState], mi[MaxState], link[MaxState];
    int newState() {
        const int x = tot++;
        for(int i = 0; i < SigmaSize; ++i) {
            trans[x][i] = -1;
        }
        link[x] = -1;
        mx[x] = mi[x] = 0;
        return x;
    }
    void init() {
        tot = 0;
        lst = newState();
    }
    int getId(T x) {
        return (int)(x - 'a');
        // return (int)(x - 1);
    }
    void add(int ch) {
        int p = lst;
        int np = lst = newState();
        mx[np] = mx[p] + 1;
        while(p != -1 && trans[p][ch] == -1) {
            trans[p][ch] = np;
            p = link[p];
        }
        if(p == -1) {
            mi[np] = 1;
            link[np] = 0;
            return ;
        }
        int q = trans[p][ch];
        if(mx[p] + 1 == mx[q]){
            mi[np] = mx[q] + 1;
            link[np] = q;
            return ;
        }
        int nq = newState();
        mx[nq] = mx[p] + 1;
        link[nq] = link[q];
        for(int i = 0; i < SigmaSize; ++i) {
            trans[nq][i] = trans[q][i];
        }
        mi[np] = mi[q] = mx[nq] + 1;
        link[np] = link[q] = nq;
        while(p != -1 && trans[p][ch] == q) {
            trans[p][ch] = nq;
            p = link[p];
        }
        mi[nq] = mx[link[nq]] + 1;
    }
    int tax[maxn], topo[MaxState]; // Topology order: topo[0, tot - 1]
    void getTopo() {
        for(int i = 0; i <= len; ++i) tax[i] = 0;
        for(int i = 0; i < tot; ++i) ++tax[mx[i]];
        for(int i = 1; i <= len; ++i) tax[i] += tax[i - 1];
        for(int i = 0; i < tot; ++i) topo[--tax[mx[i]]] = i;
    }
    void show() {
        for(int i = 0; i < tot; ++i) {
            printf("State %d: \n", i);
            printf("transition:");
            for(int j = 0; j < SigmaSize; ++j) {
                printf(" %d", trans[i][j]);
            }
            puts("");
            printf("link: %d, min length: %d, max length: %d\n", link[i], mi[i], mx[i]);
            puts("+++++++++++++++++++++");
        }
        printf("Topology order:");
        for(int i = 0; i < tot; ++i) {
            printf(" %d", topo[i]);
        }
        puts("");
    }
public:
    SAM() {
        tot = 0;
    }
    int id[maxn]; // always need
    void solve(int n, T a[]) {
        len = n;
        init();
        for(int i = 0; i < len; ++i) {
            id[i] = getId(a[i]);
            add(id[i]);
        }
        // 此行以下根据题目要求进行修改
        getTopo();
        // show();

    }
};
\end{lstlisting}

调用方式

\begin{lstlisting}[language=C++]
SAM<char> sam;
char a[maxn];
int main() {
    scanf("%s", a);
    int n = strlen(a);
    sam.solve(n, a);
    return 0;
}
\end{lstlisting}

\subsubsection{字符串本质不同的非空子串个数}

\begin{lstlisting}[language=C++]
void solve(int n, T a[]) {
    len = n;
    init();
    for(int i = 0; i < len; ++i) {
        id[i] = getId(a[i]);
        add(id[i]);
    }
    ll ans = 0;
    for(int i = 1; i < tot; ++i) {
        ans += (ll)(mx[i] - mi[i] + 1);
    }
    printf("%lld\n", ans);
}
\end{lstlisting}

\subsubsection{字符串的所有长度为 K 的子串出现次数最多的子串的出现次数}

\begin{lstlisting}[language=C++]
int cnt[MaxState], num[maxn], ans[maxn];
void solve(int n, T a[]) {
    len = n;
    init();
    for(int i = 0; i < len; ++i) {
        id[i] = getId(a[i]);
        add(id[i]);
    }
    // 此行以下根据题目要求进行修改
    getTopo();
    // show();
    for(int i = 0; i < tot; ++i) cnt[i] = 0;
    int p = 0;
    for(int i = 0; i < len; ++i) {
        p = trans[p][id[i]];
        ++cnt[p];
    }
    for(int i = tot - 1; i >= 0; --i) {
        int x = topo[i];
        if(link[x] != -1) {
            cnt[link[x]] += cnt[x];
        }
    }
    for(int i = 0; i <= len; ++i) {
        num[i] = 0;
    }
    for(int i = 0; i < tot; ++i) {
        num[mx[i]] = max(num[mx[i]], cnt[i]);
    }
    int val = 0;
    for(int i = len; i >= 1; --i) {
        val = max(val, num[i]);
        ans[i] = val;
    }
    for(int i = 1; i <= len; ++i) {
        printf("%d\n", ans[i]);
    }
}
\end{lstlisting}

\subsubsection{最长不重叠子串}

\begin{lstlisting}[language=C++]
int lf[MaxState], rt[MaxState]; // 该状态(的结尾位置)出现的最左位置与最右位置 max{endpos}, min{endpos}
void solve(int n, T a[]) {
    len = n;
    init();
    for(int i = 0; i < len; ++i) {
        id[i] = getId(a[i]);
        add(id[i]);
    }
    // 此行以下根据题目要求进行修改
    getTopo();
    // show();
    for(int i = 0; i < tot; ++i) {
        lf[i] = inf;
        rt[i] = -inf;
    }
    int p = 0;
    for(int i = 0; i < len; ++i) {
        p = trans[p][id[i]];
        lf[p] = min(lf[p], i);
        rt[p] = max(rt[p], i);
    }
    for(int i = tot - 1; i >= 0; --i) {
        int x = topo[i];
        int lx = link[x];
        if(lx != -1) {
            lf[lx] = min(lf[lx], lf[x]);
            rt[lx] = max(rt[lx], rt[x]);
        }
    }
    int ans = 0;
    for(int i = 1; i < tot; ++i) {
        // printf("%d: %d [%d, %d]\n", i, mx[i], lf[i], rt[i]);
        int now = min(mx[i], rt[i] - lf[i]);
        ans = max(ans, now);
    }
    printf("%d\n", ans);
}
\end{lstlisting}

\subsubsection{字典序第 K 小子串}

对于一个给定长度为 N 的字符串,求它的第 K 小子串。

T 为 0 则表示不同位置的相同子串算作一个。T 为 1 则表示不同位置的相同子串算作多个。

\begin{lstlisting}[language=C++]
ll cnt[MaxState], sum[MaxState];
void dfs(int p, int k) {
    k -= cnt[p];
    if(k <= 0) {
        return ;
    }
    for(int i = 0; i < SigmaSize; ++i) {
        int to = trans[p][i];
        if(to == -1) {
            continue;
        }
        if(sum[to] >= k) {
            printf("%c", 'a' + i);
            dfs(to, k);
            return ;
        }
        k -= sum[to];
    }
}
void solve(int n, T a[]) {
    len = n;
    init();
    for(int i = 0; i < len; ++i) {
        id[i] = getId(a[i]);
        add(id[i]);
    }
    // 此行以下根据题目要求进行修改
    getTopo();
    // show();
    int t, k; scanf("%d%d", &t, &k);
    for(int i = 0; i < tot; ++i) {
        cnt[i] = 0;
    }
    if(t == 0) {
        for(int i = 0; i < tot; ++i) {
            cnt[i] = 1;
        }
    } else {
        int p = 0;
        for(int i = 0; i < len; ++i) {
            p = trans[p][id[i]];
            ++cnt[p];
        }
        for(int i = tot - 1; i > 0; --i) {
            int x = topo[i];
            int lx = link[x];
            cnt[lx] += cnt[x];
        }
    }
    cnt[0] = 0;
    for(int i = 0; i < tot; ++i) {
        sum[i] = cnt[i];
    }
    for(int i = tot - 1; i >= 0; --i) {
        int x = topo[i];
        for(int j = 0; j < SigmaSize; ++j) {
            int to = trans[x][j];
            if(to != -1) {
                sum[x] += sum[to];
            }
        }
    }
    if(sum[0] < k) {
        printf("-1");
    } else {
        dfs(0, k);
    }
    puts("");
}
\end{lstlisting}

\subsubsection{多个字符串的最长公共子串}

以第一个字符串建 SAM,其余字符串在 SAM 上进行转移并维护相关信息

\begin{lstlisting}[language=C++]
char t[maxn];
int ans[MaxState], num[MaxState];
void solve(int n, T a[]) {
    len = n;
    init();
    for(int i = 0; i < len; ++i) {
        id[i] = getId(a[i]);
        add(id[i]);
    }
    // 此行以下根据题目要求进行修改
    getTopo();
    // show();
    for(int i = 0; i < tot; ++i) {
        ans[i] = mx[i];
    }
    while(scanf("%s", t) != EOF) {
        // if(t[0] == 'Q') break;
        int lt = strlen(t);
        for(int i = 0; i < tot; ++i) {
            num[i] = 0;
        }
        int p = 0, now = 0;
        for(int i = 0; i < lt; ++i) {
            int cid = getId(t[i]);
            if(trans[p][cid] == -1) {
                while(p != -1 && trans[p][cid] == -1) p = link[p];
                if(p == -1) {
                    p = 0;
                    now = 0;
                } else {
                    now = mx[p] + 1;
                    p = trans[p][cid];
                }
            } else {
                ++now;
                p = trans[p][cid];
            }
            num[p] = max(num[p], now);
        }
        for(int i = tot - 1; i >= 1; --i) {
            int x = topo[i];
            int to = link[x];
            num[to] = max(num[to], num[x]);
            if(num[to] > mx[to]) num[to] = mx[to];
        }
        for(int i = 1; i < tot; ++i) ans[i] = min(ans[i], num[i]);
    }
    int lcs = 0;
    for(int i = 1; i < tot; ++i) lcs = max(lcs, ans[i]);
    printf("%d\n", lcs);
}    
\end{lstlisting}

\subsection{Lyndon 分解}

Lyndon 串:对于字符串 $s$,如果 $s$ 的字典序严格小于 $s$ 的所有后缀的字典序,称  是简单串,或者 Lyndon 串 。举一些例子, a , b , ab , aab , abb , ababb , abcd 都是 Lyndon 串。当且仅当  的字典序严格小于它的所有非平凡的循环同构串时,  才是 Lyndon 串。

Lyndon 分解:串 $s$ 的 Lyndon 分解记为 $s = w_1 w_2 \dots w_k$ ,其中所有 $w_i$ 为简单串,并且他们的字典序按照非严格单减排序,即 $w_1 \ge w_2 \ge \dots \ge w_k$ 。这样的分解存在且唯一。

\begin{lstlisting}[language=C++]
vector<string> duval(string const& s) {
    int n = s.size(), i = 0;
    vector<string> factorization;
    while (i < n) {
        int j = i + 1, k = i;
        while (j < n && s[k] <= s[j]) {
            if (s[k] < s[j]) k = i;
            else k++;
            j++;
        }
        while (i <= k) {
            factorization.push_back(s.substr(i, j - k));
            i += j - k;
        }
    }
    return factorization;
}
\end{lstlisting}

\subsection{回文自动机}

\begin{lstlisting}[language=C++]
struct PAM {
    static const int N = 1e5 + 5;
    int sz, tot, last;
    struct Node {
        int ch[26], len, fail;
        int cnt, dep, dif, slink;
        Node(int l = 0): len(l) {}
    } node[N];
    char s[N];
    int newNode(int l) {
        node[sz] = Node(l);
        return sz++;
    }
    void clear() {
        sz = -1, last = 0, tot = 0;
        s[0] = '$';
        newNode(0);
        newNode(-1);
        node[0].fail = 1;
    }
    int getfail(int x) {
        while(s[tot - node[x].len - 1] != s[tot]) x = node[x].fail;
        return x;
    }
    void insert(char c) {
        s[++tot] = c;
        int code = c - 'a';
        Node& now = node[getfail(last)];
        if (!now.ch[code]) {
            int itx = newNode(now.len + 2);
            Node &x = node[itx];
            x.fail = node[getfail(now.fail)].ch[code];
            Node &fx = node[x.fail];
            x.dep = fx.dep + 1;
            now.ch[code] = itx;
            x.dif = x.len - fx.len;
            if (x.dif == fx.dif)
                x.slink = fx.slink;
            else
                x.slink = x.fail;
        }
        last = now.ch[code];
        node[last].cnt++;
    }
};    
\end{lstlisting}