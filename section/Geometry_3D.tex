

\section{立体几何}

\subsection{jls 的 3D 几何板子(待整理)}

\begin{lstlisting}[language=C++]
#define mp make_pair
#define fi first
#define se second
#define pb push_back
typedef double db;
const db eps = 1e-6;
const db pi = acos(-1);
int sgn(double x) {
    if(fabs(x) < eps) return 0;
    if(x < 0) return -1;
    return 1;
}
int cmp(double x, double y) {
    return sgn(x - y);
}
struct point{
    db x,y;
    point operator + (const point &k1) const{return (point){k1.x+x,k1.y+y};}
    point operator - (const point &k1) const{return (point){x-k1.x,y-k1.y};}
    point operator * (db k1) const{return (point){x*k1,y*k1};}
    point operator / (db k1) const{return (point){x/k1,y/k1};}
    int operator == (const point &k1) const{return cmp(x,k1.x)==0&&cmp(y,k1.y)==0;}
    // 逆时针旋转 
    point turn(db k1){return (point){x*cos(k1)-y*sin(k1),x*sin(k1)+y*cos(k1)};}
    point turn90(){return (point){-y,x};}
    bool operator < (const point k1) const{
        int a=cmp(x,k1.x);
        if (a==-1) return 1; else if (a==1) return 0; else return cmp(y,k1.y)==-1;
    }
    db abs(){return sqrt(x*x+y*y);}
    db abs2(){return x*x+y*y;}
    db dis(point k1){return ((*this)-k1).abs();}
    point unit(){db w=abs(); return (point){x/w,y/w};}
    void scan(){double k1,k2; scanf("%lf%lf",&k1,&k2); x=k1; y=k2;}
    void print(){printf("%.11lf %.11lf\n",x,y);}
    db getw(){return atan2(y,x);} 
    point getdel(){if (sgn(x)==-1||(sgn(x)==0&&sgn(y)==-1)) return (*this)*(-1); else return (*this);}
	int getP() const{return sgn(y)==1||(sgn(y)==0&&sgn(x)==-1);}
};
db cross(point k1,point k2){return k1.x*k2.y-k1.y*k2.x;}
db dot(point k1,point k2){return k1.x*k2.x+k1.y*k2.y;}
db rad(point k1,point k2){return atan2(cross(k1,k2),dot(k1,k2));}
struct P3 {
    db x,y,z;
    P3 operator + (P3 k1){return (P3){x+k1.x,y+k1.y,z+k1.z};}
    P3 operator - (P3 k1){return (P3){x-k1.x,y-k1.y,z-k1.z};}
    P3 operator * (db k1){return (P3){x*k1,y*k1,z*k1};}
    P3 operator / (db k1){return (P3){x/k1,y/k1,z/k1};}
    db abs2(){return x*x+y*y+z*z;}
    db abs(){return sqrt(x*x+y*y+z*z);}
    P3 unit(){return (*this)/abs();}
    int operator < (const P3 k1) const{
        if (cmp(x,k1.x)!=0) return x<k1.x;
        if (cmp(y,k1.y)!=0) return y<k1.y;
        return cmp(z,k1.z)==-1;
    }
    int operator == (const P3 k1){
        return cmp(x,k1.x)==0&&cmp(y,k1.y)==0&&cmp(z,k1.z)==0;
    }
    void scan(){
        double k1,k2,k3; scanf("%lf%lf%lf",&k1,&k2,&k3);
        x=k1; y=k2; z=k3;
    }
};

P3 cross(P3 k1,P3 k2){return (P3){k1.y*k2.z-k1.z*k2.y,k1.z*k2.x-k1.x*k2.z,k1.x*k2.y-k1.y*k2.x};}
db dot(P3 k1,P3 k2){return k1.x*k2.x+k1.y*k2.y+k1.z*k2.z;}
//p=(3,4,5),l=(13,19,21),theta=85 ans=(2.83,4.62,1.77)
P3 turn3D(db k1,P3 l,P3 p){
    l=l.unit(); P3 ans; db c=cos(k1),s=sin(k1);
    ans.x=p.x*(l.x*l.x*(1-c)+c)+p.y*(l.x*l.y*(1-c)-l.z*s)+p.z*(l.x*l.z*(1-c)+l.y*s);
    ans.y=p.x*(l.x*l.y*(1-c)+l.z*s)+p.y*(l.y*l.y*(1-c)+c)+p.z*(l.y*l.z*(1-c)-l.x*s);
    ans.z=p.x*(l.x*l.z*(1-c)-l.y*s)+p.y*(l.y*l.z*(1-c)+l.x*s)+p.z*(l.x*l.x*(1-c)+c);
    return ans;
}
typedef vector<P3> VP;
typedef vector<VP> VVP;
db Acos(db x){return acos(max(-(db)1,min(x,(db)1)));}
// 球面距离 , 圆心原点 , 半径 1
db Odist(P3 a,P3 b){db r=Acos(dot(a,b)); return r;}
db r; P3 rnd;
vector<db> solve(db a,db b,db c){
    db r=sqrt(a*a+b*b),th=atan2(b,a);
    if (cmp(c,-r)==-1) return {0};
    else if (cmp(r,c)<=0) return {1};
    else {
        db tr=pi-Acos(c/r); return {th+pi-tr,th+pi+tr};
    }
}
vector<db> jiao(P3 a,P3 b){
    // dot(rd+x*cos(t)+y*sin(t),b) >= cos(r)
    if (cmp(Odist(a,b),2*r)>0) return {0};
    P3 rd=a*cos(r),z=a.unit(),y=cross(z,rnd).unit(),x=cross(y,z).unit();
    vector<db> ret = solve(-(dot(x,b)*sin(r)),-(dot(y,b)*sin(r)),-(cos(r)-dot(rd,b))); 
    return ret;
}
db norm(db x,db l=0,db r=2*pi){ // change x into [l,r)
    while (cmp(x,l)==-1) x+=(r-l); while (cmp(x,r)>=0) x-=(r-l);
    return x;
}
db disLP(P3 k1,P3 k2,P3 q){
    return (cross(k2-k1,q-k1)).abs()/(k2-k1).abs();
}
db disLL(P3 k1,P3 k2,P3 k3,P3 k4){
    P3 dir=cross(k2-k1,k4-k3); if (sgn(dir.abs())==0) return disLP(k1,k2,k3);
    return fabs(dot(dir.unit(),k1-k2));
}
VP getFL(P3 p,P3 dir,P3 k1,P3 k2){
    db a=dot(k2-p,dir),b=dot(k1-p,dir),d=a-b;
    if (sgn(fabs(d))==0) return {};
    return {(k1*a-k2*b)/d};
}
VP getFF(P3 p1,P3 dir1,P3 p2,P3 dir2){// 返回一条线
    P3 e=cross(dir1,dir2),v=cross(dir1,e);
    db d=dot(dir2,v); if (sgn(abs(d))==0) return {};
    P3 q=p1+v*dot(dir2,p2-p1)/d; return {q,q+e};
}
// 3D Covex Hull Template
db getV(P3 k1,P3 k2,P3 k3,P3 k4){ // get the Volume
    return dot(cross(k2-k1,k3-k1),k4-k1);
}
db rand_db(){return 1.0*rand()/RAND_MAX;}
vector<point> ConvexHull(vector<point>A, int flag=1){ // flag=0 不严格 flag=1 严格 
    int n=A.size(); vector<point>ans(n*2); 
    sort(A.begin(),A.end()); int now=-1;
    for (int i=0;i<A.size();i++){
        while (now>0&&sgn(cross(ans[now]-ans[now-1],A[i]-ans[now-1]))<flag) now--;
        ans[++now]=A[i];
    } int pre=now;
    for (int i=n-2;i>=0;i--){
        while (now>pre&&sgn(cross(ans[now]-ans[now-1],A[i]-ans[now-1]))<flag) now--;
        ans[++now]=A[i];
    } ans.resize(now); return ans;
}
VP convexHull2D(VP A,P3 dir){
    P3 x={(db)rand(),(db)rand(),(db)rand()}; x=x.unit();
    x=cross(x,dir).unit(); P3 y=cross(x,dir).unit();
    P3 vec=dir.unit()*dot(A[0],dir);
    vector<point>B;
    for (int i=0;i<A.size();i++) B.push_back((point){dot(A[i],x),dot(A[i],y)});
    B=ConvexHull(B); A.clear();
    for (int i=0;i<B.size();i++) A.push_back(x*B[i].x+y*B[i].y+vec);
    return A;
}
namespace CH3{
    VVP ret; set<pair<int,int> >e;
    int n; VP p,q;
    void wrap(int a,int b){
        if (e.find({a,b})==e.end()){
            int c=-1;
            for (int i=0;i<n;i++) if (i!=a&&i!=b){
                if (c==-1||sgn(getV(q[c],q[a],q[b],q[i]))>0) c=i;
            }
            if (c!=-1){
                ret.push_back({p[a],p[b],p[c]});
                e.insert({a,b}); e.insert({b,c}); e.insert({c,a});
                wrap(c,b); wrap(a,c);
            }
        }
    }
    VVP ConvexHull3D(VP _p){
        p=q=_p; n=p.size();
        ret.clear(); e.clear();
        for (auto &i:q) i=i+(P3){rand_db()*1e-4,rand_db()*1e-4,rand_db()*1e-4};
        for (int i=1;i<n;i++) if (q[i].x<q[0].x) swap(p[0],p[i]),swap(q[0],q[i]);
        for (int i=2;i<n;i++) if ((q[i].x-q[0].x)*(q[1].y-q[0].y)>(q[i].y-q[0].y)*(q[1].x-q[0].x)) swap(q[1],q[i]),swap(p[1],p[i]);
        wrap(0,1);
        return ret;
    }
}
VVP reduceCH(VVP A){
    VVP ret; map<P3,VP> M;
    for (VP nowF:A){
        P3 dir=cross(nowF[1]-nowF[0],nowF[2]-nowF[0]).unit();
        for (P3 k1:nowF) M[dir].pb(k1);
    }
    for (pair<P3,VP> nowF:M) ret.pb(convexHull2D(nowF.se,nowF.fi));
    return ret;
}
//  把一个面变成 ( 点 , 法向量 ) 的形式
pair<P3,P3> getF(VP F){
    return mp(F[0],cross(F[1]-F[0],F[2]-F[0]).unit());
}
// 3D Cut 保留 dot(dir,x-p)>=0 的部分
VVP ConvexCut3D(VVP A,P3 p,P3 dir){
    VVP ret; VP sec;
    for (VP nowF: A){
        int n=nowF.size(); VP ans; int dif=0;
        for (int i=0;i<n;i++){
            int d1=sgn(dot(dir,nowF[i]-p));
            int d2=sgn(dot(dir,nowF[(i+1)%n]-p));
            if (d1>=0) ans.pb(nowF[i]);
            if (d1*d2<0){
                P3 q=getFL(p,dir,nowF[i],nowF[(i+1)%n])[0];
                ans.push_back(q); sec.push_back(q);
            }
            if (d1==0) sec.push_back(nowF[i]); else dif=1;
            dif|=(sgn(dot(dir,cross(nowF[(i+1)%n]-nowF[i],nowF[(i+1)%n]-nowF[i])))==-1);
        }
        if (ans.size()>0&&dif) ret.push_back(ans);
    }
    if (sec.size()>0) ret.push_back(convexHull2D(sec,dir));
    return ret;
}
db vol(VVP A){
    if (A.size()==0) return 0; P3 p=A[0][0]; db ans=0;
    for (VP nowF:A)
        for (int i=2;i<nowF.size();i++)
            ans+=abs(getV(p,nowF[0],nowF[i-1],nowF[i]));
    return ans/6;
}
VVP init(db INF) {
    VVP pss(6,VP(4));
    pss[0][0] = pss[1][0] = pss[2][0] = {-INF, -INF, -INF};
    pss[0][3] = pss[1][1] = pss[5][2] = {-INF, -INF, INF};
    pss[0][1] = pss[2][3] = pss[4][2] = {-INF, INF, -INF};
    pss[0][2] = pss[5][3] = pss[4][1] = {-INF, INF, INF};
    pss[1][3] = pss[2][1] = pss[3][2] = {INF, -INF, -INF};
    pss[1][2] = pss[5][1] = pss[3][3] = {INF, -INF, INF};
    pss[2][2] = pss[4][3] = pss[3][1] = {INF, INF, -INF};
    pss[5][0] = pss[4][0] = pss[3][0] = {INF, INF, INF};
    return pss;
}
\end{lstlisting}
