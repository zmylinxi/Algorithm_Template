

\section{几何基础}

\subsection{点和向量}

\begin{lstlisting}[language=C++]
struct Point{
    double x, y;
    Point(double x = 0, double y = 0):x(x),y(y){}
};
typedef Point Vector;
Vector operator + (Vector A, Vector B){
    return Vector(A.x+B.x, A.y+B.y);
}
Vector operator - (Point A, Point B){
    return Vector(A.x-B.x, A.y-B.y);
}
Vector operator * (Vector A, double p){
    return Vector(A.x*p, A.y*p);
}
Vector operator / (Vector A, double p){
    return Vector(A.x/p, A.y/p);
}
bool operator < (const Point& a, const Point& b){
    if(a.x == b.x)
        return a.y < b.y;
    return a.x < b.x;
}
const double eps = 1e-6;
int sgn(double x){
    if(fabs(x) < eps) return 0;
    if(x < 0) return -1;
    return 1;
}
bool operator == (const Point& a, const Point& b){
    if(sgn(a.x - b.x) == 0 && sgn(a.y - b.y) == 0) return true;
    return false;
}

double Dot(Vector A, Vector B){
    return A.x*B.x + A.y*B.y;
}
double Length(Vector A){
    return sqrt(Dot(A, A));
}
double Angle(Vector A, Vector B){
    return acos(Dot(A, B) / Length(A) / Length(B));
}
double Cross(Vector A, Vector B){
    return A.x*B.y-A.y*B.x;
}
double Area2(Point A, Point B, Point C){
    return Cross(B-A, C-A);
}
Vector Rotate(Vector A, double rad){//rad为弧度 且为逆时针旋转的角
    return Vector(A.x*cos(rad)-A.y*sin(rad), A.x*sin(rad)+A.y*cos(rad));
}
Vector Normal(Vector A){//向量A左转90°的单位法向量
    double L = Length(A);
    return Vector(-A.y/L, A.x/L);
}
bool ToLeftTest(Point a, Point b, Point c){
    return Cross(b - a, c - b) > 0;
}
\end{lstlisting}

\subsection{复数类实现平面向量}

\begin{lstlisting}[language=C++]
#include <complex>
using namespace std;
typedef complex<double> Point;
// 复数定义向量后,自动拥有构造函数、加减法和数量积
typedef Point Vector; 
const double eps = 1e-9;
int sgn(double x){
    if(fabs(x) < eps)
        return 0;
    if(x < 0)
        return -1;
    return 1;
}
double Length(Vector A){
    return abs(A);
}
// conj(a + bi) 返回共轭复数 a - bi
double Dot(Vector A, Vector B){
    return real(conj(A)*B);
}
double Cross(Vector A, Vector B){
    return imag(conj(A)*B);
}
// exp(p) 返回以 e 为底复数的指数
Vector Rotate(Vector A, double rad){
    return A*exp(Point(0, rad));
}
\end{lstlisting}

\subsection{点和直线}

\subsubsection{直线定义}

\begin{lstlisting}[language=C++]
struct Line{
    Point v, p;
    Line(Point v, Point p):v(v), p(p) {}
    Point point(double t){ // 返回点 P = v + (p - v)*t
        return v + (p - v)*t;
    }
};
\end{lstlisting}

\subsubsection{计算两直线交点}

\begin{lstlisting}[language=C++]
// 调用前需保证 Cross(v, w) != 0
Point GetLineIntersection(Point P, Vector v, Point Q, Vector w){
    Vector u = P-Q;
    double t = Cross(w, u)/Cross(v, w);
    return P + v*t;
}
\end{lstlisting}

\subsubsection{点 P 到直线 AB 距离公式}

\begin{lstlisting}[language=C++]
// 不取绝对值,得到的是有向距离
double DistanceToLine(Point P, Point A, Point B) {
    Vector v1 = B - A, v2 = P - A;
    return fabs(Cross(v1, v2)/Length(v1));
}
\end{lstlisting}

\subsubsection{点 P 到线段 AB 的距离公式}

\begin{lstlisting}[language=C++]
double DistanceToSegment(Point P, Point A, Point B) {
    if(A == B) return Length(P - A);
    Vector v1 = B - A, v2 = P - A, v3 = P - B;
    if(sgn(Dot(v1, v2)) < 0) return Length(v2);
    if(sgn(Dot(v1, v3)) > 0) return Length(v3);
    return DistanceToLine(P, A, B);
}
\end{lstlisting}

\subsubsection{点 P 在直线 AB 上的投影点}

\begin{lstlisting}[language=C++]
Point GetLineProjection(Point P, Point A, Point B){
    Vector v = B-A;
    return A + v*(Dot(v, P-A)/Dot(v, v));
}
\end{lstlisting}

\subsubsection{判断 p 点是否在线段 a1a2 上}

\begin{lstlisting}[language=C++]
bool SegmentProperIntersection(Point a1, Point a2, Point b1, Point b2){
    double c1 = Cross(a2-a1, b1-a1), c2 = Cross(a2-a1, b2-a1);
    double c3 = Cross(b2-b1, a1-b1), c4 = Cross(b2-b1, a2-b1);
    // if 判断控制是否允许线段在端点处相交,根据需要添加
    if(!sgn(c1) || !sgn(c2) || !sgn(c3) || !sgn(c4)){
        bool f1 = OnSegment(b1, a1, a2);
        bool f2 = OnSegment(b2, a1, a2);
        bool f3 = OnSegment(a1, b1, b2);
        bool f4 = OnSegment(a2, b1, b2);
        bool f = (f1 | f2 | f3 | f4);
        return f;
    }
    return (sgn(c1)*sgn(c2) < 0 && sgn(c3)*sgn(c4) < 0);
}
\end{lstlisting}

\subsection{求三角形的外心、垂心、重心、费马点}

\begin{lstlisting}[language=C++]
struct Point {
    double x, y;
};
struct Line{
    Point a, b;
};
Point Intersection(Line u, Line v){
    Point ret = u.a;
    double t1 = (u.a.x - v.a.x)*(v.a.y - v.b.y) - (u.a.y - v.a.y)*(v.a.x - v.b.x);
    double t2 = (u.a.x - u.b.x)*(v.a.y - v.b.y) - (u.a.y - u.b.y)*(v.a.x - v.b.x);
    double t = t1/t2;
    ret.x += (u.b.x - u.a.x)*t;
    ret.y += (u.b.y - u.a.y)*t;
    return ret;
}
// 外心
Point Circumcenter(Point a, Point b, Point c){
    Line u, v;
    u.a.x = (a.x + b.x)/2;
    u.a.y = (a.y + b.y)/2;
    u.b.x = u.a.x - a.y + b.y;
    u.b.y = u.a.y + a.x - b.x;
    v.a.x = (a.x + c.x)/2;
    v.a.y = (a.y + c.y)/2;
    v.b.x = v.a.x - a.y + c.y;
    v.b.y = v.a.y + a.x - c.x;
    return Intersection(u, v);
}
// 垂心
Point Perpencenter(Point a, Point b, Point c){
    Line u, v;
    u.a = c;
    u.b.x = u.a.x - a.y + b.y;
    u.b.y = u.a.y + a.x - b.x;
    v.a = b;
    v.b.x = v.a.x - a.y + c.y;
    v.b.y = v.a.y + a.x - c.x;
    return Intersection(u, v);
}
// 三角形重心
// 到三角形三顶点距离的平方和最小的点
// 三角形内到三边距离之积最大的点
Point barycenter(Point a, Point b, Point c){
    Line u, v;
    u.a.x = (a.x + b.x)/2;
    u.a.y = (a.y + b.y)/2;
    u.b = c;
    v.a.x = (a.x + c.x)/2;
    v.a.y = (a.y + c.y)/2;
    v.b = b;
    return Intersection(u, v);
}
inline double Dist(Point p1, Point p2){
    return sqrt((p1.x - p2.x)*(p1.x - p2.x) + (p1.y - p2.y)*(p1.y - p2.y));
}
// 三角形费马点
// 到三角形三顶点距离之和最小的点
Point Ferment(Point a, Point b, Point c){
    Point u, v;
    double step = fabs(a.x) + fabs(a.y) + fabs(b.x) + fabs(b.y) + fabs(c.x) + fabs(c.y);
    u.x = (a.x + b.x + c.x)/3;
    u.y = (a.y + b.y + c.y)/3;
    while(step >1e-10){
        for(int k = 0; k < 10; step /= 2, k++){
            for(int i = -1; i <= 1; ++i){
                for(int j = -1; j <= 1; ++j){
                    v.x = u.x + step*i;
                    v.y = u.y + step*j;
                    double t1 = Dist(u, a) + Dist(u, b) + Dist(u, c);
                    double t2 = Dist(v, a) + Dist(v, b) + Dist(v, c);
                    if (t1 > t2) u = v;
                }
            }
        }
    }
    return u;
}
\end{lstlisting}

\subsection{多边形}

\subsubsection{多边形有向面积}

\begin{lstlisting}[language=C++]
// p 为端点集合,n 为端点个数
double PolygonArea(Point *p, int n){
    double s = 0;
    for(int i = 1; i < n-1; ++i)
        s += Cross(p[i]-p[0], p[i+1]-p[0]);
    return s;
}
\end{lstlisting}

\subsubsection{判断点是否在不自交多边形(可以为凹多边形)内}

\begin{lstlisting}[language=C++]
// 判断点是否在多边形内,若点在多边形内返回1,在多边形外部返回0,在多边形上返回-1
int isPointInPolygon(Point p, vector<Point> poly){
    int wn = 0;
    int n = poly.size();
    for(int i = 0; i < n; ++i){
        if(OnSegment(p, poly[i], poly[(i+1)%n])) return -1;
        int k = sgn(Cross(poly[(i+1)%n] - poly[i], p - poly[i]));
        int d1 = sgn(poly[i].y - p.y);
        int d2 = sgn(poly[(i+1)%n].y - p.y);
        if(k > 0 && d1 <= 0 && d2 > 0) wn++;
        if(k < 0 && d2 <= 0 && d1 > 0) wn--;
    }
    if(wn != 0) return 1;
    return 0;
}
\end{lstlisting}

\subsubsection{计算不自交多边形与直线相交长度和}

\begin{lstlisting}[language=C++]
const int maxn = 1e3 + 5;
struct Point{
    double x, y;
    Point(double x = 0, double y = 0):x(x),y(y){}
} p[maxn];
typedef Point Vector;
Vector operator - (Point A, Point B){
    return Vector(A.x-B.x, A.y-B.y);
}
const double eps = 1e-5;
int sgn(double x){
    if(fabs(x) < eps)
        return 0;
    if(x < 0)
        return -1;
    return 1;
}
double Dot(Vector A, Vector B){
    return A.x*B.x + A.y*B.y;
}
double Length(Vector A){
    return sqrt(Dot(A, A));
}
double Cross(Vector A, Vector B){
    return A.x*B.y-A.y*B.x;
}
struct node{
    double len;
    int cs;
    node(double len = 0, int cs = 0):len(len), cs(cs){}
    bool operator < (const node& nd) const{
        return len < nd.len;
    }
};
double cal(Point A, Point B, Point C, Point D){
    double ret = Cross(C - A, D - C)/Cross(B - A, D - C);
    return ret;
}
vector<node> v;
// n: size of polygon a, b: points on line
double solve(int n, Point A, Point B){
    double len = Length(A - B);
    v.clear();
    for(int i = 0; i < n; ++i){
        int a = sgn(Cross(p[i] - A, B - A));
        int b = sgn(Cross(p[(i + 1)%n] - A, B - A));
        if(a == b) continue;
        v.push_back(node(len*cal(A, B, p[i], p[(i + 1)%n]), a - b));
    }
    int sz = (int)v.size();
    sort(v.begin(), v.end());
    int wn = 0;
    double ret = 0;
    for(int i = 0; i < sz - 1; ++i){
        wn += v[i].cs;
        if(wn) ret += v[i + 1].len - v[i].len;
    }
    return ret;
}
int main(){
    int n, m; scanf("%d%d", &n, &m);
    for(int i = 0; i < n; ++i) scanf("%lf%lf", &p[i].x, &p[i].y);
    while(m--){
        Point A, B;
        scanf("%lf%lf%lf%lf", &A.x, &A.y, &B.x, &B.y);
        double ans = solve(n, A, B);
        printf("%.20f\n", ans);
    }
    return 0;
}
\end{lstlisting}

\subsection{圆}

\subsubsection{常用定义}

\begin{lstlisting}[language=C++]
struct Circle{
    Point c; double r;
    Circle(Point c, double r):c(c), r(r) {}
    Point point(double a){//通过圆心角求坐标
        return Point(c.x + cos(a)*r, c.y + sin(a)*r);
    }
};
\end{lstlisting}

\subsubsection{求圆与直线交点}

\begin{lstlisting}[language=C++]
// 求圆与直线交点
int getLineCircleIntersection(Line L, Circle C, double& t1, double& t2, vector<Point>& sol){
    double a = L.v.x, b = L.p.x - C.c.x, c = L.v.y, d = L.p.y - C.c.y;
    double e = a*a + c*c, f = 2*(a*b + c*d), g = b*b + d*d - C.r*C.r;
    double delta = f*f - 4*e*g; // 判别式
    if(sgn(delta) < 0) // 相离
        return 0;
    if(sgn(delta) == 0){ // 相切
        t1 = -f /(2*e);
        t2 = -f /(2*e);
        sol.push_back(L.point(t1));//sol存放交点本身
        return 1;
    }
    // 相交
    t1 = (-f - sqrt(delta))/(2*e);
    sol.push_back(L.point(t1));
    t2 = (-f + sqrt(delta))/(2*e);
    sol.push_back(L.point(t2));
    return 2;
}
\end{lstlisting}

\subsubsection{求两圆相交面积}

\begin{lstlisting}[language=C++]
double AreaOfOverlap(Point c1, double r1, Point c2, double r2){
    double d = Length(c1 - c2);
    if(r1 + r2 < d + eps)
        return 0.0;
    if(d < fabs(r1 - r2) + eps){
        double r = min(r1, r2);
        return pi*r*r;
    }
    double x = (d*d + r1*r1 - r2*r2)/(2.0*d);
    double p = (r1 + r2 + d)/2.0;
    double t1 = acos(x/r1);
    double t2 = acos((d - x)/r2);
    double s1 = r1*r1*t1;
    double s2 = r2*r2*t2;
    double s3 = 2*sqrt(p*(p - r1)*(p - r2)*(p - d));
    return s1 + s2 - s3;
}
\end{lstlisting}

\subsection{网格图}

\subsubsection{pick 定理}

Pick定理是指一个计算点阵中顶点在格点上的多边形面积公式该公式可以表示为

$$2S = 2a + b − 2$$

其中a表示多边形内部的点数,b表示多边形边界上的点数,S表示多边形的面积。

常用形式

$$S = a + 2b - 1$$

\subsubsection{多边形内部和边上网格点数量}

\begin{lstlisting}[language=C++]
struct Point{
    int x, y;
};
// 多边形上的网格点个数
int Onedge(int n, Point* p){
int re  t = 0;
    for(int i = 0; i < n; ++i)
        ret += __gcd(abs(p[i].x - p[(i + 1)%n].x), abs(p[i].y - p[(i + 1)%n].y));
    return ret;
}
// 多边形内的网格点个数
int Inside(int n, Point* p){
    int ret = 0;
    for(int i = 0; i < n; ++i)
        ret += p[(i + 1)%n].y*(p[i].x - p[(i + 2)%n].x);
    ret = (abs(ret) - Onedge(n, p))/2 + 1;
    return ret;
}
\end{lstlisting}
