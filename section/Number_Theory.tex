

\section{数论}
\subsection{素数筛}
\subsubsection{线性筛}

\begin{lstlisting}[language=C++]
const int maxn = 1e7 + 5;
const int maxp = 1e6 + 5;
// 1e7 以内的素数一共有 664579 (不到 7e5 )个
int p[maxp], tot;
bool np[maxn];
void getPrime(int n) {
    tot = 0;
    for(int i = 0; i <= n; ++i) np[i] = false;
    for(int i = 2; i <= n; ++i) {
        if(!np[i]) p[tot++] = i;
        for(int j = 0; j < tot; ++j) {
            int x = i*p[j];
            if(x > n) break;
            np[x] = true;
            if(i%p[j] == 0) break;
        }
    }
}
\end{lstlisting}

\subsubsection{小素数 x 在 log x 时间内分解}

需对值域内所有的小素数线性预处理

\begin{lstlisting}[language=C++]
const int maxn = 1e7 + 5;
const int maxp = 1e6 + 5;
int p[maxp], mf[maxn], tot;
bool np[maxn];
void getPrime(int n) {
    tot = 0;
    for(int i = 0; i <= n; ++i) np[i] = false;
    for(int i = 2; i <= n; ++i) {
        if(!np[i]) {
            p[tot++] = i;
            mf[i] = i;
        }
        for(int j = 0; j < tot; ++j) {
            int x = i*p[j];
            if(x > n) break;
            np[x] = true; mf[x] = p[j];
            if(i%p[j] == 0) break;
        }
    }
}
int fac[35], num;
void getFactor(int x) {
    num = 0;
    while(x > 1) {
        fac[num++] = mf[x];
        x /= mf[x];
    }
}
\end{lstlisting}

\subsubsection{区间筛 ($O(nloglogn)$)}


\begin{lstlisting}[language=C++]
const int maxn = 1e6 + 5;
bool isP[maxn], isPS[maxn];
ll p[maxn];
int tot;
// O(nloglogn) 筛出在区间 [a, b) 中的所有素数
void SegmentSieve(ll a, ll b){
    tot = 0;
    for(ll i = 0; i*i < b; ++i) isPS[i] = true;
    for(int i = 0; i < b - a; ++i) isP[i] = true;
    for(ll i = 2; i*i < b; ++i){
        if(!isPS[i]) continue;
        for(ll j = 2*i; j*j < b; j += i) isPS[j] = false;
        for(ll j = max(2LL, (a + i - 1)/i)*i; j < b; j += i) isP[j - a] = false;
    }
    for(ll i = 0; i < b - a; ++i){
        if(!isP[i]) continue;
        ll num = a + i;
        if(num > 1) p[tot++] = a + i;
    }
}
\end{lstlisting}




\subsubsection{Meisell-Lehmer ($O(n ^ {\frac{2}{3}})$ 求 $[1,  n]$ 素数的个数)}

\begin{lstlisting}[language=C++]
namespace Solver {
    const int N = 5e6 + 2;
    bool np[N];
    int prime[N], pi[N];
    int getprime() {
        int cnt = 0;
        np[0] = np[1] = 1;
        pi[0] = pi[1] = 0;
        for(int i = 2; i < N; ++i) {
            if(!np[i]) prime[++cnt] = i;
            pi[i] = cnt;
            for(int j = 1; j <= cnt && i * prime[j] < N; ++j) {
                np[i * prime[j]] = 1;
                if(i % prime[j] == 0)   break;
            }
        }
        return cnt;
    }
    const int M = 7;
    const int PM = 2 * 3 * 5 * 7 * 11 * 13 * 17;
    int phi[PM + 1][M + 1], sz[M + 1];
    void init() {
        getprime();
        sz[0] = 1;
        for(int i = 0; i <= PM; ++i)  phi[i][0] = i;
        for(int i = 1; i <= M; ++i) {
            sz[i] = prime[i] * sz[i - 1];
            for(int j = 1; j <= PM; ++j) phi[j][i] = phi[j][i - 1] - phi[j / prime[i]][i - 1];
        }
    }
    int sqrt2(ll x) {
        ll r = (ll)sqrt(x - 0.1);
        while(r * r <= x)   ++r;
        return int(r - 1);
    }
    int sqrt3(ll x) {
        ll r = (ll)cbrt(x - 0.1);
        while(r * r * r <= x)   ++r;
        return int(r - 1);
    }
    ll getphi(ll x, int s) {
        if(s == 0)  return x;
        if(s <= M)  return phi[x % sz[s]][s] + (x / sz[s]) * phi[sz[s]][s];
        if(x <= prime[s]*prime[s])   return pi[x] - s + 1;
        if(x <= prime[s]*prime[s]*prime[s] && x < N) {
            int s2x = pi[sqrt2(x)];
            ll ans = pi[x] - (s2x + s - 2) * (s2x - s + 1) / 2;
            for(int i = s + 1; i <= s2x; ++i) ans += pi[x / prime[i]];
            return ans;
        }
        return getphi(x, s - 1) - getphi(x / prime[s], s - 1);
    }
    ll getpi(ll x) {
        if(x < N) return pi[x];
        ll ans = getphi(x, pi[sqrt3(x)]) + pi[sqrt3(x)] - 1;
        for(int i = pi[sqrt3(x)] + 1, ed = pi[sqrt2(x)]; i <= ed; ++i) ans -= getpi(x / prime[i]) - i + 1;
        return ans;
    }
    ll lehmer_pi(ll x) { // get ans
        if(x < N)   return pi[x];
        int a = (int)lehmer_pi(sqrt2(sqrt2(x)));
        int b = (int)lehmer_pi(sqrt2(x));
        int c = (int)lehmer_pi(sqrt3(x));
        ll sum = getphi(x, a) +(ll)(b + a - 2) * (b - a + 1) / 2;
        for (int i = a + 1; i <= b; i++) {
            ll w = x / prime[i];
            sum -= lehmer_pi(w);
            if (i > c) continue;
            ll lim = lehmer_pi(sqrt2(w));
            for (int j = i; j <= lim; j++) sum -= lehmer_pi(w / prime[j]) - (j - 1);
        }
        return sum;
    }
}
\end{lstlisting}

\subsection{归并排序求逆序对 $O(nlog(n))$ }



\begin{lstlisting}[language=C++]
// 下标,例如数组int a[5],全部排序的调用为 mgst(0, 4)
// a1为原数组,a2为临时数组
int a1[maxn], a2[maxn];
ll mg(int lf, int mid, int rt){
    int i = lf, j = mid + 1, k = lf;
    ll ret = 0;
    while(i <= mid && j <= rt){
        if(a1[i] <= a1[j]) {
            a2[k++] = a1[i++];
        } else{
            a2[k++] = a1[j++];
            ret += (ll)j - k;
        }
    }
    while(i <= mid) a2[k++] = a1[i++];
    while(j <= rt) a2[k++] = a1[j++];
    for(int i = lf; i <= rt; ++i) a1[i] = a2[i];
    return ret;
}

ll mgst(int lf, int rt){
    ll ret = 0;
    if(lf < rt){
        int mid = (lf + rt)/2;
        ret += mgst(lf, mid);
        ret += mgst(mid + 1, rt);
        ret += mg(lf, mid, rt);
    }
    return ret;
}        
\end{lstlisting}

\subsection{GCD及其应用}

\subsubsection{拓展GCD}

\begin{lstlisting}[language=C++]
//返回 d = gcd(a, b); 和对应于等式 ax + by = d 中的 x, y
ll ExGcd(ll a, ll b, ll &x, ll &y) {
    if(b == 0) {
        if(a == 0) return -1;
        x = 1; y = 0;
        return a;
    }
    ll d = ExGcd(b, a%b, y, x);
    y -= a/b*x;
    return d;
}
\end{lstlisting}

\subsubsection{EXGCD求逆元}

\begin{lstlisting}[language=C++]
// ax = 1(% mod)
ll mod_reverse(ll a, ll mod){ // 不要求mod为素数
    ll x, y;
    ll d = extend_gcd(a, mod, x, y);
    if(d == 1) return (x%mod + mod)%mod;
    return -1LL;
}
\end{lstlisting}

\subsubsection{浮点数GCD}

\begin{lstlisting}[language=C++]
int sgn(double x){
    if(fabs(x) < eps) return 0;
    if(x > 0) return 1;
    return -1;
}
double fgcd(double a, double b){
    if(sgn(a) == 0) return b;
    if(sgn(b) == 0) return a;
    return fgcd(b, fmod(a, b));
}
\end{lstlisting}

\subsection{逆元}

\subsubsection{取模意义下快速幂 $O(log(n))$}

\begin{lstlisting}[language=C++]
ll pow_m(ll a, ll b, ll mod) { //取模意义下的 a 的 b 次方
    ll r = 1, bs = a;
    while(b) {
        if(b & 1)
            r = r*bs%mod;
        bs = bs*bs%mod;
        b >>= 1;
        b %= mod;
    }
    return r;
}
\end{lstlisting}

\subsubsection{费马小定理}

当模数为素数时,在模 $mod$ 意义下, $inv(a) = a^{-1} = a^{mod - 2}$

\subsubsection{线性递推阶乘逆元}

\begin{lstlisting}[language=C++]
const int maxn = 1e6 + 5;
const ll mod = 1e9 + 7;
ll fac[maxn], inv_fac[maxn];
ll pow_m(ll a, ll b, ll mod) { //取模意义下的 a 的 b 次方
    ll r = 1, bs = a;
    while(b) {
        if(b & 1)
            r = r*bs%mod;
        bs = bs*bs%mod;
        b >>= 1;
        b %= mod;
    }
    return r;
}
void getFacInv(int n) { // n 为最大所需阶乘数
    fac[0] = 1;
    for(int i = 1; i <= n; ++i)
        fac[i] = (fac[i-1]*i)%mod;
    inv_fac[n] = pow_m(fac[n], mod - 2, mod);
    for(int i = n-1; i >= 0; --i)
        inv_fac[i] = (inv_fac[i+1]*(i+1))%mod;
    return ;
}
\end{lstlisting}

\subsubsection{线性递推逆元}

\begin{lstlisting}[language=C++]
int inv[maxn];
void GetInv(int p) {
    inv[1] = 1;
    for(int i = 2; i < p; ++i) {
        inv[i] = (ll)(p - (p/i))*inv[p%i]%p;
        printf("inv[%d] = %d\n", i, inv[i]);
    }
}
\end{lstlisting}

\subsection{中国剩余定理 CRT}

\subsubsection{CRT(要求模数数组两两互质)}

\begin{lstlisting}[language=C++]
const int maxn = 1e5 + 5;
int a[maxn]; // 余数数组
int m[maxn]; // 模数数组

// n为数组长度
ll CRT(int n) {
    ll M = 1;
    ll ans = 0;
    for(int i = 0; i < n; ++i)
        M *= m[i];
    for(int i = 0; i < n; ++i) {
        ll Mi = M/m[i];
        ans = (ans+(((Mi*inv(Mi, m[i]))%M)*a[i]%M))%M;
    }
    while(ans < 0) ans += M;
    return ans;
}
\end{lstlisting}

\subsubsection{EXCRT(不要求模数数组两两互质)}

\begin{lstlisting}[language=C++]
/*
 *   hdu 1573
 *   求在小于等于 n 的正整数中有多少个 X 满足:
 *   X mod a[0] = b[0], X mod a[1] = b[1], ...... , X mod a[m - 1] = b[m - 1]
 *   (0 < n <= 1,000,000,000, 0 < m <= 10)
*/
#include <bits/stdc++.h>
using namespace std;
typedef long long ll;
const int maxn = 10 + 5;
ll a[maxn], b[maxn], n, m;
ll extgcd(ll a, ll b, ll &x, ll &y){
    ll d = a;
    if(b != 0){
        d = extgcd(b, a%b, y, x);
        y -= (a/b)*x;
    }
    else{
        x = 1;
        y = 0;
    }
    return d;
}
ll ExCRT(){
    ll a1 = a[0], b1 = b[0];
    for(int i = 1; i < m; i++){
        ll a2 = a[i], b2 = b[i];
        ll bb = b2 - b1, x, y;
        ll d = extgcd(a1, a2, x, y);
        if(bb%d) return 0;
        ll k = bb/d*x, t = a2/d;
        if(t < 0) t = -t;
        k = (k%t + t)%t;
        b1 = b1 + a1*k;
        a1 = a1/d*a2;
    }
    ll ret = b1;
    if (ret == 0) ret = a1;
    if(ret > n) return 0;
    ret = (n - ret)/a1 + 1;
    return ret;
}

int main(){
    int t; scanf("%d", &t);
    while(t--){
        scanf("%lld %lld", &n, &m);
        for(int i = 0; i < m; ++i) scanf("%lld", &a[i]);
        for(int i = 0; i < m; ++i) scanf("%lld", &b[i]);
        printf("%lld\n", ExCRT());
    }
    return 0;
}
\end{lstlisting}

\subsection{欧拉函数}

\subsubsection{线性筛}

\begin{lstlisting}[language=C++]
const int maxn = 1e6 + 5;
int phi[maxn], p[maxn], tot;
bool np[maxn];
void get_phi(int n) {
    tot = 0;
    phi[1] = 1;
    for (int i = 2; i <= n; ++i) {
        if (!np[i]) {
            p[tot++] = i;
            phi[i] = i - 1;
        }
        for (int j = 0; j < tot; ++j) {
            int x = i*p[j];
            if (x > n) break;
            np[x] = true;
            if (i%p[j] == 0) {
                phi[x] = phi[i]*p[j];
                break;
            } else {
                phi[x] = phi[i]*(p[j] - 1);
            }
        }
    }
}
\end{lstlisting}

\subsubsection{$O(\sqrt{x})$ 求单个欧拉函数值}

\begin{lstlisting}[language=C++]
ll get_phi(ll x) {
    ll ret = x, sx = sqrt(x) + 1;
    for (ll j = 2; j <= sx; ++j) {
        if (x%j == 0) {
            ret = ret/j*(j - 1);
            while (x%j == 0) x /= j;
        }
    }
    if (x > 1) ret = ret/x*(x - 1LL);
    return ret;
}
\end{lstlisting}

\subsection{莫比乌斯函数}


如果一个数包含平方因子,那么 miu(n) = 0。

例如:miu(4), miu(12), miu(18) = 0。

如果一个数不包含平方因子,并且有 k 个不同的质因子,那么 $miu(n) = (-1)^k$。

例如:miu(2), miu(3), miu(30) = -1,miu(1), miu(6), miu(10) = 1。

\subsubsection{线性筛}

\begin{lstlisting}[language=C++]
const int maxn = 1e7 + 5;
bool isNotPrime[maxn];
vector<int> prime;
int miu[maxn];
void getMiu(int n) {
    isNotPrime[1] = true;
    miu[1] = 1;
    for(int i = 2; i <= n; i++) {
        if(!isNotPrime[i]) {
            prime.push_back(i);
            miu[i] = -1;
        }
        for(int j = 0; j < prime.size() && i*prime[j] <= n; ++j){
            isNotPrime[i*prime[j]] = true;
            if(i%prime[j] == 0) {
                miu[i*prime[j]] = 0;
                break;
            }
            miu[i*prime[j]] = -miu[i];
        }
    }
}
\end{lstlisting}

\subsubsection{$O(\sqrt{x})$ 求单个莫比乌斯函数的值}

\begin{lstlisting}[language=C++]
int getMiu(ll n){
    int v = 1;
    for(ll i = 2; i*i <= n; ++i){
        if(n%i == 0){
            v = -v;
            n /= i;
            if(n%i == 0) return 0;
        }
    }
    if(n != 1) v = -v;
    return v;
}
\end{lstlisting}

\subsubsection{数论分块}

对于给出的 n 个询问,每次求有多少个数对 $(x, y)$ ,满足 $a \le x \le b$ , $c \le y \le d$ ,且 $gcd(x, y) = k$ , $gcd(x, y)$ 函数为 x 和 y 的最大公约数。

\begin{lstlisting}[language=C++]
int tot;
int miu[maxn], p[maxn], sum[maxn];
bool notP[maxn];
void getMiu() {
    miu[1] = 1; tot = 0;
    sum[0] = 0; sum[1] = 1;
    for(int i = 2; i < maxn; ++i) {
        if(!notP[i]) {
            p[tot++] = i;
            miu[i] = -1;
        }
        for(int j = 0; j < tot; ++j) {
            int num = i*p[j];
            if(num >= maxn) break;
            notP[num] = true;
            if(i%p[j] == 0) {
                miu[num] = 0;
                break;
            }
            miu[num] = -miu[i];
        }
        sum[i] = sum[i - 1] + miu[i];
    }
}
int solve(int a, int b, int d) {
    if(a > b) swap(a, b);
    a /= d; b /= d;
    int ret = 0;
    for(int lf = 1, rt; lf <= a; lf = rt + 1) {
        int qa = a/lf, qb = b/lf;
        int ra = a/qa, rb = b/qb;
        rt = min(ra, rb);
        ret += (sum[rt] - sum[lf - 1])*qa*qb;
    }
    return ret;
}
int main() {
    getMiu();
    int t; scanf("%d", &t);
    while(t--) {
        int a, b, c, d, k; scanf("%d%d%d%d%d", &a, &b, &c, &d, &k);
        int ans = solve(b, d, k) - solve(a - 1, d, k) - solve(b, c - 1, k) + solve(a - 1, c - 1, k);
        printf("%d\n", ans);
    }
    return 0;
}
\end{lstlisting}

\subsection{离散对数 EXBSGS 算法}

已知 $a, b, Pa, b, P$,a 与 P 不一定互质

求解同余方程 $ax ≡ b (mod P)$

\begin{lstlisting}[language=C++]
class Hash {
    static const int HashMod = 3000017;
    // 99991, 3000017, 7679977, 19260817
    int tp, st[HashMod + 5];
    bool vis[HashMod + 5];
    ll h[HashMod + 5];
    ll val[HashMod + 5];
    int locate(const ll& x) const {
        int p = x%HashMod;
        while(vis[p] && h[p] != x) {
            if(++p == HashMod) p = 0;
        }
        return p;
    }
public:
    Hash() {
        tp = 0;
        memset(vis, false, sizeof(false));
    }
    void ist(const ll& x, const ll& y) {
        const int p = locate(x);
//        assert(!vis[p] || h[p] == x);
        h[p] = x; val[p] = y;
        vis[p] = true; st[++tp] = p;
    }
    bool get(const ll& x, ll& y) const {
        const int p = locate(x);
        if(vis[p] && h[p] == x) {
            y = val[p];
            return true;
        }
        return false;
    }
    void clr() {
        while(tp) vis[st[tp--]] = false;
    }
};
Hash H;
ll ExGcd(ll a, ll b, ll &x, ll &y) {
    if(b == 0) {
        if(a == 0) return -1;
        x = 1; y = 0;
        return a;
    }
    ll d = ExGcd(b, a%b, y, x);
    y -= a/b*x;
    return d;
}
ll ExBSGS(ll A, ll B, ll C) {
    H.clr();
    ll ret = 1;
    for(ll i = 0; i <= 50; ++i) {
        if(ret == B) return i;
        ret = ret*A%C;
    }
    ll d, x, y, ans = 1, cnt = 0;
    while((d = ExGcd(A, C, x, y)) != 1) {
        if(B%d) return -1;
        C /= d; B /= d;
        ans = ans*(A/d)%C;
        ++cnt;
    }
    ll m = (ll)ceil(sqrt(C*1.0)), t = 1;
    for(ll i = 0; i < m; ++i) {
        H.ist(t, i);
        t = t*A%C;
    }
    for(ll i = 0, j; i < m; ++i) {
        ExGcd(ans, C, x, y);
        ll val = x*B%C;
        val = (val%C + C)%C;
        if(H.get(val, j))  return m*i + j + cnt;
        ans = ans*t%C;
    }
    return -1;
}
\end{lstlisting}

\subsection{欧拉降幂公式}

欧拉定理:若正整数 $a, n$ 互质,则 $a^{Euler(n)} ≡ 1 (mod \space n)$

推论:若正整数 $a, n$ 互质,则对于任意正整数 $b$, 有 $a^b ≡ a^{b\%Euler(n)}(mod \space n)$

对于 $p$ 是素数,$Euler(p) = p - 1$ 且只有 $p$ 的倍数与 $p$ 不互质

\begin{lstlisting}[language=C++]
#define Mod(a, b) (a < b ? a : a%b + b)
unordered_map<ll, ll> p;
ll Pow(ll x, ll y, ll mod){
    ll ret = 1;
    while(y){
        if(y & 1) ret = Mod(ret*x, mod);
        x = Mod(x*x, mod);
        y >>= 1;
    }
    return ret;
}
ll phi(ll k){
    ll s = k, x = k;
    if(p[k] != 0) return p[k];
    for(ll i = 2; i*i <= k; ++i){
        if(k%i == 0) s = s/i*(i - 1);
        while(k%i == 0) k /= i;
    }
    if(k > 1) s = s/k*(k - 1);
    p[x] = s;
    return s;
}
ll a[maxn];
// return a[lf] ^ a[lf + 1] ^  ... ^ a[rt]
// 注意返回值需要再 % mod ,因为 Mod 被重载为欧拉降幂的形式
ll solve(int lf, int rt, ll mod){
    if(lf == rt || mod == 1) return Mod(a[lf], mod);
    return Pow(a[lf], solve(lf + 1, rt, phi(mod)), mod);
}
\end{lstlisting}

\subsection{因子相关}

\subsubsection{因子个数}

$[1, 5 \times 10^6]$ 中,$4324320$ 的因子数最多,有 $384$ 个因子

\subsubsection{因子和线性筛}

\begin{lstlisting}[language=C++]
const int maxn = 4e6 + 5;
bool np[maxn];
int p[maxn], tot;
ll sf[maxn], sp[maxn]; // sf: sum of factor sp: sum of prime
void init() {
    tot = 0; sf[1] = 1;
    for(int i = 2; i < maxn; ++i) {
        if(!np[i]) {
            p[tot++] = i;
            sf[i] = i + 1;
            sp[i] = i + 1;
        }
        for(int j = 0; j < tot && i*p[j] < maxn; ++j) {
            np[i*p[j]] = true;
            if(i%p[j] == 0) {
                sp[i*p[j]] = sp[i]*p[j] + 1;
                sf[i*p[j]] = sf[i]/sp[i]*sp[i*p[j]];
                break;
            }
            sf[i*p[j]] = sf[i]*sf[p[j]];
            sp[i*p[j]] = 1LL + p[j];
        }
    }
}
\end{lstlisting}

\subsection{威尔逊定理(利用阶乘判定素数的充要条件)}

在初等数论中,威尔逊定理给出了判定一个自然数是否为素数的充分必要条件。

即当且仅当 $p$ 为素数时:

$$(p -1)! ≡ -1 \space (mod \space p)$$


\subsection{斯特林近似(n 的阶乘的长度)}

斯特林公式 $n! = sqrt(2*PI*n)*(n/e)^n$

\begin{lstlisting}[language=C++]
const double PI = acos(-1.0);
ll solve(int n){
    double nn = n;
    ll ret = (ll)((0.5*log(2.0*PI*nn) + nn*log(nn) - nn)/log(10.0));
    ++ret;
    return ret;
}
\end{lstlisting}

\subsection{MIN25筛}

\begin{lstlisting}[language=C++]
namespace MIN25 {
    ll w[N], s1[N], s2[N];
    int Sqr, plen, m, p[N];

    int getid(ll x, ll n) {
        return x <= Sqr ? x : m - n / x + 1;
    }

    int S(ll n) {
        if (n & 1)
            return (n + 1) / 2 % MOD * (n % MOD) % MOD;
        return n / 2 % MOD * ((n + 1) % MOD) % MOD;
    }

    int f(int n, int c) {
        return (n ^ c) % MOD;
    }

    void pre(ll n) {
        Sqr = sqrt(n), m = 0;
        for (ll l = 1, r; l <= n; l = r + 1) {
            r = n / (n / l);
            w[++m] = r;
            s1[m] = r, s2[m] = S(r);
        }
        p[0] = 1;
        for(int i = 2; i <= Sqr; i++) {
            if (s1[i] != s1[i - 1]) {
                p[++plen] = i;
                ll lim = (ll)i * i;
                for (int j = m; lim <= w[j]; j--) {
                    int id1 = getid(w[j] / i, n), id2 = p[plen - 1];
                    s1[j] -= (s1[id1] - s1[id2]);
                    s2[j] -= (ll)i * (s2[id1] - s2[id2]);
                }
                s2[i] %= MOD;
            }
        }
        for(int i = 1; i <= m; i++) {
            s1[i] = (s2[i] - s1[i] + (i > 1) * 2) % MOD;
        }
    }

    ll sieve(ll n, int y) {
        if (n <= 1 || p[y] > n)
            return 0;
        pre(n);
        ll ans = s1[getid(n, n)] - s1[p[y - 1]];
        for (int i = y; i <= plen && (ll)p[i] * p[i] <= n; i++) {
            for (ll P = p[i], j = 1; P * p[i] <= n; P *= p[i], j++)
                ans += sieve(n / P, i + 1) * f(p[i], j) + f(p[i], j + 1);
        }
        return ans % MOD;
    }
}
\end{lstlisting}
