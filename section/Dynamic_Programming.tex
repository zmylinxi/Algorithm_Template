

\section{动态规划}

\subsection{最长不降(上升)子序列 O(nlogn)}

\subsubsection{模板}

\begin{lstlisting}[language=C++]
const int maxn = 1e5 + 5;
int a[maxn], b[maxn];
int LIS(int n){
    int tot = 0, ret = 0;
    for(int i = 0; i < n; ++i){
        int pos = lower_bound(b, b + tot, a[i]) - b;
        if(pos == tot) b[tot++] = a[i];
        else b[pos] = a[i];
        ret = max(ret, pos + 1);
    }
    return ret;
}
\end{lstlisting}

\subsubsection{最长上升子序列计数(离散化 + 树状数组)}


\begin{lstlisting}[language=C++]
const int mod = 1e9 + 7;
const int maxn = 5e4 + 5;
int a[maxn];
vector<int> b;
unordered_map<int, int> id;
int len, num;
int mx[maxn << 1], cnt[maxn << 1];
// 注意树状数组开 2 倍
#define lowbit(x) (x&(-x))
void query(int x) {
    while(x > 0) {
        if(mx[x] > len) {
            len = mx[x];
            num = cnt[x];
        } else if(mx[x] == len) {
            num = (cnt[x] + num)%mod;
        }
        x -= lowbit(x);
    }
}
void update(int x) {
    while(x < maxn) {
        if(mx[x] < len) {
            mx[x] = len;
            cnt[x] = num;
        } else if(mx[x] == len) {
            cnt[x] = (cnt[x] + num)%mod;
        }
        x += lowbit(x);
    }
}
int main() {
    int n; scanf("%d", &n);
    for(int i = 1; i <= n; ++i) {
        scanf("%d", &a[i]);
        b.push_back(a[i]);
        b.push_back(a[i] - 1);
    }
    sort(b.begin(), b.end());
    b.erase(unique(b.begin(), b.end()), b.end());
    int sz = b.size();
    for(int i = 0; i < sz; ++i) id[b[i]] = i + 1;
    for(int i = 1; i <= n; ++i) a[i] = id[a[i]];
    for(int i = 1; i <= n; ++i) {
        len = 0, num = 1;
        query(a[i] - 1);
        ++len;
        update(a[i]);
    }
    len = 0, num = 1;
    query(maxn - 1);
    printf("%d\n", num);
    return 0;
}
\end{lstlisting}

\subsection{编辑距离}

dp[i][j] 记录 s1 的前 i 个字符转换到 s2 的前 j 个字符的最小编辑距离

\begin{lstlisting}[language=C++]
int dp[maxn][maxn];
int CompareString(char *s1, char *s2) {
    memset(dp, 0, sizeof(dp));
    int m = strlen(s1), n = strlen(s2);
    for(int i = 0; i <= m; ++i) dp[i][0] = i;
    for(int j = 0; j <= n; ++j) dp[0][j] = j;
    for(int i = 1; i <= m; ++i){
        for(int j = 1; j <= n; ++j){
            int cost = 0;
            if(s1[i - 1] == s2[j - 1]) cost = 0;
            else cost = 1;
            dp[i][j] = min(min(dp[i - 1][j - 1] + cost, dp[i - 1][j] + 1), dp[i][j - 1] + 1);
        }
    }
    return dp[m][n];
}    
\end{lstlisting}

\subsection{计算 hx 进制下所有位 num 出现的次数}

\begin{lstlisting}[language=C++]
ll tot[20];
// n 数字最大位数, hx 进制。 比如 若最大数字为 1e18, 则 n = 19
void init(int n, ll hx) {
    tot[0] = 0;
    ll bs = 1LL;
    for(int i = 1; i < n; ++i) {
        tot[i] = tot[i - 1]*hx + bs;
        bs *= hx;
    }
    for(int i = 0; i < n; ++i) {
        printf("%d: %lld\n", i, tot[i]);
    }
}
// 计算 hx 进制下 [0, x] ,所有位(考虑算上前导零共 n 位) num 出现的次数 (lz 为是否考虑前导零)
ll cnt(int n, ll x, ll num, ll hx, bool lz) {
    ll ret = 0, dig = 0, bs = 1, tail = 0, xx = x;
    int len = 0;
    if(lz) {
        // 计算前导零
        while(len < n) {
            dig = x%hx;
            x /= hx;
            if(dig > num) ret += bs + dig*tot[len];
            else if(dig == num) ret += tail + 1LL + dig*tot[len];
            else ret += dig*tot[len];
            tail += dig*bs;
            bs *= hx;
            ++len;
        }
    } else {
        // 不计算前导零
        while(x > 0) {
            dig = x%hx;
            x /= hx;
            if(dig > num) ret += bs + dig*tot[len];
            else if(dig == num) ret += tail + 1LL + dig*tot[len];
            else ret += dig*tot[len];
            tail += dig*bs;
            bs *= hx;
            ++len;
        }
        if(num == 0) {
            bs = 1;
            while(xx > 0) {
                ret -= bs;
                xx /= hx;
                bs *= hx;
            }
        }
    }
    return ret;
} 
\end{lstlisting}

\subsection{期望、概率dp}

[讲解博客链接](http://www.cnblogs.com/CzYoL/p/7220088.html)

【期望dp】

求解达到某一目标的期望花费:

因为最终的花费无从知晓(不可能从∞推起),所以期望dp需要倒序求解。

设f[i][j]表示在(i, j)这个状态实现目标的期望值(相当于是差距是多少)。

首先f[n][m] = 0,在目标状态期望值为0。

然后f = (∑f′ × p) + w,f′为上一状态(距离目标更近的那个,倒序),p为从f转移到f′的概率(则从f′转移回f的概率也为p),w为转移的花费。

最后输出初始位置的f即可。

特别的,当转移关系不成环时,期望dp可以线性递推。

但当转移关系成环时,期望dp的最终状态相当于一个已知量,而转移关系相当于一个个方程,可以使用【高斯消元】解决。

【概率dp】

概率dp通常已知初始的状态,然后求解最终达到目标的概率,所以概率dp需要顺序求解。

概率dp相对简单,当前状态只需加上所有上一状态乘上转移概率即可:f = ∑f′i × pi

\subsection{数位 dp}

\subsubsection{模板}

\begin{lstlisting}[language=C++]
int a[20];
ll dp[20][state]; // 不同题目状态不同
ll dfs(int pos, /* state变量 */, bool lead /* 前导零 */, bool limit /* 数位上界变量 */) // 不是每个题都要判断前导零
{
    // 递归边界,既然是按位枚举,最低位是 0,那么pos == -1 说明这个数我枚举完了
    if (pos == -1) return 1; /*这里一般返回1,表示你枚举的这个数是合法的,那么这里就需要你在枚举时必须每一位都要满足题目条件,也就是说当前枚举到pos位,一定要保证前面已经枚举的数位是合法的。不过具体题目不同或者写法不同的话不一定要返回1 */
    // 第二个就是记忆化(在此前可能不同题目还能有一些剪枝)
    if (!limit && !lead && dp[pos][state] != -1) return dp[pos][state];
    /*常规写法都是在没有限制的条件记忆化,这里与下面记录状态是对应,具体为什么是有条件的记忆化后面会讲*/
    int up = limit ? a[pos] : 9; // 根据limit判断枚举的上界up;这个的例子前面用213讲过了
    ll ans = 0;
    // 开始计数
    for (int i = 0; i <= up; i++) // 枚举,然后把不同情况的个数加到ans就可以了
    {
        if () ...
        else if ()...
        ans += dfs(pos - 1, /* 状态转移 */, lead && i == 0, limit && i == a[pos]) // 最后两个变量传参都是这样写的
        /* 这里还算比较灵活,不过做几个题就觉得这里也是套路了
        大概就是说,我当前数位枚举的数是 i,然后根据题目的约束条件分类讨论
        去计算不同情况下的个数,还有要根据 state 变量来保证 i 的合法性,比如题目
        要求数位上不能有 62 连续出现,那么就是 state 就是要保存前一位 pre,然后分类,
        前一位如果是 6 那么这意味就不能是 2,这里一定要保存枚举的这个数是合法 */
    }
    // 计算完,记录状态
    if(!limit && !lead) dp[pos][state] = ans;
    /* 这里对应上面的记忆化,在一定条件下时记录,保证一致性,当然如果约束条件不需要考虑 lead,这里就是 lead 就完全不用考虑了 */
    return ans;
}
ll solve(ll x)
{
    int pos = 0;
    while (x) // 把数位都分解出来
    {
        a[pos++] = x%10; // 个人老是喜欢编号为[0, pos),看不惯的就按自己习惯来,反正注意数位边界就行
        x /= 10;
    }
    return dfs(pos - 1 /* 从最高位开始枚举 */, /* 一系列状态 */, true, true); // 刚开始最高位都是有限制并且有前导零的,显然比最高位还要高的一位视为 0 嘛
}
\end{lstlisting}

\subsubsection{数字计数 count}


给定两个正整数 a 和 b ,求在 [a, b] 中的所有整数中,每个数码(digit)各出现了多少次。

\begin{lstlisting}[language=C++]
#include <iostream>
#include <cstdio>
#include <cmath>
using namespace std;
typedef long long ll;
const int mod = 1e9 + 7;
const int maxn = 1e5 + 5;
ll dp[12][15][15];
int a[15];
ll dfs(int pos, int dig, bool lead, bool limit, int cnt) {
    if(pos == -1) return cnt;
    if(!limit && !lead && dp[dig][pos][cnt] != -1) return dp[dig][pos][cnt];
    int num = a[pos];
    int up = 9;
    if(limit) up = num;
    ll ret = 0;
    for(int i = 0; i <= up; ++i) {
        int nc = cnt;
        if(i == dig) {
            if(dig != 0) ++nc;
            else if(!lead) ++nc;
        }
        ret += dfs(pos - 1, dig, lead && (i == 0), limit && (i == up), nc);
    }
    if(!lead && !limit) dp[dig][pos][cnt] = ret;
    return ret;
}
ll solve(ll x, int dig) {
    int pos = 0;
    while(x > 0) {
        a[pos++] = x%10LL;
        x /= 10LL;
    }
    ll ret = dfs(pos - 1, dig, true, true, 0);
    return ret;
}
int main() {
    ll a, b;
    for(int i = 0; i < 10; ++i) {
        for(int j = 0; j <= 13; ++j) {
            for(int k = 0; k <= 13; ++k) {
                dp[i][j][k] = -1;
            }
        }
    }
    while(scanf("%lld%lld", &a, &b) != EOF) {
        for(int i = 0; i <= 9; ++i) {
            ll ans = solve(b, i) - solve(a - 1, i);
            if(i > 0) printf(" ");
            printf("%lld", ans);
        }
        puts("");
    }
    return 0;
}
\end{lstlisting}

\subsection{括号序列相关}

括号序列配对时满足贪心性质,从左向右考虑,能配对的括号直接配对,结果不会变差。

\subsubsection{最长合法括号序列}

\begin{lstlisting}[language=C++]
char str[maxn];
int dp[maxn];
stack<int> s;
int mx, sum;//最长长度,以及最长长度的个数
void LongestRegularBracketSequence(){
    mx = 0, sum = 1;
    int len = strlen(str);
    for(int i = 0; i < len; i++) {
        if(str[i] == '('){
            s.push(i);
            continue;
        }
        if(!s.empty()) {
            int now = s.top();
            s.pop();
            dp[i] = i - now + 1;
            if(now) dp[i] += dp[now - 1];
            if(mx < dp[i]) {
                mx = dp[i];
                sum = 0;
            }
            if(mx == dp[i]) ++sum;
        }
    }
}
\end{lstlisting}

\subsubsection{区间询问最长合法括号序列}

\begin{lstlisting}[language=C++]
struct SegTree {
    int lf, rt;
    int a, b, c;
} st[maxn << 2];
char s[maxn];
void PushUp(int x) {
    SegTree &now = st[x];
    SegTree &lc = st[x << 1];
    SegTree &rc = st[x << 1 | 1];
    int ok = min(lc.a, rc.b);
    now.a = lc.a + rc.a - ok;
    now.b = lc.b + rc.b - ok;
    now.c = lc.c + rc.c + ok;
}
void Build(int x, int lf, int rt) {
    SegTree &now = st[x];
    now.lf = lf; now.rt = rt;
    now.a = now.b = now.c = 0;
    if(lf == rt) {
        if(s[lf] == '(') ++now.a;
        else ++now.b;
        return ;
    }
    int mid = (lf + rt) >> 1;
    Build(x << 1, lf, mid);
    Build(x << 1 | 1, mid + 1, rt);
    PushUp(x);
}
int qa, qb, qc;
void Query(int x, int lf, int rt) {
    SegTree &now = st[x];
    if(now.lf >= lf && now.rt <= rt) {
        int ok = min(qa, now.b);
        qa = qa + now.a - ok;
        qb = qb + now.b - ok;
        qc = qc + now.c + ok;
        return ;
    }
    int mid = (now.lf + now.rt) >> 1;
    if(lf <= mid) Query(x << 1, lf, rt);
    if(rt > mid) Query(x << 1 | 1, lf, rt);
}
int main() {
    scanf("%s", s + 1);
    int n = strlen(s + 1);
    Build(1, 1, n);
    int m; scanf("%d", &m);
    for(int im = 1; im <= m; ++im) {
        int l, r; scanf("%d%d", &l, &r);
        qa = qb = qc = 0;
        Query(1, l, r);
        printf("%d\n", qc*2);
    }
    return 0;
}
\end{lstlisting}

\subsection{斜率优化 dp}

\subsubsection{模板}

\begin{lstlisting}[language=C++]
// 此处维护下凸壳
// 若需要维护上凸壳,则需要改变 judge1(), judge2() 中的不等号方向
ll dp[maxn], up[maxn], down[maxn], k[maxn];
int mq[maxn];
bool judge1(int x, int y, int z) {
    ll v1 = up[y] - up[x];
    ll v2 = (down[y] - down[x])*k[z];
    if(v1 <= v2) return true;
    return false;
}
bool judge2(int x, int y, int z) {
    ll v1 = (up[y] - up[x])*(down[z] - down[y]);
    ll v2 = (up[z] - up[y])*(down[y] - down[x]);
    if(v1 >= v2) return true;
    return false;
}
void getZero() {

}
ll getK(int x) {
    ll ret = 0;
    return ret;
}
ll getDp(int x, int y) {
    ll ret = 0;
    return ret;
}
ll getUp(int x) {
    ll ret = 0;
    return ret;
}
ll getDown(int x) {
    ll ret = 0;
    return ret;
}
ll solve(int n) {
    getZero();
    int lf = 0, rt = 0;
    for(int i = 0; i <= n; ++i) {
        k[i] = getK(i);
        while(lf + 1 < rt && judge1(mq[lf], mq[lf + 1], i)) ++lf;
        dp[i] = getDp(mq[lf], i);
        up[i] = getUp(i); down[i] = getDown(i);
        while(lf + 1 < rt && judge2(mq[rt - 2], mq[rt - 1], i)) --rt;
        mq[rt++] = i;
    }
    return dp[n];
}
\end{lstlisting}

\subsubsection{例题}

dp 方程:$dp[i] = min{dp[j] + pre[j] − x[i] × sum[j])} + c[i] + sum[i] × x[i] − pre[i]$

\begin{lstlisting}[language=C++]
const int maxn = 1e6 + 5;
ll x[maxn], p[maxn], c[maxn];
ll sum[maxn], dp[maxn], pre[maxn];
ll up[maxn], down[maxn];
int mq[maxn], lf = 0, rt = 1;
int judge1(int p1, int p2, ll val) {
    ll dt = up[p2] - up[p1] - (down[p2] - down[p1])*val;
    if(dt == 0) return 0;
    if(dt < 0) return -1;
    return 1;
}
int judge2(int p1, int p2, int p3) {
    ll dt = (up[p3] - up[p2])*(down[p2] - down[p1]) - (up[p2] - up[p1])*(down[p3] - down[p2]);
    if(dt == 0) return 0;
    if(dt < 0) return -1;
    return 1;
}
int main(){
    int n; scanf("%d", &n);
    for(int i = 1; i <= n; ++i) scanf("%lld%lld%lld", &x[i], &p[i], &c[i]);
    for(int i = 1; i <= n; ++i) {
        sum[i] = sum[i - 1] + p[i];
        pre[i] = pre[i - 1] + p[i]*x[i];
    }
    for(int i = 1; i <= n; ++i) {
        while(rt - lf >= 2 && judge1(mq[lf], mq[lf + 1], x[i]) < 0) ++lf;
        int id = mq[lf];
        dp[i] = dp[id] + (sum[i] - sum[id])*x[i] - (pre[i] - pre[id]) + c[i];
        up[i] = dp[i] + pre[i]; down[i] = sum[i];
        while(rt - lf >= 2 && judge2(mq[rt - 2], mq[rt - 1], i) < 0) --rt;
        mq[rt++] = i;
    }
    printf("%lld\n", dp[n]);
    return 0;
}
\end{lstlisting}

\subsection{四边形不等式}

\subsubsection{基本理论}

如果对于任意的 $a_1 \le a_2 < b_1 \le b_2 $ 有 $m[a_1, b_1] + m[a_2, b_2] \le m[a_1, b_2] + m[a_2, b_1]$,那么 $m[i, j]$ 满足四边形不等式。

设 $m[i, j]$ 表示动态规划的状态量。

$m[i, j]$ 有类似如下的状态转移方程:

$m[i,j] = min\{m[i, k] + m[k, j]\} (i \le k \le j)$

m满足四边形不等式是适用这种优化方法的必要条件

对于一道具体的题目,我们首先要证明它满足这个条件,一般来说用数学归纳法证明,根据题目的不同而不同。

通常的动态规划的复杂度是 $O(n^3)$,我们可以优化到 $O(n^2)$

定义 s(i, j) 为函数 m(i, j) 对应的使得 m(i, j) 取得最小值的 k 值。

我们可以证明,s[i, j - 1] <= s[i, j] <= s[i + 1, j]

那么改变状态转移方程为:

m[i, j] = min{m[i, k] + m[k, j]} (s[i, j - 1] <= k <= s[i + 1, j])

复杂度分析:不难看出,复杂度决定于 s 的值,以求 m[i, i + L] 为例,

(s[2, L + 1] - s[1, L]) + (s[3, L + 2] - s[2, L + 1]) … + (s[n - L + 1, n] - s[n - L, n - 1]) = s[n - L + 1, n] - s[1, L] <= n

所以总复杂度是 $O(n)$

\subsubsection{优化环形石子合并  ($O(n^3)$ To $O(n^2)$)}

N 堆石子摆成一个环。现要将石子有次序地合并成一堆。

规定每次只能选相邻的 2 堆石子合并成新的一堆,并将新的一堆石子数记为该次合并的代价。

计算将N堆石子合并成一堆的最小代价。

\begin{lstlisting}[language=C++]
#include <bits/stdc++.h>
using namespace std;
typedef long long ll;
const int inf = 0x3f3f3f3f;
const ll linf = 0x3f3f3f3f3f3f3f3f;
const int mod = 1e9 + 7;
const int maxn = 2e3 + 5;
ll a[maxn], pre[maxn];
ll m[maxn][maxn], w[maxn][maxn];
int s[maxn][maxn], n;
void init() {
    for(int i = 1; i <= n; ++i) a[i + n] = a[i];
    for(int i = 1; i <= 2*n; ++i) {
        pre[i] = pre[i - 1] + a[i];
        for(int j = 1; j <= 2*n; ++j) {
            if(j > i) w[j][i] = linf;
            else w[j][i] = pre[i] - pre[j - 1];
            m[i][j] = linf;
        }
    }
    for(int i = 1; i <= 2*n; ++i) {
        m[i][i] = 0;
        s[i][i] = i;
    }
}
ll solve() {
    for(int len = 2; len <= n; ++len) {
        for(int lf = 1; lf <= 2*n; ++lf) {
            int rt = lf + len - 1;
            if(rt >= 2*n) break;
            for(int mid = s[lf][rt - 1]; mid <= s[lf + 1][rt]; ++mid) {
                ll now = m[lf][mid] + m[mid + 1][rt] + w[lf][rt];
                if(now > m[lf][rt]) continue;
                s[lf][rt] = mid; m[lf][rt] = now;
            }
        }
    }
    ll ret = linf;
    for(int i = 1; i <= n; ++i) ret = min(ret, m[i][i + n - 1]);
    return ret;
}
int main() {
    scanf("%d", &n);
    for(int i = 1; i <= n; ++i) scanf("%lld", &a[i]);
    init();
    ll ans = solve();
    printf("%lld\n", ans);
    return 0;
}
\end{lstlisting}

\subsection{最大 m 子段和}

\begin{lstlisting}[language=C++]
const int maxn = 5e3 + 5;
ll a[maxn], dp[2][maxn];
ll MaxSum(int m, int n) {
    memset(dp, 0, sizeof(dp));
    int now = 1, lst = 0;
    for (int i = 1; i <= m; i++) {
        dp[now][i] = dp[lst][i - 1] + a[i];
        ll mx = dp[lst][i - 1];
        for (int j = i + 1; j <= n - m + i; j++) {
            mx = max(mx, dp[lst][j - 1]);
            dp[now][j] = max(dp[now][j - 1], mx) + a[j];
        }
        swap(now, lst);
    }
    ll ret = -linf;
    for (int j = m; j <= n; ++j) ret = max(ret, dp[lst][j]);
    return ret;
}
\end{lstlisting}


\subsection{虚树 DP}

只需按照需求修改 solve() 函数并用 lca() 维护相关信息即可。

给定一棵有 $n$ 个节点的树,$q$ 次询问,每次询问给 $k$ 个点,求至少删除多少点,使得这 $k$ 个点两两不属于同一联通块 $n, q, \sigma k \le 100000$

\begin{lstlisting}[language=C++]
#include <bits/stdc++.h>
using namespace std;
typedef long long ll;
typedef unsigned long long ull;
const int inf = 0x3f3f3f3f;
const ll linf = 0x3f3f3f3f3f3f3f3f;
const ll mod = 998244353LL;

const int maxn = 1e5 + 5;
const int maxm = log2(maxn) + 2;
struct Graph {
    struct Edge {
        int to, nxt;
    } E[maxn << 1];
    int head[maxn], tot;
    void init(int n) {
        for (int i = 0; i <= n; ++i) head[i] = -1;
        tot = 0;
    }
    void add_edge (int u, int v) {
        E[tot] = {v, head[u]};
        head[u] = tot++;
    }
} G, VG;
int dfn[maxn], cnt;
int d[maxn], fa[maxn][maxm];
void dfs(int x, int y) {
    dfn[x] = ++cnt;
    for (int i = 1; i < maxm; ++i) {
        fa[x][i] = fa[fa[x][i - 1]][i - 1];
    }
    for (int i = G.head[x]; i != -1; i = G.E[i].nxt) {
        int to = G.E[i].to;
        if (to == y) continue;
        fa[to][0] = x;
        d[to] = d[x] + 1;
        dfs(to, x);
    }
}
int lca(int a, int b) {
    if (d[a] > d[b]) swap(a, b);
    int dt = d[b] - d[a];
    for (int i = 0; (1 << i) <= dt; ++i) {
        if (((dt >> i) & 1) == 0) continue;
        b = fa[b][i];
    }
    for (int i = maxm - 1; i >= 0; --i) {
        if (fa[a][i] == fa[b][i]) continue;
        a = fa[a][i]; b = fa[b][i];
    }
    if (a == b) return a;
    return fa[a][0];
}
int id[maxn];
int st[maxn], tp;
int vis[maxn];
ll dp[maxn][2];
void solve(int x, int cq) {
    ll sum0 = 0, sum1 = 0;
    ll mx = -inf;
    for (int i = VG.head[x]; i != -1; i = VG.E[i].nxt) {
        int to = VG.E[i].to;
        solve(to, cq);
        if (d[to] > d[x] + 1) dp[to][0] = min(dp[to][0], dp[to][1] + 1LL);
        sum0 += dp[to][0];
        sum1 += dp[to][1];
        mx = max(mx, dp[to][0] - dp[to][1]);
    }
    if (vis[x] == cq) {
        dp[x][0] = inf;
        dp[x][1] = sum0;
    } else {
        dp[x][0] = min(sum0, 1LL + sum1);
        dp[x][1] = min(dp[x][0], sum0 - mx);
    }
    if (dp[x][0] > inf) dp[x][0] = inf;
    if (dp[x][1] > inf) dp[x][1] = inf;
}
void vir_tree (int cq) {
    int k; scanf("%d", &k);
    for (int i = 0; i < k; ++i) {
        scanf("%d", &id[i]);
        vis[id[i]] = cq;
    }
    sort(id, id + k, [&](const int &x, const int &y) -> bool {
        return dfn[x] < dfn[y];
    });
    tp = 0;
    VG.head[1] = -1; st[tp++] = 1;
    for (int i = 0; i < k; ++i) {
        int x = id[i];
        if (x == 1) continue;
        int lx = lca(x, st[tp - 1]);
        if (lx != st[tp - 1]) {
            while (tp > 1 && dfn[lx] < dfn[st[tp - 2]]) {
                VG.add_edge(st[tp - 2], st[tp - 1]);
                --tp;
            }
            if (tp > 1 && dfn[lx] > dfn[st[tp - 2]]) {
                VG.head[lx] = -1;
                VG.add_edge(lx, st[tp - 1]);
                st[tp - 1] = lx;
            } else {
                VG.add_edge(lx, st[tp - 1]);
                --tp;
            }
        }
        VG.head[x] = -1; st[tp++] = x;
    }
    for (int i = 0; i < tp - 1; ++i) {
        VG.add_edge(st[i], st[i + 1]);
    }
    solve(1, cq);
    int ans = min(dp[1][0], dp[1][1]);
    if (ans == inf) ans = -1;
    printf("%d\n", ans);
}
int main() {
    int n; scanf("%d", &n);
    G.init(n); VG.init(n);
    for (int i = 1; i < n; ++i) {
        int u, v; scanf("%d%d", &u, &v);
        G.add_edge(u, v);
        G.add_edge(v, u);
    }
    cnt = fa[1][0] = 0;
    d[1] = 1;
    dfs(1, -1);
    for (int i = 1; i <= n; ++i) vis[i] = 0;
    int q; scanf("%d", &q);
    for (int cq = 1; cq <= q; ++cq) {
        vir_tree(cq);
    }
    return 0;
}
\end{lstlisting}