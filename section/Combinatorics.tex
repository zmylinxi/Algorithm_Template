

\section{组合数学}

\subsection{组合数}

\subsubsection{模板}

\begin{lstlisting}[language=C++]
const int maxn = 1e6 + 5;
const ll mod = 1e9 + 7;
ll fac[maxn], invFac[maxn];
ll pw(ll a, ll b){
    ll ret = 1;
    while(b){
        if(b&1) ret = (ret*a)%mod;
        a = (a*a)%mod;
        b >>= 1;
    }
    return ret;
}
void getFacInv(int n){
    fac[0] = 1;
    for(int i = 1; i <= n; ++i) fac[i] = (fac[i - 1]*i)%mod;
    invFac[n] = pw(fac[n], mod - 2);
    for(int i = n - 1; i >= 0; --i) invFac[i] = (invFac[i + 1]*(i + 1))%mod;
}
ll C(int a, int b){
    if(b < 0 || b > a) return 0LL;
    ll ret = (fac[a]*invFac[b])%mod;
    ret = (ret*invFac[a - b])%mod;
    return ret;
}
\end{lstlisting}

\subsubsection{Lucas 定理(求解 n, m 大,小模数取模的组合数)}

在取模意义下:

$C(n,m) = (C(n\%mod,m\%mod)*C(n/mod,m/mod))\%mod$

\begin{lstlisting}[language=C++]
ll Lucas(ll a, ll b){
    if(b == 0) return 1;
    ll ret = (C(a%mod, b%mod, mod)*Lucas(a/mod, b/mod))%mod;
    return ret;
}
\end{lstlisting}

\subsection{康托展开与逆康托展开(全排列到自然数的双射)}

康托展开是一个全排列到一个自然数的双射,常用于构建哈希表时的空间压缩。

$$X = a_n(n - 1)! + a_{n - 1}(n - 2)! + \cdots + a_1\times 0!$$

其中 $a_i$ 为 $0 <= a_i < i$ 的非负整数,$1 <= i <= n$

$a_i$ 表示原数的第 $i$ 位在当前未出现的元素中是排在第几个

(也可以理解为位置 $i$ 后面的数有多少个是小于第 $i$ 位的数字)

\begin{lstlisting}[language=C++]
int a[12], fac[12];
bool vis[12];
void getfac(int n) {
    fac[0] = 1;
    for (int i = 1; i < n; ++i) fac[i] = fac[i - 1]*i;
}
int cantor(int n) {
    int ret = 0;
    for (int i = 1; i <= n; ++i) {
        int cnt = 0;
        for (int j = i + 1; j <= n; ++j) if(a[i] > a[j]) ++cnt;
        ret += cnt*fac[n - i];
    }
    return ret;
}
void decantor(int x, int n) {
    for (int i = 1; i <= n; ++i) vis[i] = false;
    int tot = 0;
    for (int i = n; i >= 1; --i) {
        int q = x/fac[i - 1];
        int cnt = 0;
        for (int j = 1; j <= n; ++j) {
            if (vis[j]) continue;
            if (++cnt > q) {
                a[++tot] = j;
                vis[j] = true;
                break;
            }
        }
        x = x%fac[i - 1];
    }
}
\end{lstlisting}

\subsection{快速傅里叶变换 FFT (加速加法操作卷积运算 O(nlogn))}

\begin{lstlisting}[language=C++]
const int maxn = 2e5 + 5;
const int fMaxn = maxn << 2; // fMaxn >= 4*maxn (1025 need 4096)
namespace FFT {
    // typedef long double ld;
    const double pi = acos(-1.0);
    struct Complex {
        double r, i;
        Complex(double r = 0, double i = 0):r(r), i(i) {};
        Complex operator + (const Complex & rhs) {
            return Complex(r + rhs.r, i + rhs.i);
        }
        Complex operator - (const Complex & rhs) {
            return Complex(r - rhs.r, i - rhs.i);
        }
        Complex operator * (const Complex &rhs) {
            return Complex(r * rhs.r - i * rhs.i, i * rhs.r + r * rhs.i);
        }
    } va[fMaxn], vb[fMaxn], vc[fMaxn];
    void change(Complex F[], int len) {
        int j = len >> 1;
        for(int i = 1; i < len - 1; ++i) {
            if(i < j) {
                swap(F[i], F[j]);
            }
            int k = len >> 1;
            while(j >= k) {
                j -= k;
                k >>= 1;
            }
            if(j < k) {
                j += k;
            }
        }
    }
    void fft(Complex F[], int len, int t) {
        change(F, len);
        for(int h = 2; h <= len; h <<= 1) {
            Complex wn(cos(-t*2.0*pi/h), sin(-t*2.0*pi/h));
            for(int j = 0; j < len; j += h) {
                Complex E(1, 0);
                for(int k = j; k < j + h/2; ++k) {
                    Complex u = F[k];
                    Complex v = E*F[k + h/2];
                    F[k] = u + v;
                    F[k + h/2] = u - v;
                    E = E*wn;
                }
            }
        }
        if(t == -1) {
            for(int i = 0; i < len; i++) {
                F[i].r /= len;
            }
        }
    }
    template<class T>
    int init(T a[], T b[], int la, int lb) {
        int lc = 1, mi = 2*max(la, lb);
        while(lc < mi) {
            lc <<= 1;
        }
        for(int i = 0; i < la; ++i) {
            va[i] = {(double)a[i], 0.0};
        }
        for(int i = la; i < lc; ++i) {
            va[i] = {0.0, 0.0};
        }
        for(int i = 0; i < lb; ++i) {
            vb[i] = {(double)b[i], 0.0};
        }
        for(int i = lb; i < lc; ++i) {
            vb[i] = {0.0, 0.0};
        }
        return lc;
    }
    template<class T>
    int solve(T a[], T b[], T c[], int la, int lb) { // id: [0, la), [0, lb), [0, lc)
        int lc = init(a, b, la, lb);
        fft(va, lc, 1);
        fft(vb, lc, 1);
        for(int i = 0; i < lc; ++i) {
            vc[i] = va[i]*vb[i];
        }
        fft(vc, lc, -1);
        for(int i = 0; i < lc; ++i) {
            c[i] = vc[i].r; // double
            // c[i] = (int)round(vc[i].r); // int
            // c[i] = (ll)round(vc[i].r); // ll
        }
        return lc;
    }
}
\end{lstlisting}

\subsection{n 个 m 边形划分平面}

用 n 个 m 边方形最多可以把平面分成 S(n, m) = 2 + (n - 1)*n*m 个区域

\subsection{高维前缀和(对子集(超集)求和)}

\begin{lstlisting}[language=C++]
void GetSum(int n, int mx) {
    // sum[S] = \Sigma sum[T], T \in S (T 是 S 的子集)
    // 对子集求和
    // for(int i = 0; i < n; ++i) {
    //     for(int j = 0; j <= mx; ++j) {
    //         if((j & (1 << i)) != 0) sum[j] += sum[j ^ (1 << i)];
    //     }
    // }

    // sum[S] = \Sigma sum[T], S \in T (T 是 S 的超集)
    // 对超集求和
    for(int i = 0; i < n; ++i) {
        for(int j = 0; j <= mx; ++j) {
            if((j & (1 << i)) != 0) sum[j ^ (1 << i)] += sum[j];
        }
    }
}
// GetSum(int(floor(log2(1000000)) + 1), 1000000);
// GetSum(n, (1 << n) - 1);
\end{lstlisting}

\subsection{快速沃尔什变换 FWT (加速位操作卷积操作 O(nlogn))}

$$H[x] = \sum \limits _ {i \space OP \space j = x} F[i]*G[j]$$

的位操作卷积(OP 代表一种位运算,比如:与、或、异或)

\begin{lstlisting}[language=C++]
const int mod = 1e9 + 7;
// inv2 = pw(2, p - 2, p);
const int inv2 = 5e8 + 4; 
// ty == 1: FWT, ty == -1: IFWT
void FWT(int a[], int n, int ty){  
    for(int d = 1; d < n; d <<= 1)  
        for(int m = d << 1, i = 0; i < n; i += m)  
            for(int j = 0; j < d; ++j){  
                int x = a[i + j], y = a[i + j + d];  
                if(ty == 1) {
                    a[i + j] = (x + y)%mod;
                    a[i + j + d] = (x - y + mod)%mod;  
                    //xor: a[i + j] = x + y; a[i + j + d] = x - y;  
                    //and: a[i + j] = x + y;
                    //or: a[i + j + d] = x + y;
                } else {
                    a[i + j] = 1LL*(x + y)*inv2%mod;
                    a[i + j + d] = (1LL*(x - y)*inv2%mod + mod)%mod;
                    //xor: a[i + j] = (x + y)/2; a[i + j + d] = (x - y)/2;
                    //and: a[i + j] = x - y;
                    //or: a[i + j + d] = y - x;
                }
            }  
}
void solve(int a[], int b[], int n){  
    FWT(a, n, 1);
    FWT(b, n, 1);
    for(int i = 0; i < n; ++i) {
        a[i] = 1LL*a[i]*b[i]%mod;
    }  
    FWT(a, n, -1);
}
\end{lstlisting}

\subsection{整数划分}

\subsubsection{划分为不同整数}

将 $N (1 <= N <= 50000)$ 分为若干个不同整数的和,有多少种不同的划分方式,例如:n = 6,{6} {1, 5} {2, 4} {1, 2, 3},共4种。由于数据较大,输出 Mod $10^9 + 7$ 的结果即可。

$F[i][j]$ 表示将数 $i$ 分为 $j$ 份的方案数总和。

于是便有了状态转移方程:

$$f[i][j] := f[i - j][j] + f[i - j][j - 1]$$

下面来具体解释一下这个方程:

1、$f[i - j][j]$ 表示的是将i分为不包含1(min >= 2)的方案总个数,例如,6( = 9 - 3)分成3份可以分为{1, 2, 3},则9可以分为{1 + 1, 2 + 1, 3 + 1} -> {2, 3, 4} 仅1种

2、$f[i - j][j - 1]$ 表示的是将i分为包含 1 (min = 1)的方案总个数,例如,6( = 9 - 3)分成2( = 3 - 1)可以分为{0, 1, 5}{0, 2, 4}

则 9 可以分为 {0 + 1, 1 + 1, 5 + 1} {0 + 1, 2 + 1, 4 + 1} -> {1, 2, 6}{1, 3, 5}(共 2 种)

\begin{lstlisting}[language=C++]
#include <bits/stdc++.h>
using namespace std;
typedef long long ll;
const int mod = 1e9 + 7;
const int maxn = 5e4 + 5;
int dp[maxn][405];
int main() {
    int n; scanf("%d", &n);
    int m = sqrt(2*n) + 1;
    dp[0][0] = 1;
    for(int i = 1; i <= n; ++i) {
        for(int j = 1; j <= m; ++j) {
            if(j > i) break;
            dp[i][j] = (dp[i - j][j] + dp[i - j][j - 1])%mod;
        }
    }
    int ans = 0;
    for(int i = 1; i <= m; ++i) ans = (ans + dp[n][i])%mod;
    printf("%d\n", ans);
    return 0;
}
\end{lstlisting}

\subsubsection{可存在相同整数}

将 $N (1 <= N <= 50000)$ 分为若干个整数的和,有多少种不同的划分方式,例如:n = 4,{4}  {1,3}  {2,2}  {1,1,2} {1,1,1,1},共 5 种。由于数据较大,输出 Mod $10^9 + 7$ 的结果即可。

欧拉的五边形定理:

$P(n)$ 表示 $n$ 的划分种数

$$P(n) = \sum_{k \ge 1} (P(n - \frac{k(3k - 1)}{2}) + P(n - \frac{k(3k + 1)}{2}))$$

n < 0 时,P(n) = 0, n = 0 时, P(n) = 1

\begin{lstlisting}[language=C++]
const int mod = 1e9 + 7;
const int maxn = 5e4 + 5;
int dp[maxn];
void solve() {
    int n; scanf("%d", &n);
    dp[0] = 1;
    for(int i = 1; i <= n; ++i) {
        dp[i] = 0;
        for(int j = 1; ; ++j) {
            int k1 = j*(3*j - 1)/2;
            int k2 = j*(3*j + 1)/2;
            if(k1 > i && k2 > i) break;
            int f;
            if(j%2 == 1) f = 1;
            else f = -1;
            if(k1 <= i) dp[i] += f*dp[i - k1];
            dp[i] %= mod;
            if(k2 <= i) dp[i] += f*dp[i - k2];
            dp[i] %= mod;
            if(dp[i] < 0) dp[i] += mod;
        }
    }
    printf("%d\n", dp[n]);
}
\end{lstlisting}

\subsection{Catalan 数相关}

递推公式:

$Catalan_{n + 1} = \Sigma _ {i = 0} ^ {n} Catalan_{i} Catalan_{n - i}$

$Catalan_{n} = \frac{C_{2n - 2} ^ {n - 1}}{n}$
