

\section{线性代数}

\subsection{矩阵快速幂}

\subsubsection{模板}

\begin{lstlisting}[language=C++]
#define mod(x) ((x) % MOD)
const ll MOD = 1e9 + 7;
const int maxn = 1e2 + 5;
int n;// 矩阵大小
struct mat {
    ll m[maxn][maxn];
    mat operator *(const mat& a) const { //  矩阵乘法
        mat ret;
        for(int i = 0; i < n; ++i) {
            for(int j = 0; j < n; ++j) {
                ll x = 0;
                for(int k = 0; k < n; ++k) {
                    x += mod(m[i][k]*a.m[k][j]);
                    x = mod(x);
                }
                ret.m[i][j] = mod(x);
            }
        }
        return ret;
    }
} unit; //  单元矩阵
void init_unit() {
    for(int i = 0; i < n; i++) unit.m[i][i] = 1;
    return ;
}
// 调用 powMat() 前应首先初始化单位矩阵 init_unit()
mat powMat(mat a, ll x) {
    mat ret = unit;
    while(x){
        if(x&1) ret = ret*a;
        x >>= 1;
        a = a*a;
    }
    return ret;
}
\end{lstlisting}

\subsubsection{斐波那契数列第 N 项}

斐波那契数列矩阵形式

$$
M =
\begin{bmatrix}
    1 & 1 \\
    1 & 0
\end{bmatrix}
$$

$$F_{n} = M^{n} [1][0]$$

通过矩阵快速幂求解

\subsubsection{一般线性递推式}

$f_i = a_{1} f_{i - 1} + a_{2} f_{i - 2} + \cdots + a_{k} f_{i - k}$

$
A =
\begin{bmatrix}
    a_{1} & a_{2} & a_{3} & \cdots & a_{k} \\
    1 & 0 & 0 & \cdots & 0 \\
    0 & 1 & 0 & \cdots & 0 \\
    \vdots & \vdots & \vdots & \ddots & \vdots \\
    0 & 0 & \cdots & 1 & 0
\end{bmatrix}
$

\subsubsection{矩阵幂的和}

求解 $S = A + A^2 + A^3 + ...... + A^k, ( 1 <= n <= 30, 1 <= k <= 10^9, 1 <= M <= 10^4)$

令

$S_k = I + \cdots + A^{k - 1}$

则有

$$
\begin{bmatrix}
    A_{k} \\
    S_{k}
\end{bmatrix} =
\begin{bmatrix}
    A & 0\\
    I & I
\end{bmatrix} ^{k}
\begin{bmatrix}
    I \\
    0
\end{bmatrix}
$$

\begin{lstlisting}[language=C++]
#include <iostream>
#include <cstdio>
using namespace std;
const int mod = 10;
int N;
struct Mat {
    int a[85][85];
    Mat() {
        for(int i = 0; i < N; ++i) {
            for(int j = 0; j < N; ++j) {
                a[i][j] = 0;
            }
        }
    }
    Mat operator *(const Mat& rhs) const {
        Mat ret;
        for(int i = 0; i < N; ++i) {
            for(int j = 0; j < N; ++j) {
                int& x = ret.a[i][j];
                x = 0;
                for(int k = 0; k < N; ++k) {
                    x = (x + a[i][k]*rhs.a[k][j])%mod;
                }
            }
        }
        return ret;
    }
};
void init_unit(Mat& unit) {
    for(int i = 0; i < N; ++i) {
        for(int j = 0; j < N; ++j) {
            unit.a[i][j] = 0;
        }
        unit.a[i][i] = 1;
    }
}
Mat pw(Mat a, int n, Mat& unit) {
    Mat ret = unit;
    while(n > 0) {
        if(n & 1) ret = ret*a;
        a = a*a;
        n >>= 1;
    }
    return ret;
}
int main() {
    int n, k;
    while(scanf("%d%d", &n, &k) != EOF) {
        if(n == 0) break;
        N = 2*n;
        Mat a, unit;
        init_unit(unit);
        for(int i = 0; i < n; ++i) {
            for(int j = 0; j < n; ++j) {
                scanf("%d", &a.a[i][j]);
                a.a[i][j] %= mod;
            }
        }
        for(int i = 0; i < n; ++i) {
            a.a[i + n][i] = 1;
            a.a[i + n][i + n] = 1;
        }
        Mat b = pw(a, k + 1, unit);
        for(int i = 0; i < n; ++i) {
            for(int j = 0; j < n; ++j) {
                int x = b.a[i + n][j];
                if(i == j) x = (x - 1 + mod)%mod;
                if(j > 0) printf(" ");
                printf("%d", x);
            }
            puts("");
        }
        puts("");
    }
    return 0;
}
\end{lstlisting}

\subsection{BM 算法(线性递推式)}

\begin{lstlisting}[language=C++]
typedef long long ll;
#define rep(i,a,n) for (int i=a;i<n;i++)
#define per(i,a,n) for (int i=n-1;i>=a;i--)
#define pb push_back
#define mp make_pair
#define all(x) (x).begin(),(x).end()
#define fi first
#define se second
#define SZ(x) ((ll)(x).size())
typedef vector<ll> VI;
typedef pair<ll,ll> PII;
const ll mod = 1e9+7;
ll powmod(ll a,ll b) {ll res=1;a%=mod; assert(b>=0); for(;b;b>>=1){if(b&1)res=res*a%mod;a=a*a%mod;}return res;}
// head

ll _,n;
namespace linear_seq {
    const int N=10010;
    ll res[N],base[N],_c[N],_md[N];

    vector<ll> Md;
    void mul(ll *a,ll *b,ll k) {
        rep(i,0,k+k) _c[i]=0;
        rep(i,0,k) if (a[i]) rep(j,0,k) _c[i+j]=(_c[i+j]+a[i]*b[j])%mod;
        for (ll i=k+k-1;i>=k;i--) if (_c[i])
            rep(j,0,SZ(Md)) _c[i-k+Md[j]]=(_c[i-k+Md[j]]-_c[i]*_md[Md[j]])%mod;
        rep(i,0,k) a[i]=_c[i];
    }
    ll solve(ll n,VI a,VI b) {
//         a 系数 b 初值 b[n+1]=a[0]*b[n]+...
        for(int i = 0; i < a.size(); i++)
            printf("sss  %d\n", a[i]);
        ll ans=0,pnt=0;
        ll k=SZ(a);
        assert(SZ(a)==SZ(b));
        rep(i,0,k) _md[k-1-i]=-a[i];_md[k]=1;
        Md.clear();
        rep(i,0,k) if (_md[i]!=0) Md.push_back(i);
        rep(i,0,k) res[i]=base[i]=0;
        res[0]=1;
        while ((1ll<<pnt)<=n) pnt++;
        for (ll p=pnt;p>=0;p--) {
            mul(res,res,k);
            if ((n>>p)&1) {
                for (ll i=k-1;i>=0;i--) res[i+1]=res[i];res[0]=0;
                rep(j,0,SZ(Md)) res[Md[j]]=(res[Md[j]]-res[k]*_md[Md[j]])%mod;
            }
        }
        rep(i,0,k) ans=(ans+res[i]*b[i])%mod;
        if (ans<0) ans+=mod;
        return ans;
    }
    VI BM(VI s) {
        VI C(1,1),B(1,1);
        ll L=0,m=1,b=1;
        rep(n,0,SZ(s)) {
            ll d=0;
            rep(i,0,L+1) d=(d+(ll)C[i]*s[n-i])%mod;
            if (d==0) ++m;
            else if (2*L<=n) {
                VI T=C;
                ll c=mod-d*powmod(b,mod-2)%mod;
                while (SZ(C)<SZ(B)+m) C.pb(0);
                rep(i,0,SZ(B)) C[i+m]=(C[i+m]+c*B[i])%mod;
                L=n+1-L; B=T; b=d; m=1;
            } else {
                ll c=mod-d*powmod(b,mod-2)%mod;
                while (SZ(C)<SZ(B)+m) C.pb(0);
                rep(i,0,SZ(B)) C[i+m]=(C[i+m]+c*B[i])%mod;
                ++m;
            }
        }
        return C;
    }
    ll gao(VI a,ll n) {
        VI c=BM(a);
        c.erase(c.begin());
        rep(i,0,SZ(c)) c[i]=(mod-c[i])%mod;
        return solve(n,c,VI(a.begin(),a.begin()+SZ(c)));
    }
};

int main() {
    while (~scanf("%d",&n)) {
        vector<int>v;
        v.push_back(1);
        v.push_back(2);
        v.push_back(4);
        v.push_back(7);
        v.push_back(13);
        v.push_back(24);
        //VI{1,2,4,7,13,24}
        printf("%d\n", linear_seq::gao(v,n-1));
    }
}
\end{lstlisting}

\subsection{高斯消元}

高斯消元基本思想是将方程式的系数和常数化为矩阵,通过将矩阵通过行变换成为阶梯状(上三角矩阵),然后从小往上逐一求解。

\subsubsection{代数意义下}

求解下列线性方程组:

a[0][0]*x0 + a[0][1]*x1 + ... + a[0][n - 1]*x{n - 1} == a[0][n]
...
...
a[n - 1][0]*x0 + a[n - 1][1]*x1 + ... + a[n - 1][n - 1]*x{n - 1} == a[n - 1][n]

最终结果为:

x0 = a[0][n]
...
...
x{n - 1} = a[n - 1][n]

若无解,则返回 false

\begin{lstlisting}[language=C++]
double a[maxn][maxn];
bool gauss(int n) {
    for(int i = 0; i < n; ++i){
        int r = i;
        for(int j = i + 1; j < n; j++) if(fabs(a[j][i]) > fabs(a[r][i])) r = j;
/*
* 浮点高斯消元中找到最大绝对值的系数来消元本来是为了优化精度
* 如果在整数的模意义下,这种行为就没有意义了
* 此时我们只需要找到第i行及以下任意一个系数非零的元来消即可。
* for(int j = i ; j < n; j++) if(a[j][i]!=0) { r = j; break;}
*/
        if(r != i) for(int j = 0; j <= n; j++) swap(a[i][j], a[r][j]);

        for(int j = n; j >= i; --j)
            for(int k = i + 1; k < n; ++k)
                a[k][j] -= a[k][i]/a[i][i]*a[i][j];
    }
    for(int i = n - 1 ; i >= 0 ; --i){
        for(int j = i + 1 ; j < n ; j++){
            double val = a[i][j]*a[j][n];
            a[i][n] -= val;
        }
        if(fabs(a[i][i]) < 1e-10){
            if(fabs(a[i][n]) > 1e-10) return false;
            continue;
        }
        a[i][n] = a[i][n]/a[i][i];
    }
    return true;
}
\end{lstlisting}

\subsection{线性基(区间子集异或最值)}

\begin{lstlisting}[language=C++]
struct Linear_Basis {
    int tv, tb; // total variable, total base
    ll b[64], nb[64];
    void init() {
        tv = tb = 0;
        memset(b, 0, sizeof(b));
        memset(nb, 0, sizeof(nb));
    }
    bool ist(ll x){
        ++tv;
        for(int i = 62; i >= 0; --i) {
            if (x & (1LL << i)) {
                if(!b[i]) {
                    b[i] = x;
                    break;
                }
                x ^= b[i];
            }
        }
        if(x <= 0) return false;
        return true;
    }
    ll Max(ll x) {
        ll ret = x;
        for (int i = 62; i >= 0; --i) ret = max(ret, ret ^ b[i]);
        return ret;
    }
    ll Min(ll x) {
        ll ret = x;
        for (int i = 0; i <= 62; ++i)
            if (b[i]) ret ^= b[i];
        return ret;
    }
    void rebuild() {
        for (int i = 62; i >= 0; --i)
            for (int j = i - 1; j >= 0; --j)
                if (b[i] & (1LL << j)) b[i] ^= b[j];
        tb = 0;
        for (int i = 0; i <= 62; ++i)
            if (b[i]) nb[tb++] = b[i];
    }
    ll Kth_Max(ll k) {
        if (tv != tb) --k;
        if (k >= (1LL << tb)) return -1LL;
        ll ret = 0;
        for (int i = 62; i >= 0; --i)
            if (k & (1LL << i)) ret ^= nb[i];
        return ret;
    }
} LB;
\end{lstlisting}

\subsection{线性规划问题(单纯形法)}

例题 bzoj 3112: [Zjoi2013] 防守战线

战线可以看作一个长度为 n 的序列,现在需要在这个序列上建塔来防守敌
兵,在序列第 i 号位置上建一座塔有 Ci 的花费,且一个位置可以建任意多的塔
费用累加计算。有 m个区间[L1, R1], [L2, R2], …, [Lm, Rm],在第 i 个区间
的范围内要建至少 Di 座塔。求最少花费。

对其对偶问题使用单纯形法

standard:

$max z = c_0*x_0 + ... + c_{n - 1}*x_{n - 1}$

$a_{0 0}*x_0 + a_{0 1}*x_1 ... + a_{0 n - 1}*x_{n - 1} <= b_0$

$\cdots$

$a_{m - 1 0}*x_0 + a_{m 1}*x_1 ... + a_{m - 1 n - 1}*x_{n - 1} <= b_{m - 1}$


\begin{lstlisting}[language=C++]
const double M = 1e100;
const double thres = 1e50;
const double eps = 1e-6;
int sgn(double x) {
    if(fabs(x) < eps) return 0;
    if(x > 0) return 1;
    return -1;
}
namespace simplex {
    const int maxr = 2e3 + 5;
    const int maxc = 2e4 + 5;
    int row, column;
    double a[maxr][maxc], b[maxr], c[maxc], s[maxc];
    int rid[maxr], cid[maxc];
    void output() {
        puts("");
        printf("row: %d column: %d\n", row, column);
        puts("++++++++++++++++++++++++++++++++++++++++");
        printf("max z = ");
        for(int i = 0; i < column; ++i) {
            if(i > 0) printf(" + ");
            printf("%.2f*x_%d", c[i], i);
        }
        puts("");
        puts("subject to: ");
        for(int i = 0; i < row; ++i) {
            for(int j = 0; j < column; ++j) {
                if(j > 0) printf(" + ");
                printf("%.2f*x_%d", a[i][j], j);
            }
            printf(" = %.2f\n", b[i]);
        }
        puts("++++++++++++++++++++++++++++++++++++++++");
        printf("row id:");
        for(int i = 0; i < row; ++i) {
            printf(" %d", rid[i]);
        }
        puts("");
        printf("column id:");
        for(int i = 0; i < column; ++i) {
            printf(" %d", cid[i]);
        }
        puts("");
        puts("++++++++++++++++++++++++++++++++++++++++");
        double z = 0.0;
        printf("z = ");
        for(int i = 0; i < row; ++i) {
            if(i > 0) printf("+ ");
            if(rid[i] < column) {
                printf("%.2f*%.2f ", b[i], c[rid[i]]);
                z += b[i]*c[rid[i]];
            } else {
                printf("%.2f*%.2f ", b[i], 0.0);
            }
        }
        printf("= %.2f\n", z);
        puts("++++++++++++++++++++++++++++++++++++++++");
        puts("");
    }
    void init(int m, int n) {
        row = m; column = n + 1;
        c[n] = -M;
        for(int i = 0; i < row; ++i) {
            a[i][n] = -1.0;
            rid[i] = column + i;
        }
        for(int i = 0; i < column; ++i) {
            cid[i] = i;
            s[i] = c[i];
        }
        // output();
    }
    bool getRC(int &rv, int &cv) {
        rv = -1; cv = -1;
        double sigma = -M;
        for(int i = 0; i < column; ++i) {
            if(cv == -1 || s[i] > sigma) {
                sigma = s[i];
                cv = i;
            }
        }
        // printf("cv: %d sigma: %.2f\n", cv, sigma);
        if(sgn(sigma) <= 0) return 1;
        double theta = M;
        for(int i = 0; i < row; ++i) {
            if(sgn(a[i][cv]) <= 0) continue;
            double ntheta = b[i]/a[i][cv];
            if(rv == -1 || sgn(ntheta - theta) < 0) {
                theta = ntheta;
                rv = i;
            }
        }
        if(theta > thres) return 2;
        return 0;
    }
    void pivot(const int &rv, const int &cv) {
        assert(rv != -1 && cv != -1);
        b[rv] /= a[rv][cv];
        for(int i = 0; i < column; ++i) {
            if(i == cv) continue;
            a[rv][i] /= a[rv][cv];
        }
        a[rv][cv] = 1.0/a[rv][cv];
        for(int i = 0; i < row; ++i) {
            if(i == rv || sgn(a[i][cv]) == 0) continue;
            b[i] -= a[i][cv]*b[rv];
            for(int j = 0; j < column; ++j) {
                if(j == cv) continue;
                a[i][j] -= a[i][cv]*a[rv][j];
            }
            a[i][cv] = -a[i][cv];
        }
        for(int i = 0; i < column; ++i) {
            if(i == cv) continue;
            s[i] -= s[cv]*a[rv][i];
        }
        s[cv] = -s[cv]*a[rv][cv];
        swap(rid[rv], cid[cv]);
    }
    int solve(int m, int n, double &z) {
        init(m, n);
        while(true) {
            int rv, cv;
            int sta = getRC(rv, cv);
            if(sta == 2) return 2;
            if(sta == 1) break;
            pivot(rv, cv);
        }
        z = 0.0;
        for(int i = 0; i < row; ++i) {
            if(rid[i] >= column) continue;
            z += b[i]*c[rid[i]];
        }
        if(z < -thres) return 1;
        return 0;
    }
}

int main() {
    int n, m; scanf("%d%d", &n, &m);
    for(int i = 0; i < n; ++i) {
        scanf("%lf", &simplex::b[i]);
    }
    for(int i = 0; i < m; ++i) {
        int l, r; scanf("%d%d%lf", &l, &r, &simplex::c[i]);
        --l, --r;
        for(int j = 0; j < n; ++j) {
            if(j >= l && j <= r) {
                simplex::a[j][i] = 1.0;
            } else {
                simplex::a[j][i] = 0.0;
            }
        }
    }
    double ans = 0.0;
    int sta = simplex::solve(n, m, ans);
    if(sta == 0) {
        printf("%.0f\n", ans);
        // printf("max z = %.2f\n", ans);
    } else if(sta == 1) {
        assert(false);
        // printf("no feasible solution\n");
    } else {
        assert(false);
        // printf("unbounded solution. (z -> inf)\n");
    }
    return 0;
}
/*
input:
    5 3
    1 5 6 3 4
    2 3 1
    1 5 4
    3 5 2
output:
    11
*/
\end{lstlisting}
