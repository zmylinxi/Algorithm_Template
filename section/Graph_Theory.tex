

\section{图论}

\subsection{最小生成树(无向图)}

\subsubsection{Kruskal 算法}

贪心按照边权升序排序,并查集维护块的连通性

每次贪心选择当前能选择的边权最小且不成环的边

\subsubsection{性质}

树满足以下性质:

\begin{itemize}
    \item 是边数最多的无环图
    \item 是边数最少的连通图
    \item 任意两点间的简单路径唯一
\end{itemize}

最小生成树满足以下性质:

\begin{itemize}
    \item 是总边权和最小的生成树
    \item 是最大边权最小的生成树
    \item 边权各不相同的图对应的最小生成树一定唯一
    \item 存在相同边权的最小生成树可能唯一,也可能不唯一
\end{itemize}

\subsubsection{拓展}

\begin{itemize}
    \item 多目标规划的最小生成树只需将权值按照目标优先级依次比较即可
\end{itemize}

\subsubsection{生成树计数(Matrix Tree 定理)}

Matrix-Tree 定理(Kirchhoff 矩阵-树定理)

\begin{itemize}
    \item G 的度数矩阵 D[G]是一个 n*n 的矩阵,并且满足:当 i != j 时,dij=0;当 i=j 时,dij 等于 vi 的度数。
    \item G 的邻接矩阵 A[G]也是一个 n*n 的矩阵, 并且满足:如果 vi、vj 之间有边直接相连,则 aij = 1,否则为 0。
\end{itemize}

我们定义 G 的 Kirchhoff 矩阵(也称为拉普拉斯算子) C[G] 为 C[G] = D[G] - A[G]

则 Matrix-Tree 定理可以描述为:G 的所有不同的生成树的个数等于其 Kirchhoff 矩阵 C[G]任何一个 n-1 阶主子式的行列式的绝对值。

所谓 n - 1 阶主子式,就是对于 r(1 ≤ r ≤ n), 将 C[G]的第 r 行、第 r 列同时去掉后得到的新矩阵,用 Cr[G] 表示。

\begin{lstlisting}[language=C++]
struct Edge {
    int u, v, w;
} E[maxn*maxn];
ll d[maxn][maxn], a[maxn][maxn];
struct Matrix {
    ll mat[maxn][maxn];
    void init(int n) {
        for(int i = 0; i < n; ++i) {
            for(int j = 0; j < n; ++j) mat[i][j] = d[i][j] - a[i][j];
        }
    }
    ll det(int n) {
        ll res = 1;
        for(int i = 0; i < n; ++i){
            for(int j = i + 1; j < n; ++j){
                while(mat[j][i] != 0) {
                    ll t = mat[i][i]/mat[j][i]%mod;
                    for(int k = i; k < n; ++k)
                        mat[i][k] = (mat[i][k] - t*mat[j][k] + mod)%mod;
                    swap(mat[i], mat[j]);
                    res = -res;
                }
            }
            res=(res*mat[i][i])%mod;
        }
        return (res + mod)%mod;
    }
} mat;
int n, m;
void init_da() {
    for(int i = 0; i < n; ++i) {
        for(int j = 0; j < n; ++j) {
            d[i][j] = a[i][j] = 0;
        }
    }
}
void solve() {
    scanf("%d%d", &n, &m);
    init_da();
    for(int ie = 1; ie <= m; ++ie) {
        int u, v; scanf("%d%d", &u, &v);
        ++a[u][v]; ++a[v][u];
        ++d[u][u]; ++d[v][v];
    }
    mat.init(n);
    ll ans = mat.det(n - 1);
}
\end{lstlisting}

\subsubsection{最小生成树计数(Kruskal + Matrix Tree 定理)}

算法引入:

给定一个含有N个结点M条边的无向图,求它最小生成树的个数t(G);

算法思想:

抛开“最小”的限制不看,如果只要求求出所有生成树的个数,是可以利用Matrix-Tree定理解决的;

Matrix-Tree定理此定理利用图的Kirchhoff矩阵,可以在O(N3)时间内求出生成树的个数;

kruskal算法:

将图G={V,E}中的所有边按照长度由小到大进行排序,等长的边可以按照任意顺序;

初始化图G’为{V,Ø},从前向后扫描排序后的边,如果扫描到的边e在G’中连接了两个相异的连通块,则将它插入G’中;

最后得到的图G’就是图G的最小生成树;

由于kruskal按照任意顺序对等长的边进行排序,则应该将所有长度为L0的边的处理当作一个阶段来整体看待;

令kruskal处理完这一个阶段后得到的图为G0,如果按照不同的顺序对等长的边进行排序,得到的G0也是不同;

虽然G0可以随排序方式的不同而不同,但它们的连通性都是一样的,都和F0的连通性相同(F0表示插入所有长度为L0的边后形成的图);

在kruskal算法中的任意时刻,并不需要关注G’的具体形态,而只要关注各个点的连通性如何(一般是用并查集表示);

所以只要在扫描进行完第一阶段后点的连通性和F0相同,且是通过最小代价到达这一状态的,接下去都能找到最小生成树;

经过上面的分析,可以看出第一个阶段和后面的工作是完全独立的;

第一阶段需要完成的任务是使G0的连通性和F0一样,且只能使用最小的代价;

计算出这一阶段的方案数,再乘上完成后续事情的方案数,就是最终答案;

由于在第一个阶段中,选出的边数是一定的,所有边的长又都为L0;

所以无论第一个阶段如何进行代价都是一样的,那么只需要计算方案数就行了;

此时Matrix-Tree定理就可以派上用场了,只需对F0中的每一个连通块求生成树个数再相乘即可;

Matrix-Tree定理:

G的所有不同的生成树的个数等于其Kirchhoff矩阵C[G]任何一个n-1阶主子式的行列式的绝对值;

n-1阶主子式就是对于r(1≤r≤n),将C[G]的第r行,第r列同时去掉后得到的新矩阵,用Cr[G]

\begin{lstlisting}[language=C++]
const int maxn = 2e2 + 5;
const int maxm = 2e3 + 5;
struct Edge{
    int u, v;
    int w;
    bool operator < (const Edge &ed)const{
        return w < ed.w;
    }
}edge[maxm];
int n, m;
ll mod;
int fa1[maxn], fa2[maxn];//fa1, fa2都是并查集,fa2是每组边临时使用
bool vis[maxn];
ll g[maxn][maxn], kir[maxn][maxn]; //G顶点之间的关系,kir为生成树计数用的Kirchhoff矩阵
vector<int> v[maxn]; //记录每个连通分量
int findfa(int x, int f[]){
    int y = x;
    while(f[x] != x) x = f[x];
    while(f[y] != x){
        int z = f[y];
        f[y] = x;
        y = z;
    }
    return x;
}
ll det(int nn){ //生成树计数:Matrix-Tree定理
    for(int i = 0; i < nn; ++i)
        for(int j = 0; j < nn; ++j)
            kir[i][j] %= mod;
    ll ret = 1LL;
    for(int i = 1; i < nn; ++i) {
        for(int j = i + 1; j < nn; ++j)
            while(kir[j][i]) {
                ll t = kir[i][i]/kir[j][i];
                for(int k = i; k < nn; ++k) kir[i][k] = (kir[i][k] - kir[j][k]*t)%mod;
                for(int k = i; k < nn; ++k) swap(kir[i][k], kir[j][k]);
                ret = -ret;
            }
        if(kir[i][i] == 0) return 0LL;
        ret = (ret*kir[i][i])%mod;
    }
    ret = (ret + mod)%mod;
    return ret;
}
ll Solve(){
    sort(edge, edge + m); //按权值排序
    for(int i = 1; i <= n; ++i){ //初始化并查集
        fa1[i] = i;
        vis[i] = false;
    }
    int wt = -1; //记录相同的权值的边
    ll ret = 1;
    for(int k = 0; k <= m; ++k){
        if(k == m || edge[k].w != wt){ //一组相等的边,即权值都为Edge的边加完
            for(int i = 1; i <= n; ++i){
                if(vis[i]){
                    int fai = findfa(i, fa2);
                    v[fai].push_back(i);
                    vis[i] = false;
                }
            }
            for(int i = 1; i <= n; ++i){ //枚举每个连通分量
                if(v[i].size() > 1){
                    for(int a = 1; a <= n; ++a)
                        for(int b = 1; b <= n; ++b)
                            kir[a][b] = 0;
                    int len = v[i].size();
                    for(int a = 0; a < len; ++a) //构建Kirchhoff矩阵C
                        for(int b = a + 1; b < len; ++b){
                            int aa = v[i][a];
                            int bb = v[i][b];
                            kir[b][a] -= g[aa][bb];
                            kir[a][b] = kir[b][a];
                            kir[a][a] += g[aa][bb]; //连通分量的度
                            kir[b][b] += g[aa][bb];
                        }
                    ll now = det(len);
                    ret = (ret*now)%mod; //对V中的每一个连通块求生成树个数再相乘
                    for(int a = 0; a < len; ++a) fa1[v[i][a]] = i;
                }
            }
            for(int i = 1; i <= n; ++i){
                fa1[i] = findfa(i, fa1);
                fa2[i] = fa1[i];
                v[i].clear();
            }
            if(k == m) break;
            wt = edge[k].w;
        }
        int u = edge[k].u;
        int v = edge[k].v;
        int fu = findfa(u, fa1);
        int fv = findfa(v, fa1);
        if(fu == fv) continue;
        vis[fu] = vis[fv] = true;
        fa2[findfa(fu, fa2)] = findfa(fv, fa2);//并查集操作
        ++g[fu][fv];
        ++g[fv][fu];
    }
    bool flag = false;
    for(int i = 2; i <= n && !flag; ++i) if(fa2[i] != fa2[i - 1]) flag = true;
    if(m == 0) flag = true;
    if(flag) ret = 0;
    ret %= mod;
    return ret;
}
\end{lstlisting}

\subsection{堆优化的 Dijkstra 算法}

时间复杂度 $O(n \log m)$

可进行如下拓展

\begin{itemize}
    \item 多目标规划的最短路需:在用当前出堆的节点(当前能拓展的最近点),更新其他相邻节点时权值按照目标优先级依次比较,决定是否将目标入堆
    \item 最短路径数量统计需:在某个节点最短距离被缩短时重置为当前节点的方案数,当前距离相同时,累加当前节点的方案数
    \item 维护次短路需:当最短距离可以被松弛时,原最短距离更新为现最短距离,次短距离可更新为原最短距离;否则若能松弛次短距离就松弛(类似于多目标规划)
\end{itemize}

\subsection{SPFA O(玄学) 判负权环}

\subsubsection{模板}

\begin{lstlisting}[language=C++]
const int maxn = 1e5 + 5;
int n;
const int inf = 0x3f3f3f3f;
struct node{
    int to, val;
    node(int to = 0, int val = 0):to(to), val(val){}
};
bool inq[maxn];
int dist[maxn], cnt[maxn];
vector<node> g[maxn];
queue<int> q;
bool spfa(int s) {
    for (int i = 1; i <= n; ++i) {
        inq[i] = false;
        dist[i] = inf;
        cnt[i] = 0;
    }
    q.push(s); dist[s] = 0;
    inq[s] = true; ++cnt[s];
    while (!q.empty()){
        int x = q.front(); q.pop();
        inq[x] = false;
        int sz = g[x].size();
        for (int i = 0; i < sz; ++i) {
            node now = g[x][i];
            int nv = dist[x] + now.val;
            if (dist[now.to] > nv){
                dist[now.to] = nv;
                if (!inq[now.to]) {
                    inq[now.to] = true;
                    q.push(now.to);
                    if (++cnt[now.to] >= n) return false;
                }
            }
        }
    }
    return true;
}    
\end{lstlisting}

\subsubsection{差分约束系统}

$X_a - X_b \le c$

n: number of variables 变量数

m: number of constraint conditions 约束条件数

Super Source 0: $X_i - X_0 \le 0$ 超级源点 0

算法运行结束后,若无解则返回 false

否则,dist 数组里为一组非正整数解

\begin{lstlisting}[language=C++]
const int maxn = 5e5 + 5;
const ll linf = 0x3f3f3f3f3f3f3f3f;
struct node{
    int to; ll val;
    node(int to = 0, ll val = 0):to(to), val(val){}
};
bool inq[maxn];
ll dist[maxn];
int cnt[maxn], n, m;
vector<node> g[maxn];
queue<int> q;
bool spfa(int s){
    for(int i = 0; i <= n; ++i){
        inq[i] = false;
        dist[i] = linf;
        cnt[i] = 0;
    }
    q.push(s); dist[s] = 0;
    inq[s] = true; ++cnt[s];
    while(!q.empty()){
        int x = q.front(); q.pop();
        int sz = g[x].size();
        for(int i = 0; i < sz; ++i){
            node now = g[x][i];
            if(now.to == x && now.val < 0) return false;
            ll nv = dist[x] + now.val;
            if(dist[now.to] > nv){
                dist[now.to] = nv;
                if(!inq[now.to]){
                    inq[now.to] = true;
                    q.push(now.to);
                    if(++cnt[now.to] >= n) return false;
                }
            }
        }
        inq[x] = false;
    }
    return true;
}
int a[maxn], b[maxn]; ll c[maxn];
bool DCS() {
    for(int i = 1; i <= n; ++i) g[0].push_back(node(i, 0));
    for(int i = 0; i < m; ++i) g[b[i]].push_back(node(a[i], c[i]));
    if(spfa(0)) return true;
    return false;
}
\end{lstlisting}

\subsection{强连通分量(Tarjan 算法)}

\subsubsection{无向图}

\begin{lstlisting}[language=C++]
struct Node {
    int next, v;
} star[maxn<<1];
int head[maxn], tot;
int dfn[maxn], low[maxn], vis[maxn], cnt;
int s[maxn], color[maxn], top, col;
int pre[maxn];

void init(){
    tot = 0, cnt = 0;
    col = 0, top = -1;
    memset(vis, 0, sizeof(vis));
    memset(pre, 0, sizeof(vis));
    memset(head, -1, sizeof(vis));
}

void addedge(int u, int v){
    star[tot].next = head[u];
    star[tot].v = v;
    head[u] = tot++;
}

void Tarjan(int u){
    vis[u] = true;
    dfn[u] = low[u] = cnt++;
    s[++top] = u;
    for(int i = head[u]; ~i; i = star[i].next){
        int v = star[i].v;
        if(!vis[v]) {
            pre[v] = u;
            Tarjan(v);
        }
        if(v != pre[u]) low[u] = min(low[u], low[v]);
    }

    if(dfn[u] == low[u]){
        ++col;
        do{
            color[s[top]] = col;
        }while(top != -1 && s[top--] != u);
    }
}
\end{lstlisting}

\subsubsection{普通有向图转 DAG 全家桶(去重边,去自环)}

\begin{lstlisting}[language=C++]
const int maxn = 1e5 + 5;
struct Gragh {
    struct Edge {
        int to, nxt;
    } E[maxn << 1];
    int head[maxn], tot;
    void init(int n) {
        for (int i = 0; i <= n; ++i) head[i] = -1;
        tot = 0;
    }
    void add_edge(int u, int v) {
        E[tot] = {v, head[u]};
        head[u] = tot++;
    }
} G1, G2;
int dfn[maxn], low[maxn], cnt;
int st[maxn], tp, color[maxn];
bool ins[maxn];
int n, m, t, N;
void Tarjan(int x) {
    ins[x] = true;
    dfn[x] = low[x] = ++cnt;
    st[tp++] = x;
    for (int i = G1.head[x]; i != -1; i = G1.E[i].nxt) {
        int to = G1.E[i].to;
        if (dfn[to] == 0) {
            Tarjan(to);
            low[x] = min(low[x], low[to]);
        } else if (ins[to]) {
            low[x] = min(low[x], dfn[to]);
        }
    }
    if (dfn[x] == low[x]) {
        ++N;
        while (true) {
            int now = st[--tp];
            color[now] = N; ins[now] = false;
            // maintain something
            if (now == x) break;
        }
    }
}
unordered_map<int, int> vis[maxn];
void solve() {
    cnt = tp = N = 0;
    for(int i = 0; i <= n; ++i) {
        ins[i] = false;
        dfn[i] = low[i] = 0;
    }
    for(int i = 1; i <= n; ++i) {
        if(dfn[i] != 0) continue;
        Tarjan(i);
    }
    for(int i = 1; i <= n; ++i) {
        for(int j = G1.head[i]; j != -1; j = G1.E[j].nxt) {
            int to = G1.E[j].to;
            int ci = color[i], ct = color[to];
            if(ci == ct) continue;
            ++vis[ci][ct];
            if(vis[ci][ct] == 1) continue;
            G2.add_edge(ci, ct);
        }
    }
}
\end{lstlisting}

\subsubsection{在DAG的拓扑序上动态规划(BZOJ-1179 Atm)}

1 号点出发,单向边所构成的图,每个点有点权,经过某个点一次和多次只会获得一次该点的点权的收益,问最大能够获得多大收益。

\begin{lstlisting}[language=C++]
const int maxn = 5e5 + 5;
struct Edge {
    int to, nxt;
} E1[maxn << 1], E2[maxn << 1];
int head1[maxn], head2[maxn], tot1, tot2;
int n, m;
void init() {
    for(int i = 0; i <= n; ++i) {
        head1[i] = head2[i] = -1;
    }
    tot1 = tot2 = 0;

}
void addEdge1(int u, int v) {
    E1[tot1] = {v, head1[u]};
    head1[u] = tot1++;
}
void addEdge2(int u, int v) {
    E2[tot2] = {v, head2[u]};
    head2[u] = tot2++;
}
int dfn[maxn], low[maxn], cnt;
int st[maxn], color[maxn], tp, N;
bool ins[maxn], ok1[maxn], ok2[maxn];
int val1[maxn], val2[maxn];
void Tarjan(int x) {
    ins[x] = true;
    dfn[x] = low[x] = ++cnt;
    st[tp++] = x;
    for(int i = head1[x]; i != -1; i = E1[i].nxt) {
        int to = E1[i].to;
        if(dfn[to] == 0) {
            Tarjan(to);
            low[x] = min(low[x], low[to]);
        } else if(ins[to]) {
            low[x] = min(low[x], dfn[to]);
        }
    }
    if(dfn[x] == low[x]) {
        ++N;
        while(true) {
            int now = st[--tp];
            color[now] = N; ins[now] = false;
            //    这里根据题目要求维护相关信息即可
            ok2[N] |= ok1[now];
            val2[N] += val1[now];
            if(now == x) break;
        }
    }
}
unordered_map<int, bool> vis[maxn];
void Solve() {
    cnt = tp = N = 0;
    for(int i = 0; i <= n; ++i) {
        ins[i] = false;
        dfn[i] = low[i] = val2[i] = 0;
    }
    for(int i = 1; i <= n; ++i) {
        if(dfn[i] != 0) continue;
        Tarjan(i);
    }
    for(int i = 1; i <= n; ++i) {
        for(int j = head1[i]; j != -1; j = E1[j].nxt) {
            int to = E1[j].to;
            int ci = color[i], ct = color[to];
            if(ci == ct) continue;
            if(vis[ci][ct]) continue;
            vis[ci][ct] = true;
            addEdge2(ci, ct);
        }
    }
}
int sum[maxn], q[maxn], ind2[maxn], lf, rt;
void Dfs(int x) {
    for(int i = head2[x]; i != -1; i = E2[i].nxt) {
        int to = E2[i].to;
        if(ind2[to] == 0) Dfs(to);
        ++ind2[to];
    }
}
void Topo(int s) {
    lf = rt = 0;
    for(int i = 1; i <= N; ++i) ind2[i] = 0;
    Dfs(s);
    q[rt++] = s;
    while(lf < rt) {
        int x = q[lf++];
        sum[x] += val2[x];
        for(int i = head2[x]; i != -1; i = E2[i].nxt) {
            int to = E2[i].to;
            sum[to] = max(sum[to], sum[x]);
            if(--ind2[to] == 0) q[rt++] = to;
        }
    }
}
int main() {
    scanf("%d%d", &n, &m);
    init();
    for(int i = 0; i < m; ++i) {
        int x, y; scanf("%d%d", &x, &y);
        addEdge1(x, y);
    }
    for(int i = 1; i <= n; ++i) scanf("%d", &val1[i]);
    int s, p; scanf("%d%d", &s, &p);
    for(int i = 0; i < p; ++i) {
        int x; scanf("%d", &x);
        ok1[x] = true;
    }
    Solve();
    for(int i = 0; i <= n; ++i) sum[i] = 0;
    Topo(color[s]);
    int ans = 0;
    for(int i = 1; i <= n; ++i) {
        if(!ok2[i]) continue;
        ans = max(ans, sum[i]);
    }
    printf("%d\n", ans);
    return 0;
}
\end{lstlisting}

\subsection{Cayley 公式(无根树计数)}

\subsubsection{公式内容}

对于任何正整数 $n$

$n$ 个节点的带标号的无根树有 $n^{n − 2}$个

还有一种等价的说法:

$n$ 个有标号节点的无向完全图的生成树有 $n^{n − 2}$ 棵

\subsubsection{限制特定节点的度的树的计数}

限制编号为 $i$ 的点的度数为 $d_ {i}$,则树的个数有

$$\frac{(n - 2)!}{\Pi_{i = 1} ^ {n} (d_{i} - 1)!}$$

\subsubsection{n 个节点 k 棵树的森林计数问题}

令 $T _ {n, k}$ 表示有着 $n$ 个节点的 $k$ 个联通块($k$ 棵树)的森林的种类数,使得$1, 2, 3, \cdots, k$属于不同的子树

根据Cayley定理,可以推导出

$$T_{n, k} = k n^{n − k - 1}$$

例题:BZOJ 1005 [HNOI2008] 明明的烦恼

给出标号为 1 到 N ($0 < N \le 1000$)的点,以及某些点最终的度数 $D _ i$(如果对度数不要求,则输入  $-1$)

允许在任意两点间连线,可产生多少棵度数满足要求的树?

根据 Cayley公式的推广式一

$$\frac{(n − 2)!} {\Pi ^ {n} _{i = 1} (d_i − 1)!}$$

如果这道题目没有不受限制的点,那么这道问题就结束啦

可是题目里面有不受限制的点,我们应该如何处理呢

假设度数有限制的点的数量为 cnt,它们的度数为 $D_i$

令

$$sum = \sum _{i = 1} ^ {cnt} (D_i - 1)$$

那么,在序列中不同的排列的总数为:

$$C_{n - 2}^{sum} \times \frac{sum!}{ \Pi_{i = 1} ^ {cnt}(D_i - 1)!}$$

而剩下的 $n - 2 - sum$ 个位置,可以随意的排列剩余的 $n - cnt$ 个点,于是,总的方案数就应该是:

$$C_{n - 2}^{sum} \times \frac{sum!}{\Pi_{i = 1} ^ {cnt}(D_i - 1)!} \times (n - cnt)^{n - 2 - sum}$$

化简后得到:

$$\frac{(n - 2)!}{(n - 2 - sum) \Pi_{i = 1}^{cnt}(D_i - 1)!} \times (n - cnt)^{n - 2 - sum}$$

注意答案会很大,并且没有取模操作

所以需要用到大整数进行运算

代码实现

\begin{lstlisting}[language=java]
import java.math.BigInteger;
import java.util.Scanner;
public class Main {
    static int d[] = new int[1005];
    static BigInteger fac[] = new BigInteger[1005];
    public static void main(String[] args) {
    // write your code here
        Scanner in = new Scanner(System.in);
        int n = in.nextInt();
        int sum = 0, cnt = 0;
        boolean f = false;
        for(int i = 0; i < n; ++i){
            d[i] = in.nextInt();
            if(d[i] == 0 || d[i] >= n) f = true;
            if(d[i] == -1) continue;
            sum += d[i] - 1;
            ++cnt;
        }
        in.close();
        if(n == 1){
            if(!f) System.out.println(1);
            else System.out.println(0);
            return ;
        }
        if(n == 2){
            if(f || d[0] == 0 || d[1] == 0) System.out.println(0);
            else System.out.println(1);
            return ;
        }
        if(f){
            System.out.println(0);
            return ;
        }
        fac[0] = BigInteger.valueOf(1);
        for(int i = 1; i <= n; ++i) fac[i] = fac[i - 1].multiply(BigInteger.valueOf(i));
        BigInteger ans = fac[n - 2].divide(fac[n - 2 - sum]);
        for(int i = 0; i < n; ++i){
            if(d[i] == -1) continue;
            ans = ans.divide(fac[d[i] - 1]);
        }
        for(int i = 0; i < n - 2 - sum; ++i) ans = ans.multiply(BigInteger.valueOf(n - cnt));
        System.out.println(ans);
    }
}
\end{lstlisting}

\subsection{树上倍增求 LCA}

\begin{lstlisting}[language=C++]
struct edge{
    int to; ll w;
    edge(int to = 0, ll w = 0):to(to), w(w){}
};
vector<edge> g[maxn];
int fa[maxn][maxm], d[maxn];
ll dist[maxn], mxx[maxn][maxm];
void addedge(int u, int v, ll w){
    g[u].push_back(edge(v, w));
    g[v].push_back(edge(u, w));
}
void init(int n){
    for(int i = 0; i <= n; ++i){
        g[i].clear();
        for(int j = 0; j < maxm; ++j){
            mxx[i][j] = 0;
            fa[i][j] = 0;
        }
    }
}
void dfs(int x, int y){
    for(int i = 1; i < maxm; ++i){
        mxx[x][i] = max(mxx[fa[x][i - 1]][i - 1], mxx[x][i - 1]);
        fa[x][i] = fa[fa[x][i - 1]][i - 1];
    }
    int sz = g[x].size();
    for(int i = 0; i < sz; ++i){
        int to = g[x][i].to;
        ll w = g[x][i].w;
        if(to == y) continue;
        d[to] = d[x] + 1;
        dist[to]  = dist[x] + w;
        fa[to][0] = x;
        mxx[to][0] = w;
        dfs(to, x);
    }
}

ll query(int a, int b){
    int x = a, y = b;
    int lca;
    ll len, mx = 0;
    if(d[a] > d[b]) swap(a, b);
    int dt = d[b] - d[a];
    for(int i = 0; (1<<i) <= dt; i++){
        if(((dt>>i)&1) == 0) continue;
        mx = max(mx, mxx[b][i]);
        b = fa[b][i];
    }
    for(int i = maxm - 1; i >= 0; --i){
        if(fa[a][i] == fa[b][i]) continue;
        mx = max(mx, max(mxx[a][i], mxx[b][i]));
        a = fa[a][i]; b = fa[b][i];
    }
    if(a == b){
        lca = a;
    }
    else{
        lca = fa[a][0];
        mx = max(mx, max(mxx[a][0], mxx[b][0]));
    }
    len = dist[x] + dist[y] - 2LL*dist[lca];
    return mx; // len or lca or mx or mi
}
\end{lstlisting}

\subsection{树上点分治(树上合法点对(路径)计数)}

\subsubsection{写法1(使用id标记来自哪一棵子树去重)}

给一颗n个节点的树,每条边上有一个距离v。

给定k值,求有多少点对(u,v)使u到v的距离小于等于k。

\begin{lstlisting}[language=C++]
#include <cstdio>
#include <algorithm>
using namespace std;
typedef long long ll;
const ll mod = 1e9 + 7;
const int maxn = 1e4 + 5;
const int inf = 0x3f3f3f3f;
struct Edge {
    int nxt, to; ll w;
    Edge(int nxt = 0, int to = 0, ll w = 0):nxt(nxt), to(to), w(w) {}
} E[maxn << 1];
int tot, head[maxn];
int n, root, mi, nsz, atot, sz[maxn];
bool vis[maxn];
ll k, ans, cnt[maxn];
void init() {
    for(int i = 0; i <= n; ++i) {
        head[i] = -1; cnt[i] = 0;
        vis[i] = false;
    }
    tot = 0; ans = 0;
}
void addEdge(int u, int v, ll w) {
    E[tot] = Edge(head[u], v, w);
    head[u] = tot++;
}
void GetRoot(int x, int y) {
    sz[x] = 1; int mxs = 0;
    for(int i = head[x]; i != -1; i = E[i].nxt) {
        int to = E[i].to;
        if(vis[to] || to == y) continue;
        GetRoot(to, x);
        sz[x] += sz[to];
        if(sz[to] > mxs) mxs = sz[to];
    }
    if(nsz - sz[x] > mxs) mxs = nsz - sz[x];
    if(mxs < mi) mi = mxs, root = x;
}
struct node {
    int id; ll w;
    bool operator <(const node& nd) const {
        return w < nd.w;
    }
    node(int id = 0, ll w = 0):id(id), w(w) {}
} a[maxn];
void dfs(int x, int y, int id, ll w) {
    a[atot++] = node(id, w);
    for(int i = head[x]; i != -1; i = E[i].nxt) {
        int to = E[i].to; ll nw = E[i].w;
        if(vis[to] || to == y) continue;
        dfs(to, x, id, w + nw);
    }
}
void GetA(int x) {
    a[atot++] = node(x, 0);
    for(int i = head[x]; i != -1; i = E[i].nxt) {
        int to = E[i].to; ll w = E[i].w;
        if(vis[to]) continue;
        dfs(to, x, to, w);
    }
}
ll calc(int x) {
    ll ret = 0;
    atot = 0; GetA(x);
    int rt = atot - 1;
    sort(a, a + atot);
    for(int i = 0; i < atot; ++i) ++cnt[a[i].id];
    for(int lf = 0; ; ++lf) {
        --cnt[a[lf].id];
        while(lf < rt && a[lf].w + a[rt].w > k) {
            --cnt[a[rt].id];
            --rt;
        }
        if(lf >= rt) break;
        ret += rt - lf - cnt[a[lf].id];
    }
    return ret;
}
void solve(int x) {
    ans += calc(x); vis[x] = true;
    int tsz = nsz;
    for(int i = head[x]; i != -1; i = E[i].nxt) {
        int to = E[i].to; ll w = E[i].w;
        if(vis[to]) continue;
        mi = inf;
        if(sz[to] > sz[x]) nsz = tsz - sz[x];
        else nsz = sz[to];
        GetRoot(to, -1); solve(root);
    }
}
int main(){
    while(scanf("%d%lld", &n, &k) != EOF) {
        if(n == 0 && k == 0) break;
        init();
        for(int i = 1; i < n; ++i) {
            int u, v; ll w; scanf("%d%d%lld", &u, &v, &w);
            addEdge(u, v, w); addEdge(v, u, w);
        }
        nsz = n, mi = inf;
        GetRoot(1, -1); solve(root);
        printf("%lld\n", ans);
    }
    return 0;
}
\end{lstlisting}

\subsubsection{写法2(容斥去重)}

给定一棵树,有m次询问,每次询问存不存在两个点的距离为 k

\begin{lstlisting}[language=C++]
#include <cstdio>
#include <algorithm>
using namespace std;
typedef long long ll;
const ll mod = 1e9 + 7;
const int maxn = 1e4 + 5;
const int inf = 0x3f3f3f3f;
struct Edge {
    int nxt, to; ll w;
    Edge(int nxt = 0, int to = 0, ll w = 0):nxt(nxt), to(to), w(w) {}
} E[maxn << 1];
int tot, head[maxn];
int n, root, mi, nsz, atot, sz[maxn];
bool vis[maxn];
ll k, ans;
void init() {
    for(int i = 0; i <= n; ++i) head[i] = -1;
    tot = 0;
}
void addEdge(int u, int v, ll w) {
    E[tot] = Edge(head[u], v, w);
    head[u] = tot++;
}
void GetRoot(int x, int y) {
    sz[x] = 1; int mxs = 0;
    for(int i = head[x]; i != -1; i = E[i].nxt) {
        int to = E[i].to;
        if(vis[to] || to == y) continue;
        GetRoot(to, x);
        sz[x] += sz[to];
        if(sz[to] > mxs) mxs = sz[to];
    }
    if(nsz - sz[x] > mxs) mxs = nsz - sz[x];
    if(mxs < mi) mi = mxs, root = x;
}
ll a[maxn];
void dfs(int x, int y, ll w) {
    a[atot++] = w;
    for(int i = head[x]; i != -1; i = E[i].nxt) {
        int to = E[i].to; ll nw = E[i].w;
        if(vis[to] || to == y) continue;
        dfs(to, x, w + nw);
    }
}

ll calc(int x, ll w) {
    ll ret = 0, lst = 0;
    atot = 0; dfs(x, -1, w);
    sort(a, a + atot);
    int lf = 0, rt = atot - 1;
    while(lf < rt) {
        ll sum = a[lf] + a[rt];
        if(sum > k) --rt;
        else if(sum < k) ++lf;
        else {
            if(a[lf] == a[rt]) {
                ll len = rt - lf + 1;
                ret += (len - 1LL)*len/2LL;
                break;
            }
            int pl = lf, pr = rt;
            while(pl < rt && a[pl] == a[lf]) ++pl;
            while(pr > lf && a[pr] == a[rt]) --pr;
            ll lenl = pl - lf, lenr = rt - pr;
            ret += lenl*lenr;
            lf = pl, rt = pr;
        }
    }
    return ret;
}
void solve(int x) {
    ans += calc(x, 0); vis[x] = true;
    int tsz = nsz;
    for(int i = head[x]; i != -1; i = E[i].nxt) {
        int to = E[i].to; ll w = E[i].w;
        if(vis[to]) continue;
        ans -= calc(to, w);
        mi = inf;
        if(sz[to] > sz[x]) nsz = tsz - sz[x];
        else nsz = sz[to];
        GetRoot(to, -1); solve(root);
    }
}
int main(){
    while(scanf("%d", &n) != EOF) {
        if(n == 0) break;
        init();
        for(int i = 1; i <= n; ++i) {
            int x;
            while(scanf("%d", &x) != EOF) {
                if(x == 0) break;
                ll y; scanf("%lld", &y);
                addEdge(i, x, y);
                addEdge(x, i, y);
            }
        }
        ll val;
        while(scanf("%lld", &val) != EOF) {
            if(val == 0) break;
            for(int i = 1; i <= n; ++i) vis[i] = false;
            k = val; ans = 0;
            nsz = n, mi = inf;
            GetRoot(1, -1); solve(root);
            if(ans > 0) puts("AYE");
            else puts("NAY");
        }
        puts(".");
    }
    return 0;
}
\end{lstlisting}

\subsection{输出欧拉回路(路径)}

\begin{lstlisting}[language=C++]
#include <iostream>
#include <cstdio>
#include <cmath>
#include <bits/stdc++.h>

using namespace std;
typedef long long ll;
const int mod = 1e9 + 7;
const int inf = 0x3f3f3f3f;

#pragma comment(linker, "/STACK:102400000, 102400000")
const int maxn = 1e5 + 5;
const int maxm = 2e5 + 15;
struct Edge {
    int to, nxt;
} E[maxm << 1];
int head[maxn], te;
int road[maxm], tr;
int vertex[maxm], tv;
bool vis[maxm];
void init() {
    memset(head, -1, sizeof(head));
    te = 1;
}
int ind[maxn], outd[maxn];
void addEdge(int u, int v, bool undir) {
    E[te << 1] = {v, head[u]};
    head[u] = (te << 1);
    ++outd[u], ++ind[v];
    if(undir) {
        E[te << 1 | 1] = {u, head[v]};
        head[v] = (te << 1 | 1);
        ++outd[v], ++ind[u];
    }
    ++te;
}
void dfs(int x) {
    for(int &i = head[x]; i != -1; i = E[i].nxt) { // 对 head[x] 的引用是降低复杂度关键
        int id = (i >> 1), ii = i;
        if(vis[id]) continue;
        vis[id] = true;
        dfs(E[ii].to);
        if((ii & 1) == 0) road[tr++] = id;
        else road[tr++] = -id;
        if(i == -1) break;
    }
    vertex[tv++] = x;
}
int n, m;
bool Euler(bool undir) {
    int id = E[2].to;
    for(int i = 1; i <= n; ++i) {
        if(!undir && ind[i] != outd[i]) return false; // 有向图
        if(undir && (ind[i] & 1) == 1) return false; // 无向图
    }
    tv = tr = 0;
    dfs(id);
    if(tr < m) return false;
    return true;
}
int main() {
    int ty; scanf("%d", &ty);
    scanf("%d%d", &n, &m);
    init();
    memset(vis, false, sizeof(vis));
    memset(ind, 0, sizeof(ind));
    memset(outd, 0, sizeof(outd));
    bool undir = true;
    if(ty == 1) undir = true; // 无向图
    else undir = false; // 有向图
    for(int i = 0; i < m; ++i) {
        int u, v;
        scanf("%d%d", &u, &v);
        addEdge(u, v, undir);
    }
    if(Euler(undir)) {
        puts("YES");
        for(int i = tr - 1; i >= 0; --i) {
            if(i < tr - 1) printf(" ");
            printf("%d", road[i]);
        }
        puts("");
    } else {
        puts("NO");
    }
    return 0;
}
\end{lstlisting}

\subsection{多源最短路(floyd 算法)}

\subsubsection{无向图最小环}

\begin{lstlisting}[language=C++]
const int inf = 0x3f3f3f3f;
const int maxn = 1e2 + 5;
int g[maxn][maxn], dist[maxn][maxn], n, m;
int MincostCycle() {
    int ret = inf;
    for(int k = 1; k <= n; ++k) {
        for(int i = 1; i < k; ++i) {
            if(g[i][k] == inf) continue;
            for(int j = 1; j < k; ++j) {
                if(i == j) continue;
                if(g[k][j] == inf) continue;
                if(dist[j][i] == inf) continue;
                ret = min(ret, dist[j][i] + g[i][k] + g[k][j]);
            }
        }
        for(int i = 1; i <= n; ++i) {
            if(dist[i][k] == inf) continue;
            for(int j = 1; j <= n; ++j) {
                if(dist[k][j] == inf) continue;
                dist[i][j] = min(dist[i][j], dist[i][k] + dist[k][j]);
            }
        }
    }
    return ret;
}
\end{lstlisting}

\subsubsection{输出路径}

迭代输出字典序最小的路径

\begin{lstlisting}[language=C++]
const int inf = 0x3f3f3f3f;
const int maxn = 1e2 + 5;
int g[maxn][maxn], b[maxn], n;
int path[maxn][maxn];
int way[maxn], tot;
void GetPath(int a, int b) {
    while(a != b) {
        a = path[a][b];
        way[tot++] = a;
    }
}
void Floyd() {
    for(int k = 1; k <= n; ++k) {
        for(int i = 1; i <= n; ++i) {
            if(i == k) continue;
            if(g[i][k] == inf) continue;
            for(int j = 1; j <= n; ++j) {
                if(i == j || j == k) continue;
                if(g[k][j] == inf) continue;
                int w = g[i][k] + g[k][j] + b[k];
                if(w < g[i][j]) {
                    g[i][j] = w;
                    path[i][j] = path[i][k];
                } else if(w == g[i][j] && path[i][k] < path[i][j]) {
                    path[i][j] = path[i][k];
                }
            }
        }
    }
}
\end{lstlisting}

\subsection{三元环计数}

求 $N$ 个点, $M$ 条边的无向图中,三元环的个数

将点按照度数为第一关键字,标号为第二关键字从小到大排序,定义排序后每个点的序为 $pos[x]$

对于每条无线边,定向为:pos 小的点连向 pos 大的点

这样连完之后,每个点的出度最多只有 $\sqrt{M}$

然后,依pos从小到大枚举验证三元组即可,总时间复杂度 $O(M\sqrt{M})$

\begin{lstlisting}[language=C++]
typedef pair<int, int> pii;
const int maxn = 1e5 + 5;
const int maxm = 2e5 + 5;
struct Edge {
    int to, nxt;
} E[maxm];
int head[maxn], tot;
int u[maxm], v[maxm]; // input: id \in [0, n)
int cnt[maxm]; // cnt[i]: how many triples the i-th edge involve in
int ind[maxn], vis[maxn], frm[maxn];
void init_edge(int n) {
    for(int i = 0; i <= n; ++i) {
        head[i] = -1;
    }
    tot = 0;
}
void add_edge(int u, int v) {
    E[tot] = {v, head[u]};
    head[u] = tot++;
}
int cmp(const int &x, const int &y) {
    if(ind[x] == ind[y]) {
        return x - y;
    }
    return ind[x] - ind[y];
}
void get_triples(int n, int m) {
    init_edge(n);
    for(int i = 1; i <= n; ++i) {
        ind[i] = 0; vis[i] = 0;
    }
    for(int i = 0; i < m; ++i) {
        cnt[i] = 0;
        ++ind[u[i]]; ++ind[v[i]];
    }
    for(int i = 0; i < m; ++i) {
        const int x = u[i], y = v[i];
        if(x == y) { // self-loop: depends with the problem
            // assert(false);
            continue;
        }
        if(cmp(x, y) > 0) {
            add_edge(x, y);
        } else {
            add_edge(y, x);
        }
    }
    for(int i = 1; i <= n; ++i) {
        const int x = i;
        for(int j = head[x]; j != -1; j = E[j].nxt) {
            const int y = E[j].to;
            vis[y] = x; frm[y] = j;
        }
        for(int j = head[x]; j != -1; j = E[j].nxt) {
            const int y = E[j].to;
            for(int k = head[y]; k != -1; k = E[k].nxt) {
                const int z = E[k].to;
                if(vis[z] == x) {
                    // find a triple edge(a, b, c)
                    const int a = j;
                    const int b = k;
                    const int c = frm[z];
                    ++cnt[a]; ++cnt[b]; ++cnt[c];
                }
            }
        }
    }
    // calculate something with array: cnt[]
}
\end{lstlisting}